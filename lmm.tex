\chapter{LIBOR market model}

\section{Introduction}

% Under numeraire $P_n$, 
% \begin{equation}
%   dL_i = - L_i \sum_{j=i+1}^{n-1} \frac{\delta_j L_j}{1+\delta_j L_j} \sigma_j^T
% \sigma_i dt + L_i \sigma_i^T dW^{(n)}.
% \end{equation}
% 
% Under spot LIBOR measure $Q^L$, we have
% \begin{equation}
%   dL_i = L_i \sum_{j=m(t)}^{i} \frac{\delta_j L_j}{1+\delta_j L_j} \sigma_i^T
% \sigma_j dt + L_i \sigma_i^T dW^L.
% \end{equation}

To fix notation, let $T_0,T_1,\dots,T_n$ be the tenors, and $\delta_i$ be the 
year fraction for time period $[T_{i-1},T_i]$ and 
\begin{equation}
  L_t^i = L_t[T_{i-1},T_i] = \frac{D_{tT_{i-1}} - D_{tT_i}}{\delta_i D_{tT_i}}
\end{equation}
be the LIBOR rate for period $[T_{i-1},T_i]$ determined at time $t$, where
$D_{tT_i}$ is the discount factor for time period $[t,T_i]$. We also define
\begin{equation}
  \Pi_t^i = \prod_{j=1}^i (1+\delta_j L_t^j),
\end{equation}
and index $m(t)$ to be such that
\[
  T_{m(t)-1} \leq t < T_{m(t)}.
\]


%%%%%%%%%%%%%%%%%%%%%%%%%%%%%
\section{Forward Measure}
We use $D_{tT_n}$ as the numeraire 
\footnote{Adapted from Hunt and Kennedy, Financial Derivatives in Theory and
  Practice, revised ed., Wiley(2004), 18.2, pp.338-343, 
  Glasserman, Monte Carlo Methods in Financial Engineering, 3.7, p171}
and define
\begin{equation}
  \hat{D}_t^i = \frac{D_{tT_i}}{D_{tT_n}}, \qquad i<n.
\end{equation}
Hence $\hat{D}_t^i$ is a martingale under the measure $Q$ for numeraire
$D_{tT_n}$.

Let the dynamcs of LIBOR rate $L_t^i$ under measure $Q$ be
\begin{equation}
  dL_t^i = \mu_t^i L_t^i dt + \sigma_t^i L_t^i dW_t^i,
\end{equation}
where 
\begin{equation}
  dW_t^i dW_t^j = \rho_{ij}.
\end{equation}
Using Ito's Lemma and the fact that
\[
  \hat{D}_t^{i-1} = \hat{D}_t^{i} (1+\delta_i L_t^i)
\]
we have
\[
  d\hat{D}_t^{i-1} 
    = d\hat{D}_t^{i} (1+\delta_i L_t^i) + \hat{D}_t^{i} \delta_i dL_t^i
    = d\hat{D}_t^{i} (1+\delta_i L_t^i) 
      + \hat{D}_t^{i} \delta_i L_t^i \sigma_t^i dW_t^i.
\]
Multiply both sides by $\Pi_t^{i-1}$ and use the fact that
$\Pi_t^i=\Pi_t^{i-1} (1+\delta_i L_t^i)$ we get
\[
  \Pi_t^{i-1} d\hat{D}_t^{i-1} 
    = \Pi_t^{i} d\hat{D}_t^{i} 
      + \Pi_t^{i} \hat{D}_t^{i} \frac{\delta_i L_t^i}{1+\delta_i L_t^i} 
        \sigma_t^i dW_t^i
\]
Apply this equation recursively we get
\[
  \Pi_t^{i} d\hat{D}_t^{i} 
    = \Pi_t^n d\hat{D}_t^{n}
     +\sum_{j=i+1}^n (\Pi_t^j d\hat{D}_t^{j}) 
       \frac{\delta_j L_t^j}{1+\delta_j L_t^j} \sigma_t^j dW_t^j.
\]
Since $\hat{D}_t^{n}=1$ and
\[
  \Pi_t^i \hat{D}_t^{i}
    = \Pi_t^{i+1} \hat{D}_t^{i+1}
    = \Pi_t^j \hat{D}_t^{j},
\]
for each $j$, 
we have 
\begin{equation}
  d\hat{D}_t^{i} 
    = \hat{D}_t^{i} \sum_{j=i+1}^n
       \frac{\delta_j L_t^j}{1+\delta_j L_t^j} \sigma_t^j dW_t^j.
\end{equation}

Now
\[
  \delta_i L_t^i \hat{D}_t^{i} =  \hat{D}_t^{i-1} - \hat{D}_t^{i}  
\]
is also a $Q$-martingale, hence its drift is zero. Applying Ito's Lemma we
get
\begin{align*}
  d( L_t^i \hat{D}_t^{i} )
    &= L_t^i (d\hat{D}_t^{i}) + (dL_t^i) \hat{D}_t^{i} 
      +  (dL_t^i) (d\hat{D}_t^{i})   \\
    &= \sum_j (\dots) dW_t^j + \mu_t^i L_t^i \hat{D}_t^{i} dt
      + (\sigma_t^i L_t^i dW_t^i) 
        \left( \hat{D}_t^{i} \sum_{j=i+1}^n
          \frac{\delta_j L_t^j}{1+\delta_j L_t^j} \sigma_t^j dW_t^j \right) \\
    &= \sum_j (\dots) dW_t^j 
      + dt (L_t^i \hat{D}_t^{i}) 
       \left( \mu_t^i 
          + \sigma_t^i \sum_{j=i+1}^n \frac{\delta_j L_t^j}{1+\delta_j L_t^j} \sigma_t^j \rho_{ij}
       \right).
\end{align*}
Since the drift is zero we get
\begin{equation}
  \mu_t^i = - \sigma_t^i \sum_{j=i+1}^n \frac{\delta_j L_t^j}{1+\delta_j L_t^j} \sigma_t^j \rho_{ij},
\end{equation}
and thus
\begin{equation}
  dL_t^i 
    = - L_t^i \sigma_t^i 
      \sum_{j=i+1}^n \frac{\delta_j L_t^j}{1+\delta_j L_t^j} \sigma_t^j \rho_{ij} dt
      + L_t^i \sigma_t^i dW_t^i.
\end{equation}


%%%%%%%%%%%%%%%%%%%%%%%%%%%%%
\section{Spot Measure}
Alternatively we can define an instrument
\begin{equation}
  B^*(t) = D_{tT_{m(t)}} \prod_{j=1}^{m(t)-1} (1+\delta_j L_{T_{j-1}}^j)
\end{equation}
and use it as the numeraire. And we define
\begin{equation}
  \hat{D}_t^i = \frac{D_{tT_i}}{B^*_t}.
\end{equation}
Hence $\hat{D}_t^i$ is a martingale under the measure $Q$ for numeraire
$B^*(t)$.

Let the dynamcs of LIBOR rate $L_t^i$ under measure $Q$ be
\begin{equation}
  dL_t^i = \mu_t^i L_t^i dt + \sigma_t^i L_t^i dW_t^i,
\end{equation}
where 
\begin{equation}
  dW_t^i dW_t^j = \rho_{ij}.
\end{equation}
Using Ito's Lemma and the fact that
\[
  \hat{D}_t^{i-1} = \hat{D}_t^{i} (1+\delta_i L_t^i)
\]
we get
\[
  d\hat{D}_t^{i} 
    = d\hat{D}_t^{i-1} \frac{1}{1+\delta_i L_t^i} 
      + \hat{D}_t^{i-1} d\left( \frac{1}{1+\delta_i L_t^i} \right)
    = d\hat{D}_t^{i-1} \frac{1}{1+\delta_i L_t^i} 
      + \frac{\hat{D}_t^{i-1} \delta_i L_t^i \sigma_t^i dW_t^i}{(1+\delta_i L_t^i)^2} 
\]
Multiply both sides by $\Pi_t^{i}$ and use the fact that
$\Pi_t^i=\Pi_t^{i-1} (1+\delta_i L_t^i)$ we get
\[
  \Pi_t^{i} d\hat{D}_t^{i} 
    = \Pi_t^{i-1} d\hat{D}_t^{i-1} 
      + \Pi_t^{i-1} \hat{D}_t^{i-1} \frac{\delta_i L_t^i}{1+\delta_i L_t^i} 
        \sigma_t^i dW_t^i
\]
Apply this equation recursively we get
\[
  \Pi_t^{i} d\hat{D}_t^{i} 
    = \Pi_t^{m(t)-1} d\hat{D}_t^{m(t)-1}
     -\sum_{j=m(t)}^i (\Pi_t^{j-1} d\hat{D}_t^{j-1}) 
       \frac{\delta_j L_t^j}{1+\delta_j L_t^j} \sigma_t^j dW_t^j.
\]
Since $d\hat{D}_t^{m(t)-1}=0$ and
\[
  \Pi_t^i d\hat{D}_t^{i}
    = \Pi_t^{i+1} d\hat{D}_t^{i+1}
    = \Pi_t^j d\hat{D}_t^{j},
\]
for each $j$, 
we have 
\begin{equation}
  d\hat{D}_t^{i} 
    = - \hat{D}_t^{i} \sum_{j=m(t)}^i
        \frac{\delta_j L_t^j}{1+\delta_j L_t^j} \sigma_t^j dW_t^j.
\end{equation}

Now
\[
  \delta_i L_t^i \hat{D}_t^{i} =  \hat{D}_t^{i-1} - \hat{D}_t^{i}  
\]
is also a $Q$-martingale, hence its drift is zero. Applying Ito's Lemma we
get
\begin{align*}
  d( L_t^i \hat{D}_t^{i} )
    &= L_t^i (d\hat{D}_t^{i}) + (dL_t^i) \hat{D}_t^{i} 
      +  (dL_t^i) (d\hat{D}_t^{i})   \\
    &= \sum_j (\dots) dW_t^j + \mu_t^i L_t^i \hat{D}_t^{i} dt
      + (\sigma_t^i L_t^i dW_t^i) 
        \left( - \hat{D}_t^{i} \sum_{j=m(t)}^i
          \frac{\delta_j L_t^j}{1+\delta_j L_t^j} \sigma_t^j dW_t^j \right) \\
    &= \sum_j (\dots) dW_t^j 
      + dt (L_t^i \hat{D}_t^{i}) 
       \left( \mu_t^i 
          - \sigma_t^i \sum_{j=m(t)}^i \frac{\delta_j L_t^j}{1+\delta_j L_t^j} \sigma_t^j \rho_{ij}
       \right).
\end{align*}
Since the drift is zero we get
\begin{equation}
  \mu_t^i = \sigma_t^i \sum_{j=m(t)}^i \frac{\delta_j L_t^j}{1+\delta_j L_t^j} \sigma_t^j \rho_{ij},
\end{equation}
and thus under the spot measure
\begin{equation}
  dL_t^i 
    = L_t^i \sigma_t^i 
      \sum_{j=m(t)}^i \frac{\delta_j L_t^j}{1+\delta_j L_t^j} \sigma_t^j \rho_{ij} dt
      + L_t^i \sigma_t^i dW_t^i.
\end{equation}

%d\hat{D}_t^{i-1}
