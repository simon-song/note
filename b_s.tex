\chapter{Black-Scholes Formula}

%%%%%%%%%%%%%%%%%%%%%%%%%%%%%%%%%%
\section{Facts of Ito's formula}

We may symbolically compute
\footnote{Evans,An Introduction to Stochastic Differential Equations,
          version 1.2, p.75}
\begin{equation}
  d(u(X_t,t)) = \frac{\partial u}{\partial t} dt
               + \sum_{i=1}^{n} \frac{\partial u}{\partial x_i} dX^i
               + \frac{1}{2} \sum_{i,j=1}^n 
                   \frac{\partial^2 u}{\partial x_i \partial x_j} dX^i dX^j,
\end{equation}
and then simplify the term $dX^i dX^j$ by expanding it out and using the 
formal multiplication rules
\[
  (dt)^2=0, \quad dt dW^k=0, \quad dW^k dW^l=\delta_{kl} dt.
\]


%%%%%%%%%%%%%%%%%%%%%%%%%%%%%%%%%
\section{Black-Scholes Formula}

Suppose the stock price $S_t$ and the price of zero-coupon bond $D_t$ satisfies
\[
  dS_t = S_t (\mu dt + \sigma dW_t),
\]
and 
\[
  dD_t = D_t r dt,
\]
where $r$ is the constant interest rate and $D_T=1$ at maturity $T$.
And let $V$ be a call-option that pays $V_T=(S_T-K)^+$ where $K>0$ is the 
strike price. 

Use the zero-coupon bond $D_t$ as the numeraire, then $S_t/D_t$ is a martingale
under this measure $Q$. By Ito's lemma, we have
\begin{align*}
  d(S_t D_t^{-1}) &= S_t dD_t^{-1} + D_t^{-1} dS_t + dD_t^{-1} dS_t \\
                  &= S_t (-D_t^{-1} rdt) + D_t^{-1} S_t (\mu dt + \sigma dW_t),
\end{align*}
and since $S_t D_t^{-1}$ is a martingale under the measure with numeraire 
$D_t$, the drift is zero, and we have
\[
  d(S_t D_t^{-1}) = (S_t D_t^{-1}) \sigma dW_t.
\]
Solving this SDE we get for $T>t$
\[
  \frac{S_T}{D_T} = \frac{S_t}{D_t} 
                    e^{\sigma (W_T-W_t) - \frac{1}{2}\sigma^2 (T-t)}.
\]

Now $V_t/D_t$ is also a martingale under measure $Q$, i.e.
\[
  \frac{V_t}{D_t} = E_Q\left[ \frac{V_T}{D_T} \bigg| \mathcal{F}_t \right]
    = E_Q\left[ \frac{(S_T-K)^+}{D_T} \bigg| \mathcal{F}_t \right].
\]
Recall that $D_T=1$, we thus have
\[
  V_t =  E_Q
    \left[ 
      \left( S_t e^{\sigma (W_T-W_t) -\frac{1}{2} \sigma^2 (T-t)} - K D_t \right)^+ 
    \right],
\]
and since $W_T-W_t\sim N(0,T-t)$, we get
\begin{align*}
  V_t 
   &= \int_{-\infty}^{\infty} dx \, \frac{1}{\sqrt{2\pi (T-t)}} 
        e^{-\frac{x^2}{2(T-t)}} 
      \left( S_t e^{\sigma x -\frac{1}{2} \sigma^2 (T-t)} - K D_t \right)^+  \\
   &= \int_{\frac{\ln (KD_t/S_t)}{\sigma} + \frac{\sigma}{2}(T-t) }^{\infty} 
        dx \, \frac{1}{\sqrt{2\pi (T-t)}} 
      \left( S_t e^{\sigma x  -\frac{x^2}{2(T-t)} -\frac{1}{2} \sigma^2 (T-t)} 
        - K D_t e^{-\frac{x^2}{2(T-t)}} \right)       \\
   &= \int_{\frac{\ln (KD_t/S_t)}{\sigma} + \frac{\sigma}{2}(T-t) }^{\infty} 
        dx \, \frac{1}{\sqrt{2\pi (T-t)}} 
      \left( S_t e^{ -\frac{(x-\sigma (T-t))^2}{2(T-t)} } 
        - K D_t e^{-\frac{x^2}{2(T-t)}} \right), 
\end{align*}
let the cumulative distribution function of the normal distribution be
\begin{equation}
  \Phi(x)= \int_{-\infty}^x \frac{e^{-t^2/2}}{\sqrt{2\pi}} dt,
\end{equation}
we get the Black-Scholes price of the European call option on stocks
\begin{equation}
  V_t = 
    S_t \Phi 
    \left(  
      \frac{\ln (S_t/KD_t)}{\sigma \sqrt{T-t}} + \frac{1}{2} \sigma \sqrt{T-t}  
    \right)
    - K D_t \Phi 
    \left(  
      \frac{\ln (S_t/KD_t)}{\sigma \sqrt{T-t}} - \frac{1}{2} \sigma \sqrt{T-t}  
    \right).
\end{equation}

%%%%%%%%%%%%%%%%
\begin{remark}
	We can intepret the Black-Scholes formula the following way (note that
	$D_T=1$):
	\footnote{See also Fabrice Rouah, The Heston Model and Its Extensions in 
	Matlab and C\#, pp. 4-5}
	\begin{align*}
		C_t &= D_t \cdot E_Q\left[ \frac{(S_T-K)^+}{D_T}  \bigg| \mathcal{F}_t
	                      \right] \\
   			&= D_t \cdot E_Q\left[ 
 												  \frac{S_T-K}{D_T} 1_{S_T>K} \bigg| \mathcal{F}_t
	                      \right] \\
   			&= D_t \cdot E_Q\left[ 
 												  \frac{S_T}{D_T} 1_{S_T>K} \bigg| \mathcal{F}_t
	                      \right] 
					 - K D_t \cdot E_Q[1_{S_T>K} | \mathcal{F}_t] \\
   			&= S_t \cdot E_Q\left[ 
 												  \frac{S_T/S_t}{D_T/D_t} 1_{S_T>K} \bigg| \mathcal{F}_t
	                      \right] 
					 - K D_t \cdot E_Q[1_{S_T>K} | \mathcal{F}_t] \\
			  &= S_t \cdot E_{Q_S}[1_{S_T>K} | \mathcal{F}_t] 
					 - K D_t \cdot E_Q[1_{S_T>K} | \mathcal{F}_t] 
		% E_Q\left[   \bigg| \mathcal{F}_t \right]
	\end{align*}
	where we have use the fact that the Radon-Nikodym derivative between the two
	measures $Q_S$ and $Q$ using $S_t$ and $D_t$ as numeraires respectively is:
	\footnote{See Remark \ref{R:radon_num}}
	\[
		\frac{dQ_S}{dQ} = \frac{S_T/S_t}{D_T/D_t}.
	\]
\end{remark}


%%%%%%%%%%%%%%%%%%%%%%%%%%%%%%%%%%%%%%%%%%%%%%%%%%
\section{Black-Scholes with Continuous Dividend}
Suppose the underlying is paying a continuous dividend, and let the dividend
discount factor at time $t$ be $D_t^d$, then the price of the call option
matures at $T$ with strike $K$ is
\begin{equation} \label{E:bs_div}
  C(t,T,\sigma,S_t,K) = S_t D_t^d \Phi(d_1) - K D_t \Phi(d_2),
\end{equation}
where
\[
  d_{1,2}=\frac{\ln\left( \frac{S_t D_t^d}{K D_t} \right)}{\sigma\sqrt{T-t}}
          \pm \frac{1}{2} \sigma\sqrt{T-t}.
\]
while the price of the put option is 
\begin{equation}
  P(t,T,\sigma,S_t,K) = - S_t D_t^d \Phi(-d_1) + K D_t \Phi(-d_2).
\end{equation}

To calculate the greeks, first note that
\begin{align*}
  \frac{\partial C_t}{\partial x} 
    &= \frac{\partial}{\partial x} (S_t D_t^d)\, \Phi(d_1)   
      - \frac{\partial}{\partial x} (K D_t)\, \Phi(d_2)   
      + S_t D_t^d \frac{\partial}{\partial x} \Phi(d_1)   
      - K D_t \frac{\partial}{\partial x} \Phi(d_2) \\
    &= \frac{\partial}{\partial x} (S_t D_t^d)\, \Phi(d_1)   
      - \frac{\partial}{\partial x} (K D_t)\, \Phi(d_2)   
      + S_t D_t^d \phi(d_1) \frac{\partial d_1}{\partial x} 
      - K D_t \phi(d_2) \frac{\partial d_2}{\partial x} 
\end{align*}
where
\[
  \phi(x)=\frac{e^{-x^2/2}}{\sqrt{2\pi}}.
\]
It is easy to check that 
\[
  S_t D_t^d \phi(d_1) = K D_t \phi(d_2),
\]
hence
\begin{align*}
  \frac{\partial C_t}{\partial x} 
    &= \frac{\partial}{\partial x} (S_t D_t^d)\, \Phi(d_1)   
      - \frac{\partial}{\partial x} (K D_t)\, \Phi(d_2)   
      + S_t D_t^d \phi(d_1) \frac{\partial}{\partial x} (d_1-d_2) \\
    &= \frac{\partial}{\partial x} (S_t D_t^d)\, \Phi(d_1)   
      - \frac{\partial}{\partial x} (K D_t)\, \Phi(d_2)   
      + S_t D_t^d \phi(d_1) \frac{\partial}{\partial x} (\sigma\sqrt{T-t}).
\end{align*}

Similarly we have
\[
  \frac{\partial P_t}{\partial x} 
    = -\frac{\partial}{\partial x} (S_t D_t^d)\, \Phi(-d_1)   
      + \frac{\partial}{\partial x} (K D_t)\, \Phi(-d_2)   
      + S_t D_t^d \phi(d_1) \frac{\partial}{\partial x} (\sigma\sqrt{T-t}).
\]

The deltas are: 
\begin{align}
	\frac{\partial C_t}{\partial S_t} &= D_t^d \Phi(d_1)   \\
	\frac{\partial P_t}{\partial S_t} &= - D_t^d \Phi(-d_1)
\end{align}

The gammas are:
\begin{align}
  \frac{\partial^2 C_t}{\partial S_t^2} 
	  &= \frac{D_t^d \Phi'(d_1)}{S_t \sigma \sqrt{T-t}}  \\
  \frac{\partial^2 P_t}{\partial S_t^2} 
    &= \frac{D_t^d \Phi'(d_1)}{S_t \sigma \sqrt{T-t}}
\end{align}

The thetas are:
\begin{align}
	\frac{\partial C_t}{\partial t} 
	  &= - S_t  D_t^d \Phi'(d_1) \frac{\sigma}{2\sqrt{T-t}} + q_t S_t \Phi(d_1) D_t^d 
  	   - r_t K D_t \Phi(d_2)    \\
	\frac{\partial P_t}{\partial t} 
	  &= - S_t  D_t^d \Phi'(d_1) \frac{\sigma}{2\sqrt{T-t}} - q_t S_t \Phi(-d_1) D_t^d 
  	   + r_t K D_t \Phi(-d_2)  
\end{align}

The vegas are:
\begin{align}
	\frac{\partial C_t}{\partial \sigma} &= S_t D_t^d \Phi'(d_1) \sqrt{T-t}   \\
	\frac{\partial P_t}{\partial \sigma} &= S_t D_t^d \Phi'(d_1) \sqrt{T-t}
\end{align}

The rhos are:
\begin{align}
	\frac{\partial C_t}{\partial r_t} &= K D_t \Phi(d_2) (T-t)    \\
	\frac{\partial P_t}{\partial r_t} &= - K D_t \Phi(-d_2) (T-t)
\end{align}
where
\[
	\Phi'(x)=\frac{e^{-x^2/2}}{\sqrt{2\pi}}
\]
is the density function of the normal distribution,
and
\[
	r_t = -\frac{\ln(D_t)}{T-t},
\]
\[
	q_t = -\frac{\ln(D_t^d)}{T-t}.
\]


%%%%%%%%%%%%%%%%%%%%%%%%%%%%%%%%%%%%%%%%%%%%%%%%%%
\section{Black-Scholes with Delayed Payment}
In this section we consider vanilla options of which the payoff is determined at
exercise date $\tau$ but paid later at time $T>\tau$.
Note that for any instrument exercised at $\tau$ and paid at $T>\tau$
the time $t$ value can be written as
\begin{align*}
	V_t(\tau,T) &= D_{tT} E_Q[V_T|F_t]    \\
				 &= D_{tT} E_Q[E_Q[V_T|F_{\tau}]|F_t]  \\
				 &= D_{tT} E_Q[V_{\tau}|F_t]  \\
				 &= D_{\tau T} D_{t\tau} E_Q[V_{\tau}|F_t]  \\
				 &= D_{\tau T} V_t(\tau,\tau).
\end{align*}

Thus the price of the call option with payoff $V_T=(S_{\tau}-K)^+$ is
\begin{equation} 
	C(t,\tau, T,\sigma,S_t,K) = S_t D_{\tau T} D_{t\tau}^d \Phi(d_1) - K D_{tT} \Phi(d_2),
\end{equation}
where
\[
	d_{1,2}=
  	\frac{\ln\left( \frac{S_t D_{t\tau}^d}{K D_{t\tau}}	\right)}{\sigma\sqrt{\tau-t}}
          \pm \frac{1}{2} \sigma\sqrt{\tau-t}.
\]
while the price of the put option with payoff $V_T=(K-S_{\tau})^+$ is 
\begin{equation}
	P(t,\tau,T,\sigma,S_t,K) = - S_t D_{\tau T} D_{t\tau}^d \Phi(-d_1) + K D_{tT} \Phi(-d_2).
\end{equation}

The deltas are: 
\begin{align}
	\frac{\partial C(t,\tau,T)}{\partial S_t} &=   D_{\tau T} D_{t\tau}^d \Phi(d_1)   \\
	\frac{\partial P(t,\tau,T)}{\partial S_t} &= - D_{\tau T} D_{t\tau}^d \Phi(-d_1)
\end{align}

The gammas are:
\begin{align}
	\frac{\partial^2 C(t,\tau,T)}{\partial S_t^2} 
  	&= \frac{D_{\tau T} D_{t\tau}^d \Phi'(d_1)}{S_t \sigma \sqrt{\tau-t}}  \\
	\frac{\partial^2 P(t,\tau,T)}{\partial S_t^2} 
  	&= \frac{D_{\tau T} D_{t\tau}^d \Phi'(d_1)}{S_t \sigma \sqrt{\tau-t}}
\end{align}

The thetas are:
\begin{align}
	\frac{\partial C_t}{\partial t} 
	  &= - S_t  D_{\tau T} D_{t\tau}^d \Phi'(d_1) \frac{\sigma}{2\sqrt{\tau-t}} 
     	 + q_{t\tau} S_t \Phi(d_1) D_{\tau T} D_{t\tau}^d  
			 - r_{t\tau} K D_{tT} \Phi(d_2)    \\
	\frac{\partial P_t}{\partial t} 
	&= - S_t D_{\tau T} D_{t\tau}^d \Phi'(d_1) \frac{\sigma}{2\sqrt{\tau-t}} 
	   - q_{t\tau} S_t \Phi(-d_1) D_{\tau T} D_{t\tau}^d 
	   + r_{t\tau} K D_{tT} \Phi(-d_2)  
\end{align}

The vegas are:
\begin{align}
	\frac{\partial C(t,\tau,T)}{\partial \sigma} &= S_t D_{\tau T} D_{t\tau}^d
  	\Phi'(d_1) \sqrt{\tau-t}   \\
	\frac{\partial P(t,\tau,T)}{\partial \sigma} &= S_t D_{\tau T} D_{t\tau}^d
		\Phi'(d_1) \sqrt{\tau-t}
\end{align}

% The rhos are:
% \begin{align}
% 	\frac{\partial C_t}{\partial r_t} &= K D_t \Phi(d_2) (T-t)    \\
% 	\frac{\partial P_t}{\partial r_t} &= - K D_t \Phi(-d_2) (T-t)
% \end{align}
where
\[
	\Phi'(x)=\frac{e^{-x^2/2}}{\sqrt{2\pi}}
\]
is the density function of the normal distribution,
and for $s>t$
\[
	r_{ts} = -\frac{\ln(D_{ts})}{s-t},
\]
\[
	q_{ts} = -\frac{\ln(D_{ts}^d)}{s-t}.
\]

The rhos are not well defined in this case, should it be a risk against the
movement of the whole discount rate curve?


%%%%%%%%%%%%%%%%%%%%%%%%%%%%%%%%%%%%%%%%%%%%%%%%%%
\section{Black76 Model for Futures Option}
Under the Black76 model, the underlying futures price is a martingale under the
risk-neutral measure, and the price of a call option with exercise date $T$ on a
future matures at $T_1\ge T$ with strike $K$ is:
\begin{equation}
	C_{fut}(t,T,T_1,\sigma,F_t(T_1),K) 
	  = D_t [F_t(T_1) \Phi(d_1) - K \Phi(d_2)],
\end{equation}
where
\[
	d_{1,2}= \frac{\ln(F_t(T_1)/K)}{\sigma\sqrt{T-t} }
		         \pm \frac{1}{2}\sigma\sqrt{T-t}.
\]
while the price of the put option is
\begin{equation}
	P_{fut}(t,T,T_1,\sigma,F_t(T_1),K) 
	  = D_t [-F_t(T_1) \Phi(-d_1) + K \Phi(-d_2)].
\end{equation}

Note that if $T_1=T$ the price of the call option on spot and the call option on
futures are the same:
\[
	C_{spot}(t,T,\sigma,S_t,K) = C_{fut}(t,T,T,\sigma,F_t(T),K),
\]
since
\[
	F_t(T) = D_t^d S_t / D_t.
\]
Note that the forward price $F_t(T_1)$ is a martingale under the risk-neutral
measure.
\footnote{Geman, Commodities and Commodity Derivatives, section 5.3, pp.99-100;
Hull, Options, Futures, and Other Derivatives, 9ed, section 18.7, pp.391-392}

The deltas are: 
\begin{align}
	\frac{\partial C_{fut}}{\partial F_t(T_1)} &= D_t \Phi(d_1)   \\
	\frac{\partial P_{fut}}{\partial F_t(T_1)} &= - D_t \Phi(-d_1).
\end{align}

The gammas are: 
\begin{align}
	\frac{\partial^2 C_{fut}}{\partial F_t^2(T_1)} 
  	&= D_t \frac{\Phi'(d_1)}{F_t(T_1)\sigma \sqrt{T-t}}  \\
	\frac{\partial^2 P_{fut}}{\partial F_t^2(T_1)} 
	  &= D_t \frac{\Phi'(-d_1)}{F_t(T_1)\sigma \sqrt{T-t}}
\end{align}

The thetas are:
\begin{align}
  \frac{\partial C_{fut}}{\partial t} 
	  &= - \frac{D_t F_t(T_1) \Phi'(d_1)\sigma}{2\sqrt{T-t}}
	     + r_t F_t(T_1) D_t \Phi(d_1) - r_t K D_t \Phi(d_2)   \\
  \frac{\partial P_{fut}}{\partial t} 
	  &= - \frac{D_t F_t(T_1) \Phi'(d_1)\sigma}{2\sqrt{T-t}}
	     - r_t F_t(T_1) D_t \Phi(-d_1) + r_t K D_t \Phi(-d_2)
\end{align}

The vegas are: 
\begin{align}
  \frac{\partial C_{fut}}{\partial \sigma} 
	  &= F_t(T_1) D_t \Phi'(d_1) \sqrt{T-t}    \\
  \frac{\partial P_{fut}}{\partial \sigma} 
	  &= F_t(T_1) D_t \Phi'(d_1) \sqrt{T-t}
\end{align}

The rhos are: 
\begin{align}
  \frac{\partial C_{fut}}{\partial r_t} &= K D_t \Phi(d_2) (T-t)  \\
	\frac{\partial P_{fut}}{\partial r_t} &= - K D_t \Phi(-d_2) (T-t)
\end{align}

%%%%%%%%%%%%%%%%%%%%%%%%%%%%%%%%%%%%%%%%%%%%%%%%%%
\section{Binary Options}
A binary (or digital) call option pays either an asset (called asset-or-nothing)
or \$1 (called cash-or-nothing) at the exercise time $T$ if $S_T>K$ and pays nothing otherwise.
A binary (or digital) put option pays either an asset or \$1 if $S_T<K$ and 
pays nothing otherwise. They can be priced using the standard Black-Scholes
model.

\begin{align}
	C_{AON} &= S_t D_t^d \Phi(d_1)   \\
	P_{AON} &= S_t D_t^d \Phi(-d_1)  \\
	C_{CON} &= D_t \Phi(d_2)         \\
	P_{CON} &= D_t \Phi(-d_2)        
\end{align}
where
\[
  d_{1,2}=\frac{\ln\left( \frac{S_t D_t^d}{K D_t} \right)}{\sigma\sqrt{T-t}}
          \pm \frac{1}{2} \sigma\sqrt{T-t}.
\]

The deltas are:
\begin{align}
	\frac{\partial C_{AON}}{\partial S_t}  
  	&= D_t^d (\Phi(d_1) + \frac{\Phi'(d_1)}{\sigma\sqrt{T-t}})  \\
	\frac{\partial P_{AON}}{\partial S_t}  
  	&= D_t^d (\Phi(-d_1) - \frac{\Phi'(d_1)}{\sigma\sqrt{T-t}})  \\
	\frac{\partial C_{CON}}{\partial S_t}  
  	&= \frac{D_t \Phi'(d_2)}{S_t \sigma\sqrt{T-t}}        \\
	\frac{\partial P_{CON}}{\partial S_t}  
  	&= - \frac{D_t \Phi'(d_2)}{S_t \sigma\sqrt{T-t}}        
\end{align}

The gammas are:
% \begin{align}
% 	\frac{\partial^2 C_{AON}}{\partial S_t^2}  
%   	&= \frac{D_t^d \Phi'(d_1)}{S_t\sigma\sqrt{T-t}}
%        	\left(1-\frac{d_1}{\sigma\sqrt{T-t}}  \right)  \\
% 	\frac{\partial^2 P_{AON}}{\partial S_t^2}  
%   	&= \frac{D_t^d \Phi'(d_1)}{S_t\sigma\sqrt{T-t}}
%        	\left(-1+\frac{d_1}{\sigma\sqrt{T-t}}  \right)  \\
% 	\frac{\partial^2 C_{CON}}{\partial S_t^2}  
%   	&= \frac{D_t \Phi'(d_2)}{S_t^2 \sigma\sqrt{T-t}}
%        	\left(-1-\frac{d_2}{\sigma\sqrt{T-t}}  \right)  \\
% 	\frac{\partial^2 P_{CON}}{\partial S_t^2}  
%   	&= \frac{D_t \Phi'(d_2)}{S_t^2 \sigma\sqrt{T-t}}
%        	\left(1+\frac{d_2}{\sigma\sqrt{T-t}}  \right)  
% \end{align}
\begin{align}
	\frac{\partial^2 C_{AON}}{\partial S_t^2}  
	  &= \frac{D_t^d \Phi'(d_1) (-d_2)}{S_t\sigma^2 (T-t)}  \\
	\frac{\partial^2 P_{AON}}{\partial S_t^2}  
    &= \frac{D_t^d \Phi'(d_1) (d_2)}{S_t\sigma^2 (T-t)}  \\
	\frac{\partial^2 C_{CON}}{\partial S_t^2}  
    &= \frac{D_t \Phi'(d_2) (-d_1)}{S_t^2 \sigma^2(T-t)}  \\
	\frac{\partial^2 P_{CON}}{\partial S_t^2}  
    &= \frac{D_t \Phi'(d_2) (d_1)}{S_t^2 \sigma^2(T-t)}  
\end{align}

The thetas are:
\begin{align}
	\frac{\partial C_{AON}}{\partial t} &= S_t D_t^d 
    \left[ y_t \Phi(d_1) + \Phi'(d_1) 
      \left( \frac{d_2}{2(T-t)} - \frac{r_t-y_t}{\sigma\sqrt{T-t}} \right) 
    \right]  \\
	\frac{\partial P_{AON}}{\partial t} &= S_t D_t^d 
    \left[ y_t \Phi(-d_1) - \Phi'(d_1) 
      \left( \frac{d_2}{2(T-t)} - \frac{r_t-y_t}{\sigma\sqrt{T-t}} \right) 
    \right]  \\
	\frac{\partial C_{CON}}{\partial t}  
  	&= r_t D_t \Phi(d_2) + D_t \Phi'(d_2) 
   	\left( \frac{d_1}{2(T-t)} - \frac{r_t-y_t}{\sigma\sqrt{T-t}} \right)  \\
	\frac{\partial P_{CON}}{\partial t}  
  	&= r_t D_t \Phi(-d_2) - D_t \Phi'(d_2) 
   	\left( \frac{d_1}{2(T-t)} - \frac{r_t-y_t}{\sigma\sqrt{T-t}} \right)  
\end{align}

The vegas are:
\begin{align}
	\frac{\partial C_{AON}}{\partial \sigma}  
  	&= - S_t D_t^d \Phi'(d_1) \frac{d_2}{\sigma}   \\
	\frac{\partial P_{AON}}{\partial \sigma}  
  	&= S_t D_t^d \Phi'(d_1) \frac{d_2}{\sigma}   \\
	\frac{\partial C_{CON}}{\partial \sigma}  
  	&= - D_t \Phi'(d_2) \frac{d_1}{\sigma}   \\
	\frac{\partial P_{CON}}{\partial \sigma}  
  	&= D_t \Phi'(d_2) \frac{d_1}{\sigma}   
\end{align}

The rhos are:
\begin{align}
	\frac{\partial C_{AON}}{\partial r_t}  
  	&= S_t D_t^d \Phi'(d_1) \frac{\sqrt{T-t}}{\sigma}  \\
	\frac{\partial P_{AON}}{\partial r_t}  
  	&= - S_t D_t^d \Phi'(d_1) \frac{\sqrt{T-t}}{\sigma}  \\
	\frac{\partial C_{CON}}{\partial r_t}  
  	&= D_t \Phi'(d_2) \frac{\sqrt{T-t}}{\sigma}  - (T-t) D_t \Phi(d_2)  \\
	\frac{\partial P_{CON}}{\partial r_t}  
  	&= - D_t \Phi'(d_2) \frac{\sqrt{T-t}}{\sigma}  - (T-t) D_t \Phi(-d_2)  
\end{align}

%%%%%%%%%%%%%%%%%%%%%%%%%%%%%%%%%%%%%%%%%%%%%%%%%%
\section{Binary Options on Futures}
Binary options (exercised on date $T$) on futures matured on date $T_1\ge T$ 
are priced using the Black76 model. The prices are:
\begin{align}
	C_{AON}^{fut} &= F_t(T_1) D_t \Phi(d_1)   \\
	P_{AON}^{fut} &= F_t(T_1) D_t \Phi(-d_1)  \\
	C_{CON}^{fut} &= D_t \Phi(d_2)         \\
	P_{CON}^{fut} &= D_t \Phi(-d_2)        
\end{align}
where
\[
	d_{1,2}= \frac{\ln(F_t(T_1)/K)}{\sigma\sqrt{T-t}} 
	         \pm \frac{1}{2}\sigma\sqrt{T-t}.
\]

The deltas are:
\begin{align}
	\frac{\partial C_{AON}^{fut}}{\partial F_t(T_1)}  
  	&= D_t \left( \Phi(d_1) + \frac{\Phi'(d_1)}{\sigma\sqrt{T-t}} \right) \\
	\frac{\partial P_{AON}^{fut}}{\partial F_t(T_1)}  
  	&= D_t \left( \Phi(-d_1) - \frac{\Phi'(d_1)}{\sigma\sqrt{T-t}} \right) \\
	\frac{\partial C_{CON}^{fut}}{\partial F_t(T_1)}  
  	&= \frac{D_t\Phi'(d_2)}{F_t(T_1)\sigma\sqrt{T-t}}  \\
	\frac{\partial P_{CON}^{fut}}{\partial F_t(T_1)}  
  	&= - \frac{D_t\Phi'(d_2)}{F_t(T_1)\sigma\sqrt{T-t}}  
\end{align}

The gammas are:
\begin{align}
	\frac{\partial^2 C_{AON}^{fut}}{\partial F_t^2(T_1)}  
  	&= \frac{D_t\Phi'(d_1)}{F_t(T_1)\sigma\sqrt{T-t}}  
  	   \left( 1 - \frac{d_1}{\sigma\sqrt{T-t}} \right)   \\
	\frac{\partial^2 P_{AON}^{fut}}{\partial F_t^2(T_1)}  
  	&= \frac{D_t\Phi'(d_1)}{F_t(T_1)\sigma\sqrt{T-t}}  
  	   \left( -1 + \frac{d_1}{\sigma\sqrt{T-t}} \right)   \\
	\frac{\partial^2 C_{CON}^{fut}}{\partial F_t^2(T_1)}  
  	&= \frac{D_t\Phi'(d_2)}{F_t^2(T_1)\sigma\sqrt{T-t}}  
  	   \left( -1 - \frac{d_2}{\sigma\sqrt{T-t}} \right)   \\
	\frac{\partial^2 P_{CON}^{fut}}{\partial F_t^2(T_1)}  
  	&= \frac{D_t\Phi'(d_2)}{F_t^2(T_1)\sigma\sqrt{T-t}}  
  	   \left( 1 + \frac{d_2}{\sigma\sqrt{T-t}} \right)   
\end{align}

The thetas are:
\begin{align}
	\frac{\partial C_{AON}^{fut}}{\partial t}  
	  &= F_t(T_1) D_t 
	     \left( r_t \Phi(d_1) + \frac{\Phi'(d_1) d_2}{2(T-t)} \right)   \\
	\frac{\partial P_{AON}^{fut}}{\partial t}  
	  &= F_t(T_1) D_t 
	     \left( r_t \Phi(-d_1) - \frac{\Phi'(d_1) d_2}{2(T-t)} \right)   \\
	\frac{\partial C_{CON}^{fut}}{\partial t}  
	  &= D_t 
	     \left( r_t \Phi(d_2) + \frac{\Phi'(d_2) d_1}{2(T-t)} \right)   \\
	\frac{\partial P_{CON}^{fut}}{\partial t}  
	  &= D_t 
	     \left( r_t \Phi(-d_2) - \frac{\Phi'(d_2) d_1}{2(T-t)} \right)   
\end{align}

The vegas are:
\begin{align}
	\frac{\partial C_{AON}^{fut}}{\partial \sigma}  
	  &= - F_t(T_1) D_t \Phi'(d_1) \frac{d_2}{\sigma}    \\
	\frac{\partial P_{AON}^{fut}}{\partial \sigma}  
	  &=  F_t(T_1) D_t \Phi'(d_1) \frac{d_2}{\sigma}    \\
	\frac{\partial C_{CON}^{fut}}{\partial \sigma}  
	  &=  - D_t \Phi'(d_2) \frac{d_1}{\sigma}    \\
	\frac{\partial P_{CON}^{fut}}{\partial \sigma}  
	  &=  D_t \Phi'(d_2) \frac{d_1}{\sigma} 
\end{align}

The rhos are:
\begin{align}
	\frac{\partial C_{AON}^{fut}}{\partial r_t}  
    &=  - (T-t) D_t F_t(T_1) \Phi(d_1)  \\
	\frac{\partial P_{AON}^{fut}}{\partial r_t}  
    &=  - (T-t) D_t F_t(T_1) \Phi(-d_1)  \\
	\frac{\partial C_{CON}^{fut}}{\partial r_t}  &=  - (T-t) D_t \Phi(d_2)  \\
	\frac{\partial P_{CON}^{fut}}{\partial r_t}  &=  - (T-t) D_t \Phi(-d_2)  
\end{align}

%%%%%%%%%%%%%%%%%%%%%%%%%%%%%%%%%%%%%%%%%%%%%%%%%%
\section{Binary Option with Delayed Payment}
We will consider (cash-or-nothing) binary options with exercise date $\tau$ 
earlier than the payment date $T>\tau$. This is useful for replicating range 
accruals. 

The price for binary options with delayed payment are 
\begin{align}
	C_{AON}(t,\tau,T) &= S_t D_{\tau T} D_{t\tau}^d \Phi(d_1)   \\
	P_{AON}(t,\tau,T) &= S_t D_{\tau T} D_{t\tau}^d \Phi(-d_1)  \\
	C_{CON}(t,\tau,T) &= D_{tT} \Phi(d_2)         \\
	P_{CON}(t,\tau,T) &= D_{tT} \Phi(-d_2)        
\end{align}
where $D_{\tau T}=\frac{D_{tT}}{D_{t\tau}}$ and
\[
	d_{1,2}=\frac{\ln\left( \frac{S_t D_{t\tau}^d}{K D_{t\tau}}\right)}{\sigma\sqrt{\tau-t}}
          \pm \frac{1}{2} \sigma\sqrt{\tau -t}.
\]

The deltas are:
\begin{align}
	\frac{\partial C_{AON}(t,\tau,T)}{\partial S_t}  
  	&= D_{\tau T} D_{t\tau}^d  
	     \left( \Phi(d_1) + \frac{\Phi'(d_1)}{\sigma\sqrt{\tau-t}} \right)  \\
	\frac{\partial P_{AON}(t,\tau,T)}{\partial S_t}  
  	&= D_{\tau T} D_{t\tau}^d 
	     \left( \Phi(-d_1) - \frac{\Phi'(d_1)}{\sigma\sqrt{\tau-t}} \right)  \\
	\frac{\partial C_{CON}(t,\tau,T)}{\partial S_t}  
  	&= \frac{D_{tT} \Phi'(d_2)}{S_t \sigma\sqrt{\tau-t}}        \\
	\frac{\partial P_{CON}(t,\tau,T)}{\partial S_t}  
  	&= - \frac{D_{tT} \Phi'(d_2)}{S_t \sigma\sqrt{\tau-t}}        
\end{align}

The gammas are:
\begin{align}
	\frac{\partial^2 C_{AON}(t,\tau,T)}{\partial S_t^2}  
  	&= \frac{D_{\tau T} D_{t\tau}^d \Phi'(d_1) (-d_2)}{S_t\sigma^2 (\tau-t)}  \\
	\frac{\partial^2 P_{AON}(t,\tau,T)}{\partial S_t^2}  
  	&= \frac{D_{\tau T} D_{t\tau}^d \Phi'(d_1) (d_2)}{S_t\sigma^2 (\tau-t)}  \\
	\frac{\partial^2 C_{CON}(t,\tau,T)}{\partial S_t^2}  
  	&= \frac{D_{tT} \Phi'(d_2) (-d_1)}{S_t^2 \sigma^2(\tau-t)}  \\
	\frac{\partial^2 P_{CON}(t,\tau,T)}{\partial S_t^2}  
  	&= \frac{D_{tT} \Phi'(d_2) (d_1)}{S_t^2 \sigma^2(\tau-t)}  
\end{align}

The thetas are:
\begin{align}
	\frac{\partial C_{AON}}{\partial t} &= S_t D_{\tau T} D_{t\tau}^d 
  	\left[ y_{t\tau} \Phi(d_1) + \Phi'(d_1) 
			\left( \frac{d_2}{2(\tau-t)} - \frac{r_{t\tau}-y_{t\tau}}{\sigma\sqrt{\tau-t}} \right) 
    \right]  \\
	\frac{\partial P_{AON}}{\partial t} &= S_t D_{\tau T} D_{t\tau}^d 
		\left[ y_{t\tau} \Phi(-d_1) - \Phi'(d_1) 
			\left( \frac{d_2}{2(\tau-t)} - \frac{r_{t\tau}-y_{t\tau}}{\sigma\sqrt{\tau-t}} \right) 
    \right]  \\
	\frac{\partial C_{CON}}{\partial t}  
  	&= r_{tT} D_{tT} \Phi(d_2) + D_{tT} \Phi'(d_2) 
  	\left( \frac{d_1}{2(\tau-t)} - \frac{r_{t\tau}-y_{t\tau}}{\sigma\sqrt{\tau-t}} \right)  \\
	\frac{\partial P_{CON}}{\partial t}  
  	&= r_{tT} D_{tT} \Phi(-d_2) - D_{tT} \Phi'(d_2) 
  	\left( \frac{d_1}{2(\tau-t)} - \frac{r_{t\tau}-y_{t\tau}}{\sigma\sqrt{\tau-t}} \right)  
\end{align}
where $r_{ts}=-\ln{D_{ts}}/(s-t)$ and $y_{ts}=-\ln{D^d_{ts}}/(s-t)$.

The vegas are:
\begin{align}
	\frac{\partial C_{AON}(t,\tau,T)}{\partial \sigma}  
	  &= - S_t D_{\tau T} D_{t\tau}^d \Phi'(d_1) \frac{d_2}{\sigma}   \\
	\frac{\partial P_{AON}(t,\tau,T)}{\partial \sigma}  
	  &= S_t D_{\tau T} D_{t\tau}^d \Phi'(d_1) \frac{d_2}{\sigma}   \\
	\frac{\partial C_{CON}(t,\tau,T)}{\partial \sigma}  
	  &= - D_{tT} \Phi'(d_2) \frac{d_1}{\sigma}   \\
	\frac{\partial P_{CON}(t,\tau,T)}{\partial \sigma}  
    &= D_{tT} \Phi'(d_2) \frac{d_1}{\sigma}   
\end{align}

The rhos are not well defined, should it be a derivative to $r_{tT}$ or
$r_{t\tau}$ or neither?
% The rhos are:
% \begin{align}
% 	\frac{\partial C_{AON}}{\partial r_t}  
%   	&= S_t D_t^d \Phi'(d_1) \frac{\sqrt{T-t}}{\sigma}  \\
% 	\frac{\partial P_{AON}}{\partial r_t}  
%   	&= - S_t D_t^d \Phi'(d_1) \frac{\sqrt{T-t}}{\sigma}  \\
% 	\frac{\partial C_{CON}}{\partial r_t}  
%   	&= D_t \Phi'(d_2) \frac{\sqrt{T-t}}{\sigma}  - (T-t) D_t \Phi(d_2)  \\
% 	\frac{\partial P_{CON}}{\partial r_t}  
%   	&= - D_t \Phi'(d_2) \frac{\sqrt{T-t}}{\sigma}  - (T-t) D_t \Phi(-d_2)  
% \end{align}







%%%%%%%%%%%%%%%%%%%%%%%%%%%%%%%%%%%%%%%%%%%%%%%%%%
\section{Black-Scholes PDE}
We can also derive the Black-Scholes formula using a PDE approach. Again
assuming the underlying stock price $S_t$ follows a geometric Brownian motion,
\[
	dS_t = S_t( \mu_t dt + \sigma_t dW_t),
\]
using Ito's lemma, the option price $V(t,S_t)$ follows SDE
\begin{align*}
	dV_t &= \frac{\partial V}{\partial t} dt 
	      + \frac{\partial V}{\partial S} dS_t
	      + \frac{1}{2} \frac{\partial^2 V}{\partial S^2} dS_t dS_t \\ 
			&= \left(  
	         \frac{\partial V}{\partial t} 
      	   + \frac{\partial V}{\partial S} \mu_t S_t
	         + \frac{1}{2} \frac{\partial^2 V}{\partial S^2} \sigma_t^2 S_t^2
			   \right) dt
				 + \left( \frac{\partial V}{\partial S} \sigma_t S_t \right) dW_t.
\end{align*}
On the other hand, assume that we can replicate option $V_t$ using $\alpha_t$
units of stock and $\beta_t$ units of bank account $B_t$, i.e.
\[
	V_t = \alpha_t S_t + \beta_t B_t,
\]
We require this replication to be self-financing, i.e. at any time we can adjust
the portfolio from $(\alpha_t,\beta_t)$ to 
$(\alpha_{t+\Delta t},\beta_{t+\Delta t})$ without any external cash, 
\[
  \alpha_t S_t + \beta_t B_t
  = \alpha_{t+\Delta t} S_t + \beta_{t+\Delta t} B_t
\]
hence
\[
  S_t d\alpha_t + B_t d\beta_t = 0.
\]
We then have
\[
	dV_t = \alpha_t dS_t + \alpha_t (q_t dt) S_t + \beta_t dB_t
	     = ( \alpha_t \mu_t S_t + \beta_t r_t B_t + \alpha_t q_t S_t) dt
			   + (\alpha_t \sigma_t S_t) dW_t,
\]
where $q_t$ is the dividend rate of the stock.
Match the $dW_t$ term we get
\[
	\alpha_t = \frac{\partial V}{\partial S},
\]
hence 
\[
	\beta_t B_t = V_t - \alpha_t S_t = V_t - \frac{\partial V}{\partial S} S_t.
\]
Now we match the $dt$ term and after substituting the expressions for $\alpha_t$
and $\beta_t$ we get the Black-Scholes PDE
\footnote{Note that the drift term $\mu_t$ does not appear in this PDE!}
\begin{equation}
	\frac{\partial V}{\partial t} 
	+ (r_t - q_t) S_t \frac{\partial V}{\partial S} 
	+ \frac{1}{2} \sigma_t^2 S_t^2 \frac{\partial^2 V}{\partial S^2} 
	- r_t V_t =0,
\end{equation}
with terminal condition (i.e. the payoff at time $T$)
\[
	V(T,S_T) = g(S_T).
\]

We can transform this PDE into a standard heat equation. First we remove the
$r_t V$ term by letting
\[
  V(t,S_t) = e^{-\int_t^T r_u du} F(t,S_t),
\]
and we get
\[
  \frac{\partial F}{\partial t} + (r_t-q_t) S_t \frac{\partial F}{\partial S_t}
    + \frac{1}{2}\sigma_t^2 S_t^2 \frac{\partial^2 F}{\partial S_t^2} = 0.
\]
Next we remove the dependence of $S_t$ in the coeffients by letting
\[
  X = \ln S_t,
\]
we get
\[
  \frac{\partial F}{\partial t} + (r_t-q_t-\frac{1}{2}\sigma_t^2) \frac{\partial F}{\partial X}
    + \frac{1}{2}\sigma_t^2 \frac{\partial^2 F}{\partial X^2} = 0.
\]
Next we remove the first derivative term by letting
\[
  Y = X + \int_t^T (r_s-q_s-\frac{1}{2}\sigma_s^2) ds, \qquad t' = t,
\]
and we have
\[
  \frac{\partial F}{\partial t'} + \frac{1}{2}\sigma_{t'}^2 \frac{\partial^2 F}{\partial Y^2} = 0.
\]
Finally, let
\[
  \tau=\int_{t'}^T \sigma_s^2 ds = \int_t^T \sigma_s^2 ds,
\]
we get
\[
  \frac{\partial F}{\partial \tau} = \frac{1}{2} \frac{\partial^2 F}{\partial Y^2}.
\]

To summarize, we define a set of new variables $(\tau, Y, F)$
\begin{align}
	Y &= \ln S_t + \int_t^T \left( r_s - q_s - \frac{1}{2} \sigma_s^2 \right) ds,
	\\
	%\tau &= \frac{1}{2} \int_t^T \sigma_s^2 ds, \\
	\tau &= \int_t^T \sigma_s^2 ds, \\
	V(t,S_t) &= e^{- \int_t^T r_s ds } F(\tau, Y),
\end{align}
then the Black-Scholes can be rewritten in a canonical form of PDE of the
parabolic kind
\begin{align}
	& \frac{\partial F}{\partial \tau} 
	   = \frac{1}{2} \frac{\partial^2 F}{\partial Y^2}, \\
	& F(0,Y) = g(e^Y).
\end{align}
The solution can be written as 
\footnote{Evans, Partial Differential Equations, 2ed, p.47; Strauss, Partial
	Differential Equations: An Introduction, pp.46-51. Note we can also rewrite it
  as $F(\tau,Y)=\int_{-\infty}^{\infty} p(t,y,y') g(e^{y'}) dy'$ where
	$p(t,y,y')=\frac{1}{\sqrt{2\pi\tau}} e^{-\frac{(y-y')^2}{2\tau}}$ is the
  transition density of a standard Brownian motion.}
\begin{equation}
	F(\tau,Y) = \int_{-\infty}^{\infty} \frac{1}{\sqrt{2\pi\tau}} 
	            e^{-\frac{(y-y')^2}{2\tau}} g(e^{y'}) dy'.
\end{equation}
% hence the price of an option is
% \[
% 	V(t,S_t) 
% 	  = e^{- \int_t^T r_s ds } F(\tau, Y)
% 	  = e^{- \int_t^T r_s ds } 
% 	    \int_{-\infty}^{\infty} 
% 			  \frac{1}{\sqrt{2\pi\tau}} e^{-\frac{(y-y')^2}{2\tau}} g(e^{y'}) dy'.
% \]
For an European vanilla call option, the payoff function is $g(x)=(x-K)^+$ 
where $K$ is the strike price, we then have
\begin{align*}
	F(\tau,Y) 
	  &= \int_{-\infty}^{\infty} 
  		 \frac{1}{\sqrt{2\pi\tau}} e^{-\frac{(y-y')^2}{2\tau}} (e^{y'}-K)^+ dy' \\
	  &= \int_{-\infty}^{\infty} 
			 \frac{1}{\sqrt{2\pi}} e^{-\frac{x^2}{2}} (e^y e^{\sqrt{\tau} x} -K)^+ dx
			 \qquad (x=\frac{y'-y}{\sqrt{\tau}}) \\
	 &= \int_{\frac{\ln K -y}{\sqrt{\tau}}}^{\infty} 
    	 \frac{1}{\sqrt{2\pi}} e^{-\frac{x^2}{2}} (e^y e^{\sqrt{\tau} x} -K) dx \\
	 &= e^y e^{\tau /2} \cdot \Phi \left(\frac{y-\ln K+\tau}{\sqrt{\tau}}\right)
		  - K \Phi \left( \frac{y-\ln K}{\sqrt{\tau}} \right).
\end{align*}
We define two discount factors
\[
	D_t = e^{-\int_t^T r_s ds},
\]
and
\[
	D_t^d = e^{-\int_t^T q_s ds},
\]
substitude the definition of $(\tau,Y,F)$, we get the price of an European
vanilla call option
\begin{align*}
	V(t,S_t) 
	  &= D_t F(\tau,Y)  \\
		&= S_t D_t^d 
	       \Phi\left( 
					\frac{\ln \frac{S_t D_t}{K D_t^d}+\frac{1}{2}\int_t^T \sigma_s^2 ds}
					     { \int_t^T \sigma_s^2 ds }
				\right)
	    -K D_t \Phi\left( 
					\frac{\ln \frac{S_t D_t}{K D_t^d}-\frac{1}{2}\int_t^T \sigma_s^2 ds}
					     { \int_t^T \sigma_s^2 ds }
				\right).
\end{align*}
This matches the Black-Scholes price (Eq. \ref{E:bs_div}) computed using
stochastic calculus.


%%%%%%%%%%%%%%%%%%%%%%%%%%%%%%%%%%%%%%%%%%%%%%%%%%
\section{Double Barrier Options}
In this section we study the analytical pricing of double barrier options.
Let the lower barrier be $A$, upper barrier be $B$, the option maturity be
$T$, and the spot price of the underlying be $S_0$. 
As in the case for the Black-Scholes model for vanilla European options
we assume the price of the underlying follows a geometric Brownian motion
\[
	dS_t = S_t(rdt + \sigma dW_t).
\]
Let $X_t=\ln S_t$, by Ito's Lemma we have
\[
	dX_t = (r-\frac{\sigma^2}{2})dt + \sigma dW_t,
\]
and the spot value of $X_t$ is $x_0=\ln S_0$. For convenience we also define
variables $a=\ln A$, $b=\ln B$, and
\[
	\mu = r - \frac{\sigma^2}{2}.
\]

We first consider a double knock-out option which will kick in iff neither
barrier is breached before maturity $T$.
The price of a double knock-out call option with strike $K$ is
\begin{align*}
	C_{dko} 
	&= e^{-rT} E_Q[ I_{T_{AB}>T} (S_T-K)^+ ]   \\
	&= e^{-rT} \int_A^B (S_T-K)^+ Q_{S_0}(T_{AB}>T, S_T\in dy)  \\
	&= e^{-rT} \int_a^b (e^{X_T}-K)^+ Q_{x_0}(T_{ab}>T, X_T\in dx)  \\
	&= \begin{cases}
	     0,     & K\ge B  \\
	     e^{-rT} \int_{a\vee\ln K}^b (e^{X_T}-K) Q_{x_0}(T_{ab}>T, X_T\in dx),
			       & K<B 
	   \end{cases}
\end{align*}
where $x=\ln y$.

It is convenient to define an auxilary function
\footnote{Luo, L.S.J.(2001), Various Types of Double-barrier Options, Journal of
Computational Finance, p.128}
\[
	 L(c;G,H)   
	= e^{-rT} 
     \int_G^H (e^x -K) e^{\frac{\mu}{\sigma^2}(x-x_0) - \frac{\mu^2}{2 \sigma^2} T}
	            \phi(x;c,\sigma^2 T) dx,    
\]
where 
\[
	\phi(x;y,t) = \frac{e^{-\frac{(y-x)^2}{2t}}}{\sqrt{2\pi t}}.
\]
After some algebra we get
\begin{align*}
	 & L(c;G,H)   \\
	=& e^{\frac{\mu}{\sigma^2}(c-x_0) - rT}  \\
	 & \times 
	   \bigg\{
			\left[
				e^{c+(\mu+ \frac{\sigma^2}{2}) T}
					\Phi\left(\frac{H-c-(\mu+\sigma^2) T}{\sigma\sqrt{T}}  \right)
				- K \Phi\left(\frac{H-c-\mu T}{\sigma\sqrt{T}}  \right)
			\right]   \\
	&-	\left[
				e^{c+(\mu+ \frac{\sigma^2}{2}) T}
					\Phi\left(\frac{G-c-(\mu+\sigma^2) T}{\sigma\sqrt{T}}  \right)
				- K \Phi\left(\frac{G-c-\mu T}{\sigma\sqrt{T}}  \right)
			\right]
		\bigg\}
\end{align*}

Now using the hitting probability of drifted Brownian motion in Equation
\ref{E:prob_drift_notouch}, we get for $K<B$
\begin{align*}
	C_{dko}
	 &= e^{-rT} \int_{a\vee\ln K}^b (e^{X_T}-K) Q_{x_0}(T_{ab}>T, X_T\in dx) \\
	 &= e^{-rT} \int_{a\vee\ln K}^b dx (e^x-K) 
      e^{\frac{\mu}{\sigma^2}(x-x_0) - \frac{\mu^2}{2 \sigma^2} T}
			\sum_{n=-\infty}^{\infty} 
			\left[
				\phi(x_0; x+2n(b-a),\sigma^2 T) - \phi(x_0; 2b-x-2n(b-a), \sigma^2 T)
			\right]   \\
	 &= e^{-rT} \int_{a\vee\ln K}^b dx (e^x-K) 
      e^{\frac{\mu}{\sigma^2}(x-x_0) - \frac{\mu^2}{2 \sigma^2} T}
			\sum_{n=-\infty}^{\infty} 
			\left[
				\phi(x; x_0-2n(b-a),\sigma^2 T) - \phi(x; 2b-x_0-2n(b-a), \sigma^2 T)
			\right]   \\
	 &= \sum_{n=-\infty}^{\infty} 
			\left[
				L(x_0-2n(b-a); a\vee\ln K, b) - L(2b-x_0-2n(b-a); a\vee\ln K, b) 
			\right]   
\end{align*}
hence
\begin{equation}
  C_{dko}  = I_{K<B}
    \sum_{n=-\infty}^{\infty} 
    \left[
  	L(x_0-2n(b-a); a\vee\ln K, b) - L(2b-x_0-2n(b-a); a\vee\ln K, b) 
    \right]    
\end{equation}


Alternatively using the Fourier expansion of the crossing probability in 
Equation \ref{E:prob_drift_notouch_alt}, we get for $K<B$
\begin{align*}
	C_{dko}
	 &= e^{-rT} \int_{a\vee\ln K}^b (e^{X_T}-K) Q_{x_0}(T_{ab}>T, X_T\in dx) \\
	 &= e^{-rT} \int_{a\vee\ln K}^b dx (e^x-K) 
	    e^{\frac{\mu}{\sigma^2}(x-x_0) - \frac{\mu^2}{2 \sigma^2} T} \frac{2}{b-a}
			\sum_{n=1}^{\infty} 
			e^{- \frac{n^2 \pi^2 \sigma^2 T}{2(b-a)^2} }
			\sin \frac{n\pi (x_0-a)}{b-a} \sin \frac{n\pi (x-a)}{b-a} \\
\end{align*}

We now define an auxilary indefinite integral
\begin{equation} \label{F:intQ}
  Q(\lambda,x) = \int e^{\lambda x} Q_{x_0}(T_{ab}>T, X_T\in dx),
\end{equation}
and
\begin{align*}
	Q(\lambda,x) 
	  &= \int e^{\lambda x} 
	          e^{\frac{\mu}{\sigma^2}(x-x_0) - \frac{\mu^2}{2 \sigma^2} T} \frac{2}{b-a}
			      \sum_{n=1}^{\infty} 
			        e^{- \frac{n^2 \pi^2 \sigma^2 T}{2(b-a)^2} }
			        \sin \frac{n\pi (x_0-a)}{b-a} \sin \frac{n\pi (x-a)}{b-a} 
			  dx \\
		&= \frac{2}{b-a} e^{-\frac{\mu x_0}{\sigma^2} - \frac{\mu^2}{2 \sigma^2} T} 
			 \sum_{n=1}^{\infty} e^{- \frac{n^2 \pi^2 \sigma^2 T}{2(b-a)^2} }
			   \sin \frac{n\pi (x_0-a)}{b-a} 
				 \int e^{\left( \lambda+\frac{\mu}{\sigma^2} \right) x}
				      \sin \frac{n\pi (x-a)}{b-a} dx  \\
\end{align*}
Using the result for the indefinite integral
\[
	\int e^{\alpha x} \sin\beta (x-a) dx 
	  = e^{\alpha x} 
		\frac{\alpha \sin\beta (x-a) - \beta \cos\beta (x-a)}{\alpha^2 + \beta^2},
\]
we have
\[
	  Q(\lambda,x) 
	= \frac{2}{b-a} 
	  e^{ - \frac{\mu x_0}{\sigma^2} - \frac{\mu^2}{2 \sigma^2} T
		    + (\lambda +\frac{\mu}{\sigma^2}) x
		  } 
	  \sum_{n=1}^{\infty} e^{- \frac{n^2 \pi^2 \sigma^2 T}{2(b-a)^2} }
			\sin \frac{n\pi (x_0-a)}{b-a}  \times
			\frac{
				     (\lambda + \frac{\mu}{\sigma^2}) \sin\frac{n\pi (x-a)}{b-a}
			        - \frac{n\pi}{b-a} \cos\frac{n\pi (x-a)}{b-a} 
					 }
			     { (\lambda+\frac{\mu}{\sigma^2})^2 + \frac{n^2 \pi^2}{(b-a)^2}
					 },
\]
and 
\begin{equation}
	C_{dko} = I_{K<B}\, e^{-rT} 
	  ( ( Q(1,b) - Q(1,a\vee \ln K) ) - K\times ( Q(0,b) - Q(0,a\vee \ln K) ) ).
\end{equation}

And similarly for a double knock-out put option with strike $K$:
\begin{align*}
  P_{dko} 
	&= e^{-rT} E_Q[I_{T_{AB}>T} (K-S_T)^+]   \\
	&= e^{-rT} \int_A^B (K-S_T)^+ Q_{S_0}(T_{AB}>T, S_T\in dy)  \\
	&= e^{-rT} \int_a^b (K- e^{X_T})^+ Q_{x_0}(T_{ab}>T, X_T\in dx)  \\
	&= I_{K>A}\, e^{-rT} \int_{a}^{b\wedge\ln K} (K- e^{X_T}) 
	   Q_{x_0}(T_{ab}>T, X_T\in dx) \\
\end{align*}
thus we have
\begin{equation}
  P_{dko}  = I_{K>A}
    \sum_{n=-\infty}^{\infty} 
    \left[
       L(2b-x_0-2n(b-a); a, b\wedge\ln K) - L(x_0-2n(b-a); a, b\wedge\ln K) 
    \right]    
\end{equation}
and alternatively
\begin{equation}
  P_{dko} = I_{K>A}\, e^{-rT} 
   (K\times (Q(0,b\wedge\ln K) - Q(0,a)) - (Q(1,b\wedge\ln K) - Q(1,a) )).
\end{equation}

To calcualte the price of a rebate, i.e. when it pays $R$ if either barrier is
touched before time $T$, 
\begin{align*}
  V_R &= e^{-rT} \int_{-\infty}^{\infty} R \times Q_{x_0}(T_{ab}<T,X_T\in dx) \\
      &= e^{-rT} R \times Q_{x_0}(T_{ab}<T) \\
      &= e^{-rT} R \times ( 1 - Q_{x_0}(T_{ab}>T)) 
\end{align*}

We define a new auxilary function
\begin{equation}
  M(c) = \int_a^b dx\, e^{\frac{\mu}{\sigma^2}(x-x_0) - \frac{\mu^2}{2 \sigma^2} T}
	 \phi(x; c,\sigma^2 T),
\end{equation}
and we have
\begin{align*}
  M(c) 
    &= \int_a^b dx\, e^{\frac{\mu}{\sigma^2}(x-x_0) - \frac{\mu^2 T}{2 \sigma^2}}
	\frac{e^{-\frac{(x-c)^2}{2\sigma^2 t}}}{\sqrt{2\pi\sigma^2 t}} \\
    &= \frac{1}{\sqrt{\sigma^2 T}} e^{\frac{\mu(c-x_0)}{\sigma^2}}
       \left[
         \Phi\left(\frac{b-c-\mu T}{\sqrt{\sigma^2 T}}\right)
         - \Phi\left(\frac{a-c-\mu T}{\sqrt{\sigma^2 T}}\right)
       \right]
\end{align*}

Hence the probability of not hitting any barrier before time $T$ is
\begin{align}
  Q_{x_0}(T_{ab}>T) 
    &= \int_a^b Q_{x_0}(T_{ab}>T, X_T\in dx) \qquad 
       \text{($T_{ab}>T$ implies $a<X_T<b$)} \notag \\
    &= \int_a^b dx\, e^{\frac{\mu}{\sigma^2}(x-x_0) - \frac{\mu^2}{2 \sigma^2} T}
       \sum_{n=-\infty}^{\infty} 
       \left[
	 \phi(x; x_0-2n(b-a),\sigma^2 T) - \phi(x; 2b-x_0-2n(b-a), \sigma^2 T)
       \right] \notag \\
    &= \sum_{n=-\infty}^{\infty} 
       \left[ M(x_0-2n (b-a)) - M(2b-x_0 -2n(b-a)) \right].
\end{align}

Alternatively, using the indefinite integral defined in \ref{F:intQ}, we have
\begin{equation}
  Q_{x_0}(T_{ab}>T) = Q(0,b) - Q(0,a).
\end{equation}

%%%%%%%%%%%%%%
\section{misc}

\[
	\frac{dS_t}{S_t} = \mu(S_t,t) dt + \sigma(S_t,t) dt
\]

\[
	F(0,T) = S_0 \, \exp \left( \int_0^T \mu(S_t,t) dt \right)
	       = S_0 \, \exp \left( \int_0^T (r_t - d_t - R_t) dt \right)
\]

Dupire equation:

\begin{align*}
	& \left[
	\frac{\partial}{\partial T} + \mu(K,t) K\frac{\partial}{\partial K}
	- \frac{1}{2} \sigma^2(K,T) K^2 \frac{\partial^2}{\partial K^2} 
	- (\mu(K,t) - r)
	\right] C(K,T) = 0  \\
	& C(K,T=0) = (S_0 - K)_+
\end{align*}

least squared error:

\[
	LSE = \sum_i \frac{1}{w_i} (C_i^{mkt} - C_i^{model})^2
\]


%%%%%%%%%%%%%%%%%%%%%%%
\section{Autocallable}

Autocallable is a popular type of structured product. The payoff of a typical 
contract depends on the performance of an underlying stock index (such as
CSI 500 in the Chinese market), and usually contains both knockout and knockin 
features.

To make it more specific, 
let $N$ be the notional, $c$ the knockout coupon, $T_0$ the issue date, $T$
the maturity date, $T_1,T_2,\dots,T_n$ the knockout observation dates (where
$T_n=T$), $T'_1,T'_2,\dots,T'_m$ the knockin observation dates, 
the payoff will be (see Figure \ref{F:autocall}):

\begin{enumerate}
\item If on any knockout observation date the price of the underlying is above a
preset knockout barrier, than the contract is called, and the payoff is
$N\times(1+c(T_i-T_0)/365)$, here $T_i$ is the earliest time the knockout event
happens;
\item If no knockout event happens during the lifetime of the contract, and on
some knockin observation date the price of the underlying is below the knockin
barrier, the payoff on the maturity date is $N\times(S_T/S_0)$, here $S_T$ and
$S_0$ is the price of the underlying on the maturity date and issue date,
respectively.
\item If neither knockout nor knockin happens, then the payoff on the maturity
date is $N\times(1+c(T_n-T_0)/365)$. 
\end{enumerate}

Autocallables can be priced using either PDE method or Monte Carlo method. Here
we study only the Monte Carlo method, especially the impact of GPU computing.


%%%%%%%%%%%%%%%%%%%%%%%%%%%%%%%%%%%%%%%%%%%%%%%%%%%%%%%%%%%%%%%%%%%%%%%%%%%%%%
%% adapted from https://texample.net/tikz/examples/simple-flow-chart/
\tikzstyle{block} = [rectangle, draw, fill=blue!20, node distance=3cm,
    rounded corners, minimum height=2em]
\tikzstyle{blank} = [rectangle, node distance=1.5cm]
\tikzstyle{line} = [draw, -latex']
\tikzstyle{cloud} = [rectangle, draw, fill=red!20, node distance=1.5cm,
    minimum height=2em]

%%%%%%%%%%%%%%%%%%%%%%%%%%%%%%%%%%%%%%%%%%%%%%%%%%%%%%%%%%%%%%%%%%%%%%%%%%%%%%
\begin{figure}
\begin{tikzpicture}[node distance = 2cm, auto]
    % Place nodes
    \node [blank] (T1) {$T_1$};
    \node [block, right of=T1] (C1) {Is index > KO barrier?};
    \node [cloud, right of=C1, node distance=6cm] (P1) {Called, repaid $N\times(1+c(T_1-T_0)/365)$};
    %\node [blank, below of=T1, node distance=1.5cm] (T2) {$T_2$};
    \node [blank, below of=T1] (T2) {$T_2$};
    \node [block, right of=T2] (C2) {Is index > KO barrier?};
    \node [cloud, right of=C2, node distance=6cm] (P2) {Called, repaid $N\times(1+c(T_2-T_0)/365)$};
    \node [blank, below of=T2] (T3) {$T_3$};
    \node [block, right of=T3] (C3) {Is index > KO barrier?};
    \node [cloud, right of=C3, node distance=6cm] (P3) {Called, repaid $N\times(1+c(T_3-T_0)/365)$};
    \node [blank, below of=T3, node distance=2cm] (Tn) {$T_n$};
    \node [block, right of=Tn] (Cn) {Is index > KO barrier?};
    \node [cloud, right of=Cn, node distance=6cm] (Pn) {Called, repaid $N\times(1+c(T_n-T_0)/365)$};
    %\node [block, below of=Cn] (Cn2) {Is index < KI barrier on any of $S_1,S_2,\dots,S_m$?};
    \node [block, below of=Cn, node distance=1.5cm] (Cn2) {Is index < KI barrier on any KI dates?};
    \node [cloud, right of=Cn2, node distance=6cm] (Pn2) {Repaid $N\times(S_n/S_0)$};
    \node [cloud, below of=Pn2] (Pn3) {Repaid $N\times(1+c(T_n-T_0)/365)$};

	% dots 
	\node at ($(T3)!0.5!(Tn)$) {\vdots};
	\node at ($(C3)!0.5!(Cn)$) {\vdots};

    % Draw edges
    \path [line] (C1) -- node {Yes}(P1);
    \path [line] (C1) -- node {No}(C2);
    \path [line] (C2) -- node {Yes}(P2);
    \path [line] (C2) -- node {No}(C3);
    \path [line] (C3) -- node {Yes}(P3);
    %\path [line, dotted] (T3) -- (Tn);
    %\path [line, dashed] (C3) -- (Cn);
    \path [line] (Cn) -- node {Yes}(Pn);
    \path [line] (Cn) -- node {No}(Cn2);
    \path [line] (Cn2) -- node {Yes}(Pn2);
    \path [line] (Cn2) |- node [near end] {No} (Pn3);
\end{tikzpicture}
  \caption[][9cm]{The payoff of a prototypical autocallable, with knockout
	  observation dates $T_1,T_2,\dots,T_n$, and knockin observation dates
		  $T'_1,T'_2,\dots,T'_m$, notional $N$, knockout coupon $c$.}
  \label{F:autocall}
\end{figure}
%%%%%%%%%%%%%%%%%%%%%%%%%%%%%%%%%%%%%%%%%%%%%%%%%%%%%%%%%%%%%%%%%%%%%%%%%%%%%%





