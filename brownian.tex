\chapter{Brownian Motion}

Unless specified otherwise, in this chapter $\tau \sim Exp(\lambda)$, that is,
$\tau$ follows an exponential distribution with rate $\lambda$,
\[
  P[\tau<t] = \int_0^t e^{-\lambda s} \lambda \, ds
\]

%%%%%%%%%%%%%%%%%%%%%%%%%%%%%%%%%%%%%%%%%%%%%%%%%%%
\section{Basic Facts}
For standard Brownian motion, we have the following basic facts:
\begin{itemize}
  \item Generator: $\mathcal{G} f = \frac{1}{2} \frac{d^2}{dx^2}$.
  \item Speed measure: $m(dx)=m(x)dx=2 dx$.
  \item Scale function: $s(x)=x$.
  \item Fundamental solutions of $\mathcal{G} f = \alpha f$ are
    $\psi_{\alpha}(x)=e^{\sqrt{2\alpha} x}$ and
    $\phi_{\alpha}(x)=e^{-\sqrt{2\alpha} x}$.
  \item Green function: $G_{\alpha}(x,y)=w_{\alpha}^{-1} e^{-\sqrt{2\alpha}(x-y)}$, 
	$x\ge y$
  \item Wronskian: $w_{\alpha}=2\sqrt{2\alpha}$.
\end{itemize}

%%%%%%%%%%%%%%%%%%%%%%%%%%%%%%%%%%%%%%%%%%%%%%%%%%%
\section{ $P_x[ \sup_{0\le s\le \tau} W_s \ge y ]$ (HOBM 1.1.1.2) }
Let $M_t = \sup\{W_s: s\le t \}$ be the maximum until time $t$ and 
$H_y=\inf\{s:W_s=y \}$ be the first hitting time of level $y\ge x$,
then we have for $x\le y$
\begin{align*}
	E_x[e^{-\lambda H_y}] 
	&= \int_0^{\infty} e^{-\lambda t} dP_x(H_y\le t)  \\
	&= e^{-\lambda t} P_x(H_y\le t) |_{t=0}^{\infty} 
	   - \int_0^{\infty} (-\lambda) e^{-\lambda t} P_x(H_y\le t) dt  \\
	&= \lambda \int_0^{\infty} e^{-\lambda t} P_x(H_y\le t) dt  \\
	&= P_x(H_y<\tau) = P_x(M_{\tau}\ge y).
\end{align*}
We know from Example \ref{Ex:hit2} that when $x\le y$
\[
  E_x[e^{-\lambda H_y}]
    =\frac{\psi_{\lambda}(x)}{\psi_{\lambda}(y)}
    = e^{\sqrt{2\lambda}(x-y)},
\]
hence
\footnote{Borodin and Salminen, Handbook of Brownian Motions, 2ed., 1.1.1.2,
p.153}
\begin{equation}
  P_x[ \sup_{0\le s\le \tau} W_s \ge y ] = e^{\sqrt{2\lambda}(x-y)}.
\end{equation}


%%%%%%%%%%%%%%%%%%%%%%%%%%%%%%%%%%%%%%%%%%%%%%%%%%%
\section{Single Boundary Hitting}

For a simple Brownian motion $X_t$ with zero drift and volatility $\sigma$,
\[
  dX_t = \sigma dW_t,
\]
let $\tau_a$ be the first hitting time at boundary $a$, i.e.
\[
  \tau_a = \inf\{t: X_t =a \},
\]
the transition probability density $p(x,t|x_0)$ of the process reaches 
$(x,x+dx)$ at time $t$ without hitting the boundary $a$, i.e.
\[
  p(x,t|x_0) dx=P(X_t\in dx, \tau_a>t | X_0=x_0),
\]
is the solution of the following equations:
\footnote{Pavliotis, Stochastic Processes and Applications, 4.2.1, pp.92-93}
\begin{align} \label{E:fp_nodrift}
  \frac{\partial}{\partial t} p(x,t|x_0)
    &= \frac{\sigma^2}{2} \frac{\partial^2 p(x,t|x_0)}{\partial x^2} \notag \\
  p(a,t|x_0) &= 0,  \qquad \text{(BC)}  \notag \\
  p(x,0|x_0) &= \delta (x-x_0),  \qquad \text{(IC)}  
\end{align}

The solution is
\footnote{Borodin and Salminen, Handbook of Brownian Motions, 2ed., p.121,
note that in Borodin the transition density is w.r.t speed measure.}
\begin{equation} \label{E:td_abs}
  p(x,t|x_0) = \frac{1}{\sqrt{2\pi \sigma^2 t}} 
    \left(  e^{-\frac{(x-x_0)^2}{2\sigma^2 t}} 
          - e^{-\frac{(x+x_0-2a)^2}{2\sigma^2 t}} \right)
\end{equation}
This can also be obtained using reflection principle. Since
\[
  P(X_t\le x, \tau_a<t | X_0=x_0) = P(X_t\ge 2a-x | X_0=x_0),
\]
hence
\[
  P(X_t\in dx, \tau_a<t | X_0=x_0) / dx
    = \frac{1}{\sqrt{2\pi \sigma^2 t}} e^{-\frac{(x+x_0-2a)^2}{2\sigma^2 t}}.
\]
Using
\[
P(X_t\in dx, \tau_a>t | X_0=x_0) 
   = P(X_t\in dx) - P(X_t\in dx, \tau_a<t | X_0=x_0),
\]
we get Equation \ref{E:td_abs}.

For a Wiener process with drift $\mu$ and volatility $\sigma$, 
\[
  dX_t = \mu dt + \sigma dW_t,
\]
the transition probability density $p_{\mu}(x,t|x_0)$ is the solution of the
forward Kolmogorov equation \ref{E:kolm_forw}:
\footnote{See also Pavliotis, Stochastic Processes and Applications, Eq. (2.54), p.48;
  Gikhman \& Skorokhod, Introduction to the Theory of Random Processes, pp. 374-375;
  Shreve, Stochastic Calculus for Finance II, Exercise 6.9, pp. 291-293}
\begin{align*}
  \frac{\partial}{\partial t} p_{\mu}(x,t|x_0)
    &= \frac{\sigma^2}{2} \frac{\partial^2 p_{\mu}(x,t|x_0)}{\partial x^2}   
     - \mu \frac{\partial p_{\mu}(x,t|x_0)}{\partial x}   \\
  p_{\mu}(a,t|x_0) &= 0,  \qquad \text{(BC)}  \\
  p_{\mu}(x,0|x_0) &= \delta (x-x_0),  \qquad \text{(IC)}  
\end{align*}

This problem can be reduced to the standard problem (\ref{E:fp_nodrift}) without
drift. First we try to remove the term with first derivative to $x$ by letting
$p_{\mu}(x,t|x_0)=e^{ax} u(x,t)$, we have
\[
 \frac{\sigma^2}{2} \frac{\partial^2 p_{\mu}}{\partial x^2}
 - \mu \frac{\partial p_{\mu}}{\partial x}
 = e^{ax} \left[
   \left( \frac{\sigma^2 a^2}{2} - \mu a \right) u
   + (a \sigma^2 -\mu) \frac{\partial u}{\partial x}
   + \frac{\sigma^2}{2} \frac{\partial^2 u}{\partial x^2}
  \right],
\]
choose $a=\mu/\sigma^2$, we then have
\[
  \frac{\partial u}{\partial t} 
    = - \frac{\mu^2}{2 \sigma^2} u + \mu \frac{\partial^2 u}{\partial x^2}   
\]
We can further simplify this by letting 
$u(x,t)=v(x,t) e^{-(\mu^2 t)/(2\sigma^2)}$, we then have
\[
  \frac{\partial v}{\partial t} = 
    \frac{\sigma^2}{2} \frac{\partial^2 v}{\partial x^2}.
\]
Now we have
$p_{\mu}(x,t|x_0)=e^{\frac{\mu}{\sigma^2}x-\frac{\mu^2 t}{2\sigma^2}} v(x,t)$,
we can see that boundary condition is satisfied with $v(a,t)=0$. However, the
initial condition of $v$ is 
$v(x,0|x_0)=e^{-\frac{\mu}{\sigma^2} x} \delta(x-x_0)$. To reduce this to the
standard $\delta$-function form, we let 
$v(x,t)=e^{-\frac{\mu}{\sigma^2} x_0} p(x,t|x_0)$, then
\[
  p(x,0|x_0)=e^{-\frac{\mu}{\sigma^2} (x-x_0)} \delta(x-x_0)=\delta(x-x_0).
\]


%By Girsanov Theorem, 
%\footnote{\textdbend need clarification. See also Albanese, Advanced Derivatives
%   Pricing and Risk Management Theory, Proposition 3.1, p.158}
%Let
%\begin{equation} 
%  p_{\mu}(x,t|x_0) = e^{\frac{\mu}{\sigma^2}(x-x_0)-\frac{\mu^2 t}{2\sigma^2}} 
%                     p(x,t|x_0),
%\end{equation}
%This problem can be reduced to the standard problem (\ref{E:fp_nodrift}) by
%letting
%\footnote{
%  Note that $e^{-\frac{\mu}{\sigma^2}(x-x_0)}\delta(x-x_0)=\delta(x-x_0)$.
%}
%i.e.

Hence
\footnote{See also Albanese, Advanced Derivatives
   Pricing and Risk Management Theory, Proposition 3.1, p.158}
\begin{equation} 
  p_{\mu}(x,t|x_0) = e^{\frac{\mu}{\sigma^2}(x-x_0)-\frac{\mu^2 t}{2\sigma^2}} 
                     p(x,t|x_0),
\end{equation}
where $p(x,t|x_0)$ is the solution of problem (\ref{E:fp_nodrift}).
Thus we get
\begin{align} 
  p_{\mu}(x,t|x_0) 
    &= \frac{1}{\sqrt{2\pi \sigma^2 t}} 
       e^{\frac{\mu}{\sigma^2}(x-x_0)-\frac{\mu^2 t}{2\sigma^2}} 
       \left(  e^{-\frac{(x-x_0)^2}{2\sigma^2 t}} 
          - e^{-\frac{(x+x_0-2a)^2}{2\sigma^2 t}} \right)
       \\ \notag
    &= \frac{1}{\sqrt{2\pi \sigma^2 t}} 
       \left(  e^{-\frac{(x-x_0-\mu t)^2}{2\sigma^2 t}} 
             - e^{-\frac{(x+x_0-2a-\mu t)^2}{2\sigma^2 t}} 
               e^{\frac{2\mu}{\sigma^2} (a-x_0)}
       \right).
\end{align} 

Let $x_L<x_0<x_H$ and 
Let $T_L=\min \{t:X_t=L \}$ and $T_H=\min \{t:X_t=H \}$ be the first hitting
time of boundaries $L$ and $H$ respectively, and let $T_{LH}=\min(T_L,T_H)$ be
the first hitting time of either boundary.
we define the following transition probability densities:
\begin{align}
  g_{\mu}(x,x_0;t) dx &= P_{x_0}( X_t \in dx)  \\
  g^u_{\mu}(x_H,x,x_0;t) dx &= P_{x_0}( T_H>t, X_t \in dx) \\
  g^l_{\mu}(x_L,x,x_0;t) dx &= P_{x_0}( T_L>t, X_t \in dx) \\
  g^{ul}_{\mu}(x_L,x_H,x,x_0;t) dx &= P_{x_0}( T_{LH}>t, X_t \in dx) 
\end{align}

Now
\begin{equation}
  g_0(x,x_0;t) 
  = \frac{e^{-\frac{(x-x_0)^2}{2\sigma^2 t}}}{\sqrt{2\pi \sigma^2 t}},
\end{equation}
and 
\begin{equation}
  g_{\mu}(x,x_0;t) 
  = e^{\frac{\mu}{\sigma^2}(x-x_0)-\frac{\mu^2 t}{2\sigma^2}} g_0(x,x_0,t)
  = \frac{e^{-\frac{(x-x_0-\mu t)^2}{2\sigma^2 t}}}{\sqrt{2\pi \sigma^2 t}}.
\end{equation}
The transition probability density for single barriers are of the same format:
for $x,x_0<x_H$,
\begin{equation}
  g^u_{\mu}(x_H,x,x_0;t)  
   = \frac{1}{\sqrt{2\pi \sigma^2 t}} 
       \left(  e^{-\frac{(x-x_0-\mu t)^2}{2\sigma^2 t}} 
             - e^{-\frac{(x+x_0-2 x_H-\mu t)^2}{2\sigma^2 t}} 
               e^{\frac{2\mu}{\sigma^2} (x_H-x_0)}
       \right),
\end{equation}
and for $x,x_0>x_L$,
\begin{equation}
  g^l_{\mu}(x_L,x,x_0;t) 
   = \frac{1}{\sqrt{2\pi \sigma^2 t}} 
       \left(  e^{-\frac{(x-x_0-\mu t)^2}{2\sigma^2 t}} 
             - e^{-\frac{(x+x_0-2 x_L-\mu t)^2}{2\sigma^2 t}} 
               e^{\frac{2\mu}{\sigma^2} (x_L-x_0)}
       \right).
\end{equation}
The transition probability density for double barriers has two equivalent
expressions:
\begin{align}
   & g^{ul}_{\mu}(x_L,x_H,x,x_0;t) \\ \notag
  =& e^{\frac{\mu}{\sigma^2}(x-x_0)-\frac{\mu^2 t}{2\sigma^2}}
     \sum_{n=-\infty}^{\infty} 
       [ g_{0}(x+2n(x_H-x_L),x_0;t) - g_{0}(2x_H-y-2n(x_H-x_L),x_0;t) ],
\end{align}
and
\begin{align}
  & g^{ul}_{\mu}(x_L,x_H,x,x_0;t) \\ \notag
 =& e^{\frac{\mu}{\sigma^2}(x-x_0)-\frac{\mu^2 t}{2\sigma^2}}
    \left[
      \frac{2}{x_H-x_L} 
      \sum_{n=1}^{\infty} e^{-\frac{n^2\pi^2 \sigma^2 t}{2(x_H-x_L)^2}}
 	 \sin \frac{n\pi (x_0-x_L)}{x_H-x_L} \sin \frac{n\pi (x-x_L)}{x_H-x_L}
    \right]
\end{align}






%%%%%%%%%%%%%%%%%%%%%%%%%%%%%%%%%%%%%%%%%%%%%%%%%%%%%%%%%%
\section{Double Boundary Hitting: Simple Brownian Motion}
Let $T_a=\min \{t:W_t=a \}$ and $T_b=\min \{t:W_t=b \}$ be the first hitting
time of boundaries $a$ and $b$ respectively, and let $T_{ab}=\min(T_a,T_b)$ be
the first hitting time of either boundary.
We also presume that $a<x<b$.

We treat the probability of hitting the lower boundary $a$ before hitting
boundary $b$ and also before time $T$ in two cases: $y<a$ and $y>a$.
In case of $y<a$, we have
\begin{align*}
	&P_x[T_a<T_b, T_a\le T, W_T\in dy]  \notag \\
	=&  P_x[T_a<T_b, W_T\in dy]   \qquad \text{($y<a$ implies $T_a\le T$)}
	    \notag \\
  =& P_x(W_T\in dy) - P_x(T_b<T_a, W_T\in dy)   \notag \\
	=& P_x(W_T\in dy) - P_x(T_b<T_a, W_T\in d(r_b y)) 
			\qquad \text{(reflection principle)}  \notag \\
	=& P_x(W_T\in dy) - P_x(W_T\in d(r_b y)) + P_x(T_a<T_b, W_T\in d(r_b y))
			\notag \\
	=& P_x(W_T\in dy) - P_x(W_T\in d(r_b y)) + P_x(W_T\in d(r_a r_b y))  
			- P_x(T_b<T_a, W_T\in d(r_a r_b y)) \notag \\
  =& etc.etc.   \notag \\
\end{align*}
where $r_a x = 2a-x$ is the reflection of $x$ with respect to $a$.
Let $f(y)$ be the density function such that
\begin{equation}
	f(y) dy = P_x(W_T\in dy), 
\end{equation}
we get
\begin{align*}
	 & P_x[T_a<T_b, T_a\le T, W_T\in dy]   \\
  =& dy\times (f(y) - f(r_b y) + f(r_a r_b y) - f(r_b r_a r_b y)
		 + \cdots) \\
	=& dy\times\sum_{n=0}^{\infty} [ f(y-2n(b-a)) - f(2b-y+2n(b-a)) ].
\end{align*}

Similarly for the case with $y>a$
\begin{align*}
	&P_x[T_a<T_b, T_a\le T, W_T\in dy]  \notag \\
	=& P_x[T_a<T_b, T_a\le T, W_T\in d(r_a y)]  \notag \\
	=& P_x[T_a<T_b, W_T\in d(r_a y)]  \qquad \text{($r_a y<a$ implies $T_a\le T$)} 
	   \notag \\
  =& P_x(W_T\in d(r_a y)) - P_x(T_b<T_a, W_T\in d(r_a y)) \notag \\
  =& P_x(W_T\in d(r_a y)) - P_x(T_b<T_a, W_T\in d(r_b r_a y)) \notag \\
  =& P_x(W_T\in d(r_a y)) - P_x(W_T\in d(r_b r_a y)) 
	   + P_x(T_a<T_b, W_T\in d(r_b r_a y)) \notag \\
	=& etc.etc. \notag \\
  =& dy\times (f(r_a y) - f(r_b r_a y) + f(r_a r_b r_a y) - f(r_b r_a r_b r_a y)
		 + \cdots) \\
  =& dy\times\sum_{n=0}^{\infty} [ f(2a-y-2n(b-a)) - f(y+2b-2a + 2n(b-a)) ].
\end{align*}

Thus we have
\begin{align}
	 & P_x[T_a<T_b, T_a\le T, W_T\in dy]  \notag \\
	=& \begin{cases}
     	 dy\times\sum_{n=0}^{\infty} [ f(y-2n(b-a)) - f(2b-y+2n(b-a)) ],         & y<a \\
	     dy\times\sum_{n=0}^{\infty} [ f(2a-y-2n(b-a)) - f(y+2b-2a + 2n(b-a)) ], & y>a
     \end{cases}
\end{align}

Similarly we treat the probability of hitting the upper boundary $b$ before 
hitting boundary $a$ and also before time $T$ in two cases: $y<b$ and $y>b$.
In case of $y<b$, we have
\begin{align*}
	&P_x[T_b<T_a, T_b\le T, W_T\in dy]  \notag \\
	=& P_x[T_b<T_a, T_b\le T, W_T\in d(r_b y)]  \notag \\
	=& P_x[T_b<T_a, W_T\in d(r_b y)]  \qquad \text{($r_b y>b$ implies $T_b\le T$)} 
	   \notag \\
  =& P_x(W_T\in d(r_b y)) - P_x(T_a<T_b, W_T\in d(r_b y)) \notag \\
	=& P_x(W_T\in d(r_b y)) - P_x(T_a<T_b, W_T\in d(r_a r_b y))  \notag \\
  =& P_x(W_T\in d(r_b y)) - P_x(W_T\in d(r_a r_b y))
     + P_x(T_b<T_a, W_T\in d(r_a r_b y)  \notag \\
  =& etc.etc. \notag \\
  =& dy\times (f(r_b y) - f(r_a r_b y) + f(r_b r_a r_b y) - f(r_a r_b r_a r_b y)
		 + \cdots) \\
  =& dy\times\sum_{n=0}^{\infty} [ f(2b-y+2n(b-a)) - f(y-2b+2a - 2n(b-a)) ].
\end{align*}
Similarly for the case with $y>b$ we have
\begin{align*}
	&P_x[T_b<T_a, T_b\le T, W_T\in dy]  \notag \\
	=&  P_x[T_b<T_a, W_T\in dy]   \qquad \text{($y>b$ implies $T_b\le T$)}
	    \notag \\
  =& P_x(W_T\in dy) - P_x(T_a<T_b, W_T\in dy)   \notag \\
	=& P_x(W_T\in dy) - P_x(T_a<T_b, W_T\in d(r_a y)) 
			\qquad \text{(reflection principle)}  \notag \\
	=& P_x(W_T\in dy) - P_x(W_T\in d(r_a y)) 
     + P_x(T_b<T_a, W_T\in d(r_a y))  \notag \\
	=& etc.etc.   \notag \\
  =& dy\times (f(y) - f(r_a y) + f(r_b r_a y) - f(r_a r_b r_a y)
		 + \cdots) \\
	=& dy\times\sum_{n=0}^{\infty} [ f(y+2n(b-a)) - f(2a-y-2n(b-a)) ].
\end{align*}
Hence
\begin{align}
	& P_x[T_b<T_a, T_b\le T, W_T\in dy]  \notag \\
	=& 
    \begin{cases}
			dy\times\sum_{n=0}^{\infty} [ f(2b-y+2n(b-a)) - f(y-2b+2a - 2n(b-a)) ], & y<b \\
			dy\times\sum_{n=0}^{\infty} [ f(y+2n(b-a)) - f(2a-y-2n(b-a)) ],  & y>b
    \end{cases}
\end{align}

We next calculate the probability of not hitting either boundary before time
$T$. First note that $T_{ab}>T$ implies $a<y<b$,
\begin{align*}
	& P_x(T_{ab}>T, W_T\in dy)  \\
	=& P_x(W_T\in dy) - P_x(T_{ab}<T,W_T\in dy)   \\
	=& P_x(W_T\in dy) - P_x(T_a<T_b, T_a<T, W_T\in dy)  
	   - P_x(T_b<T_a, T_b<T, W_T\in dy)    \\
  =& f(y) - ( f(r_a y) - f(r_b r_a y) + f(r_a r_b r_a y) - \cdots) )  
          - ( f(r_b y) - f(r_a r_b y) + f(r_b r_a r_b y) - \cdots) )  
\end{align*}
i.e.
\begin{equation}
	P_x(T_{ab}>T, W_T\in dy)  
	= dy\times\sum_{n=-\infty}^{\infty} [ f(y+2n(b-a)) - f(2b-y-2n(b-a)) ].
\end{equation}

The results in this section can be readily extended to Brownian motion with zero
drift and constant volatility $\sigma$, i.e. $\sigma W_t$, we just need to
set density function to
\[
	f(y) = P_x(\sigma W_T\in dy) 
	     = \frac{e^{-\frac{(y-x)^2}{2\sigma^2 T}}}{\sqrt{2\pi \sigma^2 T}}.
\]

Alternatively, these density functions can also be calculated using PDE method.
$P_x(T_{ab}>T, W_T\in dy)/dy$ is the unique solution $u(y,T)$ of the 
following initial-boundary value problem (IBVP) of the heat equation:
\begin{gather*}
	u_T = \frac{\sigma^2}{2} u_{yy},   \\
	u(a,T)=0,              \qquad \text{(boundary condition)}     \\
	u(b,T)=0,              \qquad \text{(boundary condition)}     \\
	u(y,0) = \delta(y-x) \qquad \text{(initial condition)}
\end{gather*}
This can be expressed as a Fourier series:
\begin{equation}
	u(y,T) = 
	  \frac{2}{b-a} 
		\sum_{n=1}^{\infty} e^{-\frac{n^2\pi^2 \sigma^2 T}{2(b-a)^2}}
			\sin \frac{n\pi (x-a)}{b-a} \sin \frac{n\pi (y-a)}{b-a}.
\end{equation}

Unfortunately, the densities $P_x(T_a<T_b, T_a\le T, W_T\in dy)/dy$ and
$P_x(T_b<T_a, T_b\le T, W_T\in dy)/dy$ do not seem to be solutions of any
Cauchy-Dirichlet problems of the heat equation.

%Similarly, the density $P_x(T_a<T_b, T_a\le T, W_T\in dy)/dy$ is the unique
%solution of the initial-boundary value problem:
%\begin{gather*}
%	u_t = \frac{\sigma^2}{2} u_{xx},   \\
%	u(a,t)=1,              \qquad \text{(boundary condition)}     \\
%	u(b,t)=0,              \qquad \text{(boundary condition)}     \\
%	u(x,0) = \delta(x-x_0) \qquad \text{(initial condition)}
%\end{gather*}
%The solution is
%\begin{equation}
%	u(x,t) = \frac{b-x}{b-a}
%	+ \sum_{n=1}^{\infty} 
%	  \left( \frac{2}{b-a} \sin \frac{n\pi (x_0-a)}{b-a} -\frac{2}{n\pi} \right)
%		e^{-\frac{n^2\pi^2 \sigma^2 t}{2(b-a)^2}} \sin \frac{n\pi (x-a)}{b-a}.
%\end{equation}
%
%And the density $P_x(T_b<T_a, T_b\le T, W_T\in dy)/dy$ is the unique
%solution of the initial-boundary value problem:
%\begin{gather*}
%	u_t = \frac{\sigma^2}{2} u_{xx},   \\
%	u(a,t)=0,              \qquad \text{(boundary condition)}     \\
%	u(b,t)=1,              \qquad \text{(boundary condition)}     \\
%	u(x,0) = \delta(x-x_0) \qquad \text{(initial condition)}
%\end{gather*}
%The solution is
%\begin{equation}
%	u(x,t) = \frac{x-a}{b-a}
%	+ \sum_{n=1}^{\infty} 
%	\left( \frac{2}{b-a} \sin \frac{n\pi (x_0-a)}{b-a} +\frac{2(-)^n}{n\pi} \right)
%		e^{-\frac{n^2\pi^2 \sigma^2 t}{2(b-a)^2}} \sin \frac{n\pi (x-a)}{b-a}.
%\end{equation}






%%%%%%%%%%%%%%%%%%%%%%%%%%%%%%%%%%%%%%%%%%%%%%%%%%%%%%%%%%%%%%
\section{Double Boundary Hitting: Brownian Motion with Drift}
In this section we consider the double boundary hitting of Brownian motion 
with constant drift $\mu$ and constant volatility $\sigma$, i.e. 
\[
	dX_t = \mu dt + \sigma dW_t.
\]
The density function $p_x(y,T)$ for any event $A$ at time $T$ satisfies the
forward Kolmogorov equation
\[
	\partial_T p + \mu \partial_y p - \frac{\sigma^2}{2} \partial_y^2 p = 0.
\]

By Girsanov Theorem, 
\footnote{\textdbend need clarification. See also Albanese, Advanced Derivatives
   Pricing and Risk Management Theory, Proposition 3.1, p.158}
\begin{equation} \label{E:density_change}
	p_x(y,T) = e^{\frac{\mu}{\sigma^2}(y-x)-\frac{\mu^2 T}{2\sigma^2}} u_x(y,T),
\end{equation}
where $u_x(y,T)$ is the density function under the measure with zero drift, i.e.
$X_t=\sigma \tilde{W}_t$. It is easy to verify that $u_x(y,T)$ satisfies the
heat equation
\[
   \partial_T u - \frac{\sigma^2}{2} \partial_y^2 u = 0.
\]

Similar to the case of Brownian motion without drift, 
let $T_a=\min \{t:X_t=a \}$ and $T_b=\min \{t:X_t=b \}$ be the first hitting
time of boundaries $a$ and $b$ respectively, and let $T_{ab}=\min(T_a,T_b)$ be
the first hitting time of either boundary. Using Equation \ref{E:density_change}
we have
\begin{align}
	 & P_x[T_a<T_b, T_a\le T, X_T\in dy]  \notag \\
	=& dy\times e^{\frac{\mu}{\sigma^2}(y-x)-\frac{\mu^2 T}{2\sigma^2}} \times
	   \begin{cases}
			 \sum_{n=0}^{\infty} [ f(y-2n(b-a)) - f(2b-y+2n(b-a)) ],         & y<a \\
			 \sum_{n=0}^{\infty} [ f(2a-y-2n(b-a)) - f(y+2b-2a + 2n(b-a)) ], & y>a
     \end{cases}
\end{align}
and
\begin{align}
	& P_x[T_b<T_a, T_b\le T, X_T\in dy]  \notag \\
	=& dy\times e^{\frac{\mu}{\sigma^2}(y-x)-\frac{\mu^2 T}{2\sigma^2}} \times
    \begin{cases}
			\sum_{n=0}^{\infty} [ f(2b-y+2n(b-a)) - f(y-2b+2a - 2n(b-a)) ], & y<b \\
			\sum_{n=0}^{\infty} [ f(y+2n(b-a)) - f(2a-y-2n(b-a)) ],  & y>b
    \end{cases}
\end{align}
and
\begin{equation} \label{E:prob_drift_notouch}
	P_x(T_{ab}>T, X_T\in dy)  
	= dy\times e^{\frac{\mu}{\sigma^2}(y-x)-\frac{\mu^2 T}{2\sigma^2}} \times
	  \sum_{n=-\infty}^{\infty} [ f(y+2n(b-a)) - f(2b-y-2n(b-a)) ].
\end{equation}
where
\[
	f(y) = P_x(\sigma W_T\in dy) 
	     = \frac{e^{-\frac{(y-x)^2}{2\sigma^2 T}}}{\sqrt{2\pi \sigma^2 T}}.
\]

Alternatively, these densities can also be written as
%\begin{align}
%	& P_x[T_a<T_b, T_a\le T, X_T\in dy]  \notag \\
%	=& dy\times e^{\frac{\mu}{\sigma^2}(y-x)-\frac{\mu^2 T}{2\sigma^2}} \times
%	   \left[
%	     \frac{b-y}{b-a}
%	     + \sum_{n=1}^{\infty} 
%	       \left( \frac{2}{b-a} \sin \frac{n\pi (x-a)}{b-a} -\frac{2}{n\pi} \right)
%		     e^{-\frac{n^2\pi^2 \sigma^2 t}{2(b-a)^2}} \sin \frac{n\pi (y-a)}{b-a}
%		\right]
%\end{align}
%
%\begin{align}
%	& P_x[T_b<T_a, T_b\le T, X_T\in dy]  \notag \\
%	=& dy\times e^{\frac{\mu}{\sigma^2}(y-x)-\frac{\mu^2 T}{2\sigma^2}} \times
%	   \left[
%	     \frac{y-a}{b-a}
%	     + \sum_{n=1}^{\infty} 
%			 \left( \frac{2}{b-a} \sin \frac{n\pi (x-a)}{b-a} +\frac{2(-)^n}{n\pi} \right)
%		     e^{-\frac{n^2\pi^2 \sigma^2 t}{2(b-a)^2}} \sin \frac{n\pi (y-a)}{b-a}
%		\right]
%\end{align}
%and
\begin{align}  \label{E:prob_drift_notouch_alt}
	& P_x[T_{ab}>T, X_T\in dy]  \notag \\
	=& dy\times e^{\frac{\mu}{\sigma^2}(y-x)-\frac{\mu^2 T}{2\sigma^2}} \times
	   \left[
	     \frac{2}{b-a} 
		   \sum_{n=1}^{\infty} e^{-\frac{n^2\pi^2 \sigma^2 t}{2(b-a)^2}}
		 	 \sin \frac{n\pi (x-a)}{b-a} \sin \frac{n\pi (y-a)}{b-a}
		\right]
\end{align}

From Theorem \ref{T:kac_diff3}, by setting $k(x)=\alpha$, $g(x)=0$, and
$f(x)=I_{x=a}$, then 
\[
	u(x) = E_x[e^{-\alpha T_a}, T_a<T_b] 
	= E_x[e^{-\alpha T_a} I_{X_{T_{ab}}=a}]
\]
is the unique solution of the Dirichlet problem
\[
	 \mu u_{x} + \frac{\sigma^2}{2} u_{xx} = \alpha u,
\]
with boundary condition
\[
	u(x)= I_{x=a}, \qquad x=a,b.
\]
Let $u(x) = e^{-\frac{\mu x}{\sigma^2}} v(x)$, we have
\[
	v_{xx}=\frac{2\alpha + \mu^2 / \sigma^2}{\sigma^2} v,
\]
and boundary conditions
\[
	v(a)=e^{\frac{\mu a}{\sigma^2}}, \qquad v(b)=0.
\]
The solution can be written as 
\[
	v(x)=C_1 e^{\beta x} + C_2 e^{-\beta x},
\]
where 
\[
	\beta = \frac{\sqrt{2\alpha+\frac{\mu^2}{\sigma^2}}}{\sigma},
\]
and $C_1,C_2$ are constants to be determined by the boundary conditions.
Solving this, we get
\footnote{Borodin and Salminen, Handbook of Brownian Motions, 2ed., 3.0.5(a),
  p.309, where $\sigma=1$.}
\begin{equation}
	u(x) = E_x[e^{-\alpha T_a}, T_a<T_b] 
	     = e^{\frac{\mu (a-x)}{\sigma^2} } 
	       \frac{\sinh \beta (b-x)}{\sinh \beta(b-a)}.
\end{equation}
Taking the inverse Laplace transform, we get the following density:
\begin{align*}
	P_x[T_a\in dt; T_a<T_b]/dt 
	&= L_{\alpha}^{-1} \left[  E_x[e^{-\alpha T_a}, T_a<T_b] \right] \\
	&= L_{\alpha}^{-1} 
	   \left[ 
		  e^{\frac{\mu (a-x)}{\sigma^2} } \frac{\sinh \beta (b-x)}{\sinh \beta(b-a)}
     \right]. 
\end{align*}
Using the definition of theta function $ss_t$ as in Eq. \ref{E:theta_ss} and Eq.
\ref{E:ilt_lin} we get
\footnote{Borodin and Salminen, Handbook of Brownian Motions, 2ed., 3.0.6(a),
  p.309, where $\sigma=1$.}
\begin{equation}
	P_x[T_a\in dt; T_a<T_b]  
	= e^{\frac{\mu (a-x)}{\sigma^2} - \frac{\mu^2 t}{2\sigma^2}}  \,
	  ss_t \left( \frac{b-x}{\sigma}, \frac{b-a}{\sigma} \right) \, dt.
\end{equation}

Similarly, from Theorem \ref{T:kac_diff3}, by setting $k(x)=\alpha$, $g(x)=0$, 
and $f(x)=I_{x=b}$, then 
\[
	u(x) = E_x[e^{-\alpha T_b}, T_b<T_a] 
	= E_x[e^{-\alpha T_b} I_{X_{T_{ab}}=b}]
\]
is the unique solution of the Dirichlet problem
\[
	 \mu u_{x} + \frac{\sigma^2}{2} u_{xx} = \alpha u,
\]
with boundary condition
\[
	u(x)= I_{x=b}, \qquad x=a,b.
\]
Solving this, we get
\footnote{Borodin and Salminen, Handbook of Brownian Motions, 2ed., 3.0.5(b),
  p.309, where $\sigma=1$.}
\begin{equation}
	u(x) = E_x[e^{-\alpha T_b}, T_b<T_a] 
	     = e^{\frac{\mu (b-x)}{\sigma^2} } 
	       \frac{\sinh \beta (x-a)}{\sinh \beta(b-a)},
\end{equation}
where 
\[
	\beta = \frac{\sqrt{2\alpha+\frac{\mu^2}{\sigma^2}}}{\sigma}.
\]
Again by taking the inverse Laplace transform, we get the following density:
\begin{align*}
	P_x[T_b\in dt; T_b<T_a]/dt 
	&= L_{\alpha}^{-1} \left[  E_x[e^{-\alpha T_b}, T_b<T_a] \right] \\
	&= L_{\alpha}^{-1} 
	   \left[ 
		  e^{\frac{\mu (b-x)}{\sigma^2} } \frac{\sinh \beta (x-a)}{\sinh \beta(b-a)}
     \right]. 
\end{align*}
Using the definition of theta function $ss_t$ as in Eq. \ref{E:theta_ss} and Eq.
\ref{E:ilt_lin} we get
\footnote{Borodin and Salminen, Handbook of Brownian Motions, 2ed., 3.0.6(b),
  p.309, where $\sigma=1$.}
\begin{equation}
	P_x[T_b\in dt; T_b<T_a]  
	= e^{\frac{\mu (b-x)}{\sigma^2} - \frac{\mu^2 t}{2\sigma^2}}  \,
	  ss_t \left( \frac{x-a}{\sigma}, \frac{b-a}{\sigma} \right) \, dt.
\end{equation}










%%%%%%%%%%%%%%%%%%%%%%%%%%%%%%%%%%%%%%%%%%%%%%%%%%%%%%%%%%%%%%
\section{Double Boundary Hitting: Geometric Brownian Motion}
In this section we consider the double boundary hitting of a Geometric Brownian 
motion 
\[
	dS_t = S_t(\mu dt + \sigma dW_t).
\]
Let $X_t = \ln{S_t}$, using Ito's lemma we get
\[
	dX_t = (\mu-\frac{\sigma^2}{2})dt + \sigma dW_t,
\]
i.e. $X_t$ is a Brownian motion with a nonzero drift.

Again, let $T_a=\min \{t:S_t=a \}$ and $T_b=\min \{t:S_t=b \}$ be the first 
hitting time of boundaries $a$ and $b$ respectively, and let 
$T_{ab}=\min(T_a,T_b)$ be the first hitting time of either boundary. 
Also let
\[
	\nu = \mu - \frac{\sigma^2}{2}.
\]
Using the results of the previous section we get
\footnote{Note that $S_t\in dy$ is equivalent to $X_t\in d(\ln y)$.}
\begin{align}
	 & P_x[T_a<T_b, T_a\le T, S_T\in dy]  \notag \\
	=& d(\ln y)\times e^{\frac{\nu}{\sigma^2}\ln(y/x) - \frac{\nu^2 T}{2\sigma^2}} \times
	   \begin{cases}
			 \sum_{n=0}^{\infty} 
			   [ f(\ln{y}-2n \ln(b/a)) - f(2\ln{b}-\ln{y}+2n\ln(b/a)) ],   & y<a \\
			 \sum_{n=0}^{\infty} 
			   [ f(2\ln a- \ln y-2n\ln(b/a)) - f(\ln y+2\ln(b/a) + 2n\ln(b/a)) ], & y>a
     \end{cases}
\end{align}
and
\begin{align}
	& P_x[T_b<T_a, T_b\le T, S_T\in dy]  \notag \\
	=& d(\ln y)\times e^{\frac{\nu}{\sigma^2}\ln(y/x) - \frac{\nu^2 T}{2\sigma^2}} \times
    \begin{cases}
			\sum_{n=0}^{\infty} 
			  [ f(2\ln b-\ln y+2n\ln(b/a)) - f(\ln y-2\ln(b/a) - 2n\ln(b/a)) ], & y<b \\
			\sum_{n=0}^{\infty} 
  			[ f(\ln y+2n\ln(b/a)) - f(2\ln a-\ln y-2n\ln(b/a)) ],  & y>b
    \end{cases}
\end{align}
and
\begin{equation}
	P_x(T_{ab}>T, S_T\in dy)  
	= d(\ln y)\times e^{\frac{\nu}{\sigma^2}\ln(y/x) - \frac{\nu^2 T}{2\sigma^2}} \times
	  \sum_{n=-\infty}^{\infty} [ f(\ln y+2n\ln(b/a)) - f(2\ln b-\ln y-2n\ln(b/a)) ].
\end{equation}
where
\[
	f(y) = P_{\ln x}(\sigma W_T\in dy) 
	     = \frac{e^{-\frac{(y-\ln x)^2}{2\sigma^2 T}}}{\sqrt{2\pi \sigma^2 T}}.
\]














%%%%%%%%%%%%%%%%%%%%%%%%%%%%%%%%%%%%%%%%%%%%%%%%%%%
\section{ $E_x[ e^{-\gamma l(\tau,r)} ]$ (HOBM 1.1.3.1)}
Notice first that from definition we have
\[
  E_x \left[ e^{-\gamma l(\tau,r)} \right] 
    = \lambda \, E_x 
        \left[ 
          \int_0^{\infty} e^{-\lambda t} e^{-\gamma l(t,r)} \, dt 
        \right],
\]
and from Theorem \ref{T:kac2}, we know that 
$E_x[ \int_0^{\infty} e^{-\lambda t} e^{-\gamma l(t,r)} \, dt ]$
is the solution of problem
\begin{align*}
  \frac{1}{2} M''(x) - \lambda M(x) &= -1  \\
  M'(r+0) - M'(r-0) &= 2\gamma M(r),
\end{align*}
$M(-\infty)$ and $M(\infty)$ are bounded.
It is easy to see that
\[
  M(x)= 
    \begin{cases}
      A e^{-\sqrt{2\lambda} (x-r)} + \frac{1}{\lambda}, & x>r \\
      A e^{\sqrt{2\lambda} (x-r)} + \frac{1}{\lambda},  & x<r
    \end{cases}
\]
satisfies the first equation, where $A$ is a constant that can be determined by
the second equation
\[
  A = - \frac{\gamma}{\lambda (\sqrt{2\lambda}+\gamma) }.
\]
Hence
\begin{equation}
  E_x \left[ \int_0^{\infty} e^{-\lambda t} e^{-\gamma l(t,r)} \, dt \right]
    = \frac{1}{\lambda} - \frac{\gamma}{\lambda (\sqrt{2\lambda}+\gamma)} 
                          e^{-\sqrt{2\lambda}\,|x-r|},
\end{equation}
and
\begin{equation}
  E_x \left[ e^{-\gamma l(\tau,r)} \right] 
    = 1 - \frac{\gamma}{\sqrt{2\lambda}+\gamma} e^{-\sqrt{2\lambda}\,|x-r|}.
\end{equation}


%%%%%%%%%%%%%%%%%%%%%%%%%%%%%%%%%%%%%%%%%%%%%%%%%%%%%%%%%%
\section{ $P_x[l(\tau,r)=0]$ and $P_x[l(\tau,r)\in dy]$ (HOBM 1.1.3.2)}
Let $f(y)$ be the probability density function such that
\[
  P_x[l(\tau,r)\in dy] = f(y) dy,
\]
then we have (total probability must be 1)
\[
  P_x[l(\tau,r)=0] + \int_0^{\infty} f(y) \, dy = 1.
\]
Note that
\[
  E_x \left[ e^{-\gamma l(\tau,r)} \right] 
    = P_x[l(\tau,r)=0] + \int_0^{\infty} e^{-\gamma y} f(y) \, dy,
\]
hence 
\[
  \int_0^{\infty} e^{-\gamma y} f(y) \, dy = \int_0^{\infty} f(y)\, dy
    - \frac{\gamma}{\gamma+\sqrt{2\lambda}} e^{-\sqrt{2\lambda} |x-r|}.
\]
Let $C=\int_0^{\infty} f(y) \, dy$, we get
\begin{align*}
  f(y) &= L_{\gamma}^{-1} 
          \left[
            C - \frac{\gamma}{\gamma+\sqrt{2\lambda}} e^{-\sqrt{2\lambda}|x-r|}
          \right]   \\
       &= \gamma [C- e^{-\sqrt{2\lambda} |x-r|}]
          + \sqrt{2\lambda}\, e^{-\sqrt{2\lambda} (|x-r|+y)}
\end{align*}
Since $C$ must be bounded, it is easy to see that $C= e^{-\sqrt{2\lambda}|x-r|}$
and
\begin{align}
  P_x[l(\tau,r)=0] &= 1 - e^{-\sqrt{2\lambda}|x-r|},   \\
  P_x[l(\tau,r)\in dy] &= \sqrt{2\lambda}\, e^{-\sqrt{2\lambda} (|x-r|+y)} dy,
    \qquad y>0
\end{align}




%%%%%%%%%%%%%%%%%%%%%%%%%%%%%%%%%%%%%%%%%%%%%%%%%%%
\section{ $E_x[ e^{-\gamma l(t,r)} ]$ (HOBM 1.1.3.3)}
Notice first that
\begin{align*}
  E_x \left[ e^{-\gamma l(t,r)} \right] 
    &= L_{\lambda}^{-1} 
         \left(
           E_x 
             \left[ 
               \int_0^{\infty} e^{-\lambda t} e^{-\gamma l(t,r)} \, dt 
             \right]
         \right) \\
    &= L_{\lambda}^{-1} 
       \left[
         \frac{1}{\lambda} - \frac{\gamma}{\lambda (\sqrt{2\lambda}+\gamma)} 
                             e^{-\sqrt{2\lambda}\,|x-r|},
       \right]  \\
    &= 1 - \gamma \, L_{\lambda}^{-1} 
       \left[
         \frac{1}{\lambda (\sqrt{2\lambda}+\gamma)} e^{-\sqrt{2\lambda}\,|x-r|}
       \right].  
\end{align*}
Using Equation \ref{E:ilt1} we get
\begin{equation}
  E_x \left[ e^{-\gamma l(t,r)} \right] 
    = 1 - \erfc \left( \frac{|x-r|}{\sqrt{2t}} \right)
      + e^{|x-r|\gamma + \frac{1}{2}\gamma^2 t}
        \erfc 
          \left( 
            \frac{\gamma \sqrt{t}}{\sqrt{2}} + \frac{|x-r|}{\sqrt{2t}}    
          \right).
\end{equation}


%%%%%%%%%%%%%%%%%%%%%%%%%%%%%%%%%%%%%%%%%%%%%%%%%%%
\section{ $E_x[ e^{-\gamma l(\tau,r)}; W_{\tau}\in dz ]$ (HOBM 1.1.3.5)}
First note that 
\[
  E_x[ e^{-\gamma l(\tau,r)}; W_{\tau}\in dz ]
    = \lambda \frac{\partial}{\partial z} E_x 
       \left[
         \int_0^{\infty} e^{-\lambda t} 1_{(-\infty,z)}(W_t)
           e^{-\gamma l(t,r)} \, dt
       \right]
      \, dz.
\]
From Theorem \ref{T:kac2}, we have that 
\[
  M(x) = E_x 
    \left[
      \int_0^{\infty} e^{-\lambda t} 1_{(-\infty,z)}(W_t)
           e^{-\gamma l(t,r)} \, dt
    \right]
\]
is the solution of problem
\begin{align*}
  \frac{1}{2} M''(x) - \lambda M(x) &= - 1_{(-\infty,z)}(x) \\
  M'(r+0) - M'(r-0) &= 2\gamma M(r).
\end{align*}
The solution would depend on whether $z>r$ or $z<r$. In the first case, we have
\[
  M(x)= 
    \begin{cases}
      A e^{\sqrt{2\lambda} x} + \frac{1}{\lambda} &x<z  \\
      B e^{\sqrt{2\lambda} x} + C e^{-\sqrt{2\lambda} x}  &z<x<r   \\
      D e^{-\sqrt{2\lambda} x}  &x>r  \\
    \end{cases}
\]
Use boundary conditions
\begin{align*}
  M(z+0) &= M(z-0),  \\
  M'(z+0) &= M'(z-0), \\
  M(r+0) &= M(r-0), \\
  M'(r+0) - M'(r-0) &= 2\gamma M(r),
\end{align*}
we get
\[
  M(x)= 
    \begin{cases}
      -\frac{\gamma}{2\lambda(\gamma+\sqrt{2\lambda})}
        e^{\sqrt{2\lambda}(x+z-2r)} - \frac{e^{\sqrt{2\lambda}(x-z)}}{2\lambda}
        + \frac{1}{\lambda},
      &x<z \\
      -\frac{\gamma}{2\lambda(\gamma+\sqrt{2\lambda})}
        e^{\sqrt{2\lambda}(x+z-2r)} + \frac{e^{\sqrt{2\lambda}(z-x)}}{2\lambda},
      &z<x<r \\
      -\frac{1}{\sqrt{2\lambda} (\gamma+\sqrt{2\lambda})}
        e^{\sqrt{2\lambda} (z-x)},  &x>r
    \end{cases}
\]
Take derivate with respect to $z$ and all three cases can be expressed as
\[
  \frac{\partial M(x)}{\partial z} 
    = \frac{e^{-\sqrt{2\lambda} |x-z|} }{\sqrt{2\lambda}}
      - \frac{\gamma}{\sqrt{2\lambda} (\gamma+\sqrt{2\lambda})}
        e^{ -\sqrt{2\lambda} (|x-r|+|z-r|) }
\]
Similarly when $z>r$, the solution has form
\[
  M(x)= 
    \begin{cases}
      A e^{\sqrt{2\lambda} x} + \frac{1}{\lambda}, &x<r  \\
      B e^{\sqrt{2\lambda} x} + C e^{-\sqrt{2\lambda} x} + \frac{1}{\lambda},
        &r<x<z   \\
      D e^{-\sqrt{2\lambda} x},  &x>z  \\
    \end{cases}
\]
and satisfies boundary conditions
\begin{align*}
  M(r+0) &= M(r-0), \\
  M'(r+0) - M'(r-0) &= 2\gamma M(r), \\
  M(z+0) &= M(z-0),  \\
  M'(z+0) &= M'(z-0). 
\end{align*}
The solution is
\[
  M(x)= 
    \begin{cases}
      -\frac{\gamma}{\lambda(\gamma+\sqrt{2\lambda})} e^{\sqrt{2\lambda}(x-r)}
      - \frac{e^{\sqrt{2\lambda}(x-z)}}{\sqrt{2\lambda}(\gamma+\sqrt{2\lambda})}
        + \frac{1}{\lambda},
      &x<r \\
      -\frac{\gamma}{2\lambda(\gamma+\sqrt{2\lambda})}
        e^{\sqrt{2\lambda}(2r-x-z)} - \frac{e^{\sqrt{2\lambda}(x-z)}}{2\lambda}
      -\frac{\gamma}{\lambda(\gamma+\sqrt{2\lambda})} e^{\sqrt{2\lambda}(r-x)}
        + \frac{1}{\lambda},
      &r<x<z \\
      -\frac{\gamma}{\lambda(\gamma+\sqrt{2\lambda})} e^{\sqrt{2\lambda}(r-x)}
      +\frac{\gamma}{2\lambda(\gamma+\sqrt{2\lambda})}
        e^{\sqrt{2\lambda}(2r-x-z)} + \frac{e^{\sqrt{2\lambda}(z-x)}}{2\lambda},
      &x>z
    \end{cases}
\]
Take derivate with respect to $z$ and all three cases can be expressed as
\[
  \frac{\partial M(x)}{\partial z} 
    = \frac{e^{-\sqrt{2\lambda} |x-z|} }{\sqrt{2\lambda}}
      - \frac{\gamma}{\sqrt{2\lambda} (\gamma+\sqrt{2\lambda})}
        e^{ -\sqrt{2\lambda} (|x-r|+|z-r|) }
\]
this is the same formula as in the case of $z<r$.
Thus 
\begin{equation} \label{E:H1.1.3.5}
  E_x[ e^{-\gamma l(\tau,r)}; W_{\tau}\in dz ]
  = \frac{\sqrt{\lambda}}{\sqrt{2}} e^{-\sqrt{2\lambda} |x-z|}
    - \frac{\gamma \sqrt{\lambda/2} }{ \gamma+\sqrt{2\lambda} }
      e^{ -\sqrt{2\lambda} (|x-r|+|z-r|) }.
\end{equation}

