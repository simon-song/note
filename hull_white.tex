\chapter{Hull-White Short Rate Model}

\section{Introduction}

Short rate models are arbitrage-free models of the term structure for which the
short rate $\{r_t, t\geq 0\}$ is a (time-inhomogeneous) Markov process in
the risk-neutral measure. The Markovian property of the short rate is essential
for practical implementation of these models
as it ensures that the price of a zero coupon bond $D_{tT}$ is a function of the
triple $(r_t, t, T)$. Thus, the state of the market at $t$ is completely
determined by the pair $(r_t,t)$.

The Hull-White model is one of the most popular short-rate models. Dynamics 
of the short rate $r_t$ under the risk-neutral measure $Q$ are
described by the following stochastic differential equation: 
\begin{equation} \label{E:hw}
    dr_t = (\theta_t - a_t r_t) dt + \sigma_t dW_t,
\end{equation}
where $\theta_t/a_t$ is the long-term level to which the short rate is
moving, $a_t$ is the rate at which the short rate is pushed towards the long
term level, $W_t$ is a Brownian motion under the risk-neutral measure, and
$\sigma_t$ is the volatility. Note that $a, \theta$, and $\sigma$
are deterministic functions of time. The exact form of these functions is chosen
to fit the model to initial bond prices (to make it arbitrage-free) and option
prices.

%%%%%%%%%%%%%%%%%%%%%%%%%%%
\section{Bond Price}
One reason for the popularity of the Hull-White model is that it is analytically
tractable. The value $D_{tT}$ of a pure discount bond at $t$ which matures at
$T$ is defined as:
\begin{equation}
  D_{tT} = E \left[ \exp \left( -\int_t^T r_u du \right) | F_t \right],
\end{equation}
it can be expressed analytically as 
\begin{equation}
  D_{tT}=A_{tT} e^{-B_{tT} r_t},
\end{equation}

Now we are going to derive the analytic formula for the bond price, i.e. 
formulas for $A_{tT}$, $B_{tT}$ and $\theta_t$ (as a function of model 
parameters $a_t$, $\sigma_t$ and initial zero curve $(D_{0t})_t$).
\footnote{Adapted from Hunt and Kennedy, Financial Derivatives in Theory and
Practice, revised ed., Wiley(2004), chapter 17, Short Rate Models.}

%where
%\begin{equation}
%  B_{tT} = \phi_t (\psi_T - \psi_t),
%\end{equation}
%\begin{equation}
%  \log A_{tT} = \log \frac{D_{0T}}{D_{0t}} 
%                - B_{tT} \frac{\partial}{\partial t} \log D_{0t} 
%                - \frac{\xi_t}{2} \left( \frac{B_{tT}}{\phi_t} \right)^2,
%\end{equation}
%and
%\begin{align}
%  \phi_t &= \exp \left( \int_0^t a_u du  \right) \notag \\
%  \psi_t &= \int_0^t \frac{du}{\phi_u} \notag \\
%  \xi_t &= \int_0^t \phi_u^2 \sigma_u^2 du \notag \\
%\end{align}
%
%We also have
%\begin{equation}
%  \theta_t = -a_t \frac{\partial}{\partial t} \log D_{0t} 
%             - \frac{\partial^2}{\partial t^2} \log D_{0t}
%             + \frac{\xi_t}{\phi_t^2}.
%\end{equation}

The bank account $N_t=\exp(\int_0^t r_s ds)$ is the numeriare of the
risk-neutral measure $Q$. Thus $D_{tT}/N_t$ is a martingale under $Q$ and must
have zero drift. Using Ito's lemma, we have 
\[
  d(e^{-\int_0^t r_s ds} D_{tT} ) 
     = e^{-\int_0^t r_s ds} ( - r_t D_{tT} dt + dD_{tT} ),
\]
hence the drift of $dD_{tT}/D_{tT}$ must be $r_t dt$. Using Ito's lemma again, 
we have
\[
  \frac{dD_{tT}}{D_{tT}} = -\sigma_t B_{tT} dW_t
    + \left(
       \frac{A'_{tT}}{A_{tT}} - B'_{tT} r_t - B_{tT} \sigma_t + B_{tT} a_t r_t
       + \frac{1}{2} \sigma_t^2 B_{tT}
      \right) dt,
\]
equalling the drift term to $r_t dt$, we get
\[
  - B'_{tT} + B_{tT} a_t = 1,
\]
and 
\[
  \frac{A'_{tT}}{A_{tT}} - B_{tT} \sigma_t + \frac{1}{2} \sigma_t^2 B_{tT} = 0.
\]

The first equation can be solved by defining $B_{tT}=y_t e^{\int_0^t a_s ds}$
and using the fact that $B_{TT}=0$ (because $D_{TT}=A_{TT}e^{-B_{TT}r_T}=1$ for
arbitrary $r_T$), this yields
\begin{equation} \label{E:b}
  B_{tT} = \phi_t (\psi_T - \psi_t),
\end{equation}
where for convenience we define the following variables
\begin{align} \label{E:conv}
  \phi_t &= e^{\int_0^t a_u du} &
  \psi_t &= \int_0^t \frac{du}{\phi_u} \notag \\
  \xi_t &= \int_0^t \phi_u^2 \sigma_u^2 du  &
  \zeta_t &= \int_0^t \psi_u \phi_u^2 \sigma_u^2 du \notag \\
  \eta_t &= \int_0^t \psi_u^2 \phi_u^2 \sigma_u^2 du &
  \mu_t &= \int_0^t \phi_u \theta_u du \notag \\
  \lambda_t &= \int_0^t \psi_u \phi_u \theta_u du \notag \\
\end{align}

Substitute $B_{tT}$ back to the second equation and do the integration, we get 
\[
  \log{A_{tT}} - \log{A_{0T}} = \psi_T \mu_t - \lambda_t -
    \frac{1}{2}\psi_T^2\xi_t + \psi_T\zeta_t - \frac{1}{2} \eta_t.
\]
To solve this, apply $\frac{\partial}{\partial_T}$ on both sides
\[
  \frac{\partial}{\partial_T} \left( \log{A_{tT}} - \log{A_{0T}} \right)
    = \frac{1}{\phi_T} [\mu_t - \psi_T \xi_t + \zeta_t],
\]
take the limit $\lim_{t\to T}$ and using the fact $A_{TT}=1$ we get
\begin{equation} \label{E:mu}
  \mu_T = \psi_T \xi_T - \zeta_T 
          - \phi_T \frac{\partial}{\partial T} \log{A_{0T}}.
\end{equation}
for any $T$.
Sustitute this back to 
\[
  \log{A_{TT}} - \log{A_{0T}} 
    = - \log{A_{0T}}
    = \psi_T \mu_T - \lambda_T 
      - \frac{1}{2}\psi_T^2 \xi_T + \psi_T \zeta_T - \frac{1}{2} \eta_T,
\]
we get
\[
  \lambda_T = \log{A_{0T}} 
              - \psi_T \phi_T \frac{\partial}{\partial_T} \log{A_{0T}}
              + \frac{1}{2} \psi_T^2 \xi_T - \frac{1}{2} \eta_T,
\]

Hence
\begin{align*}
  \log{A_{tT}} 
    &= - \log{A_{0T}} + \psi_T \mu_t - \lambda_t -
       \frac{1}{2}\psi_T^2\xi_t + \psi_T\zeta_t - \frac{1}{2} \eta_t \notag \\
    &= \log{\frac{A_{0T}}{A_{0t}}} 
       - B_{tT}\frac{\partial}{\partial_t} \log{A_{0t}}
       - \frac{\xi_t}{2} \left( \frac{B_{tT}}{\phi_t} \right)^2. \notag \\
\end{align*}
From definition $D_{0T}=A_{0T} e^{-B_{0T} r_0}$ and eqs. \ref{E:b} and
\ref{E:conv} we get
\[
  \log{\frac{A_{0T}}{A_{0t}}} 
    = \log{\frac{D_{0T}}{D_{0t}}} + \frac{B_{tT}}{\phi_t} r_0,
\]
and
\[
  \frac{\partial}{\partial t} \log{A_{0t}}
    = \frac{\partial}{\partial t} \log{D_{0t}} + \frac{r_0}{\phi_t}.
\]
Thus we get
\begin{equation} \label{E:a}
  \log A_{tT} = \log \frac{D_{0T}}{D_{0t}} 
                - B_{tT} \frac{\partial}{\partial t} \log D_{0t} 
                - \frac{\xi_t}{2} \left( \frac{B_{tT}}{\phi_t} \right)^2.
\end{equation}

We can solve $\theta_t$ by using the defintion of $\mu_t$ and eq. \ref{E:mu},
\[
  \theta_t = \frac{1}{\phi_t} \frac{\partial}{\partial_t} \mu_t
           = \frac{\xi_t}{\phi_t^2} 
             - a_t \frac{\partial}{\partial t} \log{A_{0t}}
             - \frac{\partial^2}{\partial t^2} \log{A_{0t}},
\]
using the definition of $D_{0t}=A_{0t} e^{-B_{0t}r_0}$ and eqs. \ref{E:b} and
\ref{E:conv} we get
\begin{equation} \label{E:theta}
  \theta_t = \frac{\xi_t}{\phi_t^2} 
             - a_t \frac{\partial}{\partial t} \log{D_{0t}}
             - \frac{\partial^2}{\partial t^2} \log{D_{0t}}.
\end{equation}
We also have
\begin{equation}
  \mu_t = \int_0^t \theta_s \phi_s ds
    = - r_0 + \int_0^t \frac{\xi_s}{\phi_s} ds 
      - \phi_t \frac{\partial}{\partial t} \log{D_{0t}}.
\end{equation}

In the special case that both mean reversion $a_t=a$ and $\sigma_t=\sigma$ are
constants we have
\[
  \phi_t = e^{at},   \qquad
  \psi_t = \frac{1-e^{-at}}{a}, \qquad
  \xi_t = \frac{\sigma^2}{2a} (e^{2at} -1)   
\]
and thus
\begin{align}
  &B_{tT} = \frac{1-e^{-a(T-t)}}{a}  \\
  &\log A_{tT} = \log \frac{D_{0T}}{D_{0t}} 
                - B_{tT} \frac{\partial}{\partial t} \log D_{0t} 
                - \frac{\sigma^2}{4a} \left( 1-e^{-2at} \right) B_{tT}^2
                \\
  &\theta_t = \frac{\sigma^2}{2a} \left( 1-e^{-2at} \right) 
             - a_t \frac{\partial}{\partial t} \log{D_{0t}}
             - \frac{\partial^2}{\partial t^2} \log{D_{0t}}  \\
  &\mu_t = -r_0 - e^{at} \frac{\partial}{\partial t} \log{D_{0t}}
          + \frac{\sigma^2 e^{at}}{2a^2} (1-e^{-at})^2.
\end{align}






%%%%%%%%%%%%%%%%%%%%%%%%%%%
\section{Alternative Approach for Bond Price}
We can also derive the bond price by solving the stochastic differential
equation \ref{E:hw}. Let $r_t=\frac{y_t}{\phi_t}$, where $\phi_t$ is defined 
in \ref{E:conv}, it is easy to see that
\[
  dy_t = \phi_t \theta_t dt + \phi_t \sigma_t dW_t.
\]
Hence
\[
  y_t = y_0 + \mu_t + \int_0^t \phi_u \sigma_u dW_u,
\]
where $\mu_t$ is defined in \ref{E:conv}.
It is easy to see that the initial value $y_0=r_0$. Thus we have
\[
  r_t = \frac{r_0+\mu_t}{\phi_t} 
       + \frac{1}{\phi_t} \int_0^t \phi_u \sigma_u dW_u.
\]
Now $r_t$ is Gaussian, 
$r_t\sim N(\frac{r_0+\mu_t}{\phi_t}, \frac{\xi_t}{\phi_t^2})$, and covariance
\[
  Cov[r_t,r_s] = \frac{\xi_{t\wedge s}}{\phi_t \phi_s},
\]
where $\xi_t$ is again defined in \ref{E:conv} and $t\wedge s=\min(t,s)$.

We need to calculate the expectations conditioned at $F_t$. It is easy to see
that for any $s>t$
\begin{equation}
  r_s = \frac{r_t \phi_t + \mu_s-\mu_t}{\phi_s} 
       + \frac{1}{\phi_s} \int_t^s \phi_u \sigma_u dW_u.
\end{equation}
This yields
\[
  E[r_s|F_t] = \frac{r_t \phi_t + \mu_s-\mu_t}{\phi_s},
\]
and 
\[
  Cov[r_s,r_u|F_t] = \frac{\xi_{s\wedge u} -\xi_t}{\phi_s \phi_u}.
\]
Now $\int_t^T r_s ds$ is Gaussian too, and its mean is
\[
  E[\int_t^T r_s ds|F_t] 
    = \int_t^T \frac{r_t \phi_t + \mu_s-\mu_t}{\phi_s} ds
    = (r_t \phi_t-\mu_t)(\psi_T-\psi_t) + \int_t^T \frac{\mu_s}{\phi_s}ds,
\]
and its variance is defined as
\begin{align*}
  V_{tT} 
   &\equiv Var[\int_t^T r_s ds|F_t] \notag \\
   &= Cov[\int_t^T r_s ds, \int_t^T r_u du|F_t] \notag \\
   &= \int_t^T ds \int_t^T du Cov[r_s,r_u|F_t] \notag \\
   &= \int_t^T ds \int_t^T du  \frac{\xi_{s\wedge u} -\xi_t}{\phi_s \phi_u} 
      \notag \\
   &= -\int_t^T \frac{\zeta_s}{\phi_s} ds + \psi_T \beta_T 
     + \psi_t \beta_t - 2 \psi_T \beta_t - \xi_t(\psi_T-\psi_t)^2, \notag \\
\end{align*}
where 
\begin{equation} \label{E:conv2}
  \beta_t=\int_0^t \frac{\xi_s}{\phi_s} ds,
\end{equation}
and $\phi_t$, $\psi_t$, $\mu_t$, $\xi_t$, and $\eta_t$ are defined 
in eq. \ref{E:conv}. From this we also have
\[
  V_{0T} = -\int_0^T \frac{\zeta_s}{\phi_s} ds + \psi_T \beta_T,
\]
\[
  V_{0t} = -\int_0^t \frac{\zeta_s}{\phi_s} ds + \psi_t \beta_t,
\]
and
\begin{equation} \label{E:dev_V}
  \frac{\partial}{\partial t} \log V_{0t} = \frac{2\beta_t}{\phi_t}.
\end{equation}

Given that $\int_t^T r_s ds$ is Gaussian, bond price can be rewritten as
\[
  D_{tT} = E \left[ \exp \left( -\int_t^T r_u du \right) | F_t \right]
         = \exp \left( -E[\int_t^T r_u du|F_t] + \frac{1}{2} V_{tT} \right).
\]
this yields
\[
  \log D_{tT} = - \phi_t(\psi_T-\psi_t) r_t + \mu_t(\psi_T-\psi_t)
                - \int_t^T \frac{\mu_s}{\phi_s} ds + \frac{1}{2} V_{tT}.
\]
From this we get
\[
  \log D_{0T} = - r_0 \psi_T - \int_0^T \frac{\mu_s}{\phi_s} ds 
                + \frac{1}{2} V_{0T},
\]
\[
  \log D_{0t} = - r_0 \psi_t - \int_0^t \frac{\mu_s}{\phi_s} ds 
                + \frac{1}{2} V_{0t},
\]
and 
\begin{equation} \label{E:dev_bond}
  \frac{\partial}{\partial t} \log D_{0t} 
    = \frac{-r_0-\mu_t}{\phi_t} 
      + \frac{1}{2} \frac{\partial}{\partial t} V_{0t}.
\end{equation}
Using this we get
\[
  \log D_{tT} = 
    - B_{tT} r_t - B_{tT} \frac{\partial}{\partial t} \log D_{0t} 
    + \log \left( \frac{D_{0T}}{D_{0t}} \right)
    + \frac{1}{2} 
        \left(
          V_{tT} - V_{0T} + V_{0t} 
          + B_{tT} \frac{\partial}{\partial t} V_{0t}
        \right)
\]
where 
\[
  B_{tT} = \phi_t(\psi_T-\psi_t)
\]
is the same as in eq. \ref{E:b}. And with the definition of bond
$D_{tT}=A_{tT} e^{-B_{tT} r_t}$ we have 
\[
  \log A_{tT} = 
    - B_{tT} \frac{\partial}{\partial t} \log D_{0t} 
    + \log \left( \frac{D_{0T}}{D_{0t}} \right)
    + \frac{1}{2} 
        \left(
          V_{tT} - V_{0T} + V_{0t} 
          + B_{tT} \frac{\partial}{\partial t} V_{0t}
        \right)
\]

Using the properties of $V_{tT}$, we get
\[
  V_{tT} - V_{0T} + V_{0t} 
  + B_{tT} \frac{\partial}{\partial t} \log V_{0t} 
  = - \xi_t(\psi_T-\psi_t)^2,
\]
and this yields
\[
  \log A_{tT} = 
    - B_{tT} \frac{\partial}{\partial t} \log D_{0t} 
    + \log \left( \frac{D_{0T}}{D_{0t}} \right)
    - \frac{\xi_t}{2} \left( \frac{B_{tT}}{\phi_t} \right)^2
\]
which is the same as eq. \ref{E:a}.

Now we calculate $\theta_t$. From eq. \ref{E:dev_bond} we have
\[
  \int_0^t \theta_s \phi_s ds = \mu_t 
    = -r_0 - \phi_t \frac{\partial}{\partial t} \log{D_{0t}}
      + \frac{1}{2} \phi_t \frac{\partial}{\partial t} V_{0t},
\]
hence
\[
  \theta_t = \frac{1}{\phi_t} \frac{\partial}{\partial t} \mu_t
    = - a_t \frac{\partial}{\partial t} \log{D_{0t}}
      - \frac{\partial^2}{\partial t^2} \log{D_{0t}}
      + \frac{1}{2 \phi_t} \frac{\partial}{\partial t} 
        \left(
          \phi_t \frac{\partial}{\partial t} V_{0t}
        \right).
\]
Using eqs. \ref{E:dev_V} and \ref{E:conv2} we get
\[
   \frac{1}{\phi_t} \frac{\partial}{\partial t} 
     \left( \phi_t \frac{\partial}{\partial t} V_{0t} \right)
   = \frac{2\xi_t}{\phi_t^2},
\]
hence
\[
  \theta_t 
    = - a_t \frac{\partial}{\partial t} \log{D_{0t}}
      - \frac{\partial^2}{\partial t^2} \log{D_{0t}}
      + \frac{\xi_t}{\phi_t^2},
\]
same as eq. \ref{E:theta}.

%%%%%%%%%%%%%%%%%%%%%%%%%%%
\section{Bond Option}
Now we derive the analytic formula of the price of an European on zero coupon
bond. Let $T$ be the option maturity and $S$ be the bond maturity, and the
payoff a call option with strike $K$ at time $T$ is $V_T=(D_{TS}-K)^+$. Hence
the option price at time $t$ is
\[
  \frac{V_t}{N_t} = E[\frac{V_T}{N_T}|F_t],
\]
where $N_t$ is the numeriare of our choice. Now we choose $N_t=D_{tT}$ and we
have the nice expression that $N_T=D_{TT}=1$ and
\[
  V_t= D_{tT} E_F[(D_{TS}-K)^+|F_t].
\]
where $Q_F$ is the forward measure for numeriare $N_t=D_{tT}$.

Let $Y_t=\frac{D_{tS}}{D_{tT}}$, then $D_{TS}=Y_T$. And $Y_t$ is a 
$Q_F$-martingale, i.e. lognormal under $Q_F$-measure. Using the affine 
expression
\[
  D_{tT} = A_{tT} e^{-B_{tT} r_t},
\]
we get
\[
  Y_t = \frac{A_{tS}}{A_{tT}} e^{-(B_{tS}-B_{tT}) r_t},
\]
and 
\[
  \frac{dY_t}{Y_t} = - (B_{tS}-B_{tT}) \sigma_t dW_t^F
    = - \phi_t (\psi_S-\psi_T) \sigma_t dW_t^F,
\]
Solve this we get
\[
  D_{TS} = Y_T = \frac{D_{tS}}{D_{tT}}
    \exp 
      \left(    
        -(\psi_S-\psi_T) \int_t^T \phi_u \sigma_u dW_u^F
        -\frac{1}{2}(\psi_S-\psi_T)^2 \int_t^T \phi_u^2 \sigma_u^2 du
      \right).
\]
Now we can rewrite $D_{TS}$ as $D_{TS}=\frac{D_{tS}}{D_{tT}} e^{X_t}$ where 
$X_t\sim N(-\frac{\sigma_p^2}{2}, \sigma_p^2)$ is Gaussian and 
\begin{equation} \label{E:sigma_p}
  \sigma_p \equiv (\psi_S-\psi_T) \sqrt{\xi_T-\xi_t}.
\end{equation}
In case $a_t\equiv a$ and $\sigma_t=\sigma$ are constants we have
\[
  \sigma_p = \sigma \sqrt{ \frac{1-e^{-2a(T-t)}}{2a} }
            \left( \frac{ 1-e^{-a(S-T)} }{a}  \right).
\]

Hence
\begin{align*}
  V_t&= D_{tT} E_F[(D_{TS}-K)^+|F_t] \notag \\
     &= D_{tT} E_F[(\frac{D_{tS}}{D_{tT}}e^{X_t} -K)^+|F_t] \notag \\
     &= D_{tS} E_F[(e^{X_t} -\frac{D_{tT}K}{D_{tS}})^+|F_t] \notag \\
\end{align*}
Using eq. \ref{E:int_call}, we get the price of a call option that matures at 
time $T$ and with strike $K$ on a zero coupon bond maturing at time $S>T$  
\begin{equation} \label{E:zbc}
  ZBC(t,T,S,K) = 
    D_{tS}\Phi
      \left( \frac{\sigma_p}{2}-\frac{\log(KD_{tT}/D_{tS})}{\sigma_p} \right)
    - K D_{tT} \Phi
      \left(-\frac{\sigma_p}{2}-\frac{\log(KD_{tT}/D_{tS})}{\sigma_p} \right).
\end{equation}
Similarly we have the price for a put option
\begin{equation} \label{E:zbp}
  ZBP(t,T,S,K) = 
    K D_{tT} \Phi
      \left(\frac{\sigma_p}{2}+\frac{\log(KD_{tT}/D_{tS})}{\sigma_p} \right)
    - D_{tS}\Phi
      \left( -\frac{\sigma_p}{2}+\frac{\log(KD_{tT}/D_{tS})}{\sigma_p} \right).
\end{equation}
Note that we have the call-put parity
\[
  ZBC(t,T,S,K)-ZBP(t,T,S,K) = D_{tS}-KD_{tT}.
\]

%%%%%%%%%%%%%%%%%%%%%%%%%%%%%%%%%%%%%%%
\section{Caplet/floorlet}
A caplet has payoff at time $S$
\[
  V_S = \tau(T,S) (L_T[T,S]-K)^+,
\]
and its price is related to bond options. To see this, note that
\begin{align*}
  V_t &= E_Q[e^{-\int_t^S r_u du} V_S | F_t]  \notag \\
      &= E_Q[e^{-\int_t^T r_u du} D_{TS} V_S | F_t]  \notag \\
      &= E_Q[e^{-\int_t^T r_u du} D_{TS} \tau(T,S) (L_T[T,S]-K)^+ | F_t]  
         \notag \\
      &= (1+K\tau(T,S)) 
         E_Q[e^{-\int_t^T r_u du} (\frac{1}{1+K\tau(T,S)}-D_{TS})^+ | F_t]  
         \notag \\
\end{align*}
hence
\[
  Caplet(t,T,S,K) = (1+K\tau(T,S))\, ZBP(t,T,S,\frac{1}{1+K\tau(T,S)}),
\]
and
\[
  Floorlet(t,T,S,K) = (1+K\tau(T,S))\, ZBC(t,T,S,\frac{1}{1+K\tau(T,S)}).
\]

Using the price formulaes for bond options (i.e. eqs. \ref{E:zbc} and
\ref{E:zbp}), we get 
\begin{equation} \label{E:caplet}
  Caplet(t,T,S,K) = D_{tT}\Phi(-h+\sigma_p) - D_{tS}(1+K\tau(T,S)) \Phi(-h),
\end{equation}
and
\begin{equation} \label{E:floorlet}
  Floorlet(t,T,S,K) = -D_{tT}\Phi(h-\sigma_p) + D_{tS}(1+K\tau(T,S)) \Phi(h),
\end{equation}
where 
\[
  h = \frac{1}{\sigma_p} \log\left( \frac{D_{tS}(1+K\tau(T,S))}{D_{tT}} \right)
        + \frac{\sigma_p}{2},
\]
and $\sigma_p$ is defined in eq. \ref{E:sigma_p}.

Note that these satisfy the put-call parity
\[
  Caplet(t,T,S,K)-Floorlet(t,T,S,K) = D_{tT}-D_{tS}(1+K\tau(T,S)) 
    = FRA(t,T,S,K).
\]

The formulaes for caplets and floorlets (eqs. \ref{E:caplet} and
\ref{E:floorlet}) can also be obtained by using the same method as used by
calculating bond option. Let $Y_t=\frac{D_{tT}}{D_{tS}}=1+L_t(T,S)\tau(T,S)$ 
and let numeraire be $N_t=D_{tS}$ and the corresponding measure be $Q^N$,
we have the caplet price 
\[
  V_t = D_{tS} E_N[(Y_T-1-K\tau(T,S))^+|F_t].
\]
Note the definition of $Y_t$ in this section is the inverse of the counterpart
in bond option section.

Since $Y_t$ is a $Q^N$-martingale, it is log-normal under $Q^N$ measure. And
because 
\[
  Y_t=\frac{D_{tT}}{D_{tS}}
     = \frac{A_{tT}}{A_{tS}} e^{(B_{tS}-B_{tT}) r_t},
\]
we get
\[
  \frac{dY_t}{Y_t} = (B_{tS}-B_{tT}) \sigma_t dW_t^N
    = \phi_t (\psi_S-\psi_T) \sigma_t dW_t^N,
\]
the solution is
\[
  Y_T = \frac{D_{tT}}{D_{tS}}
    \exp 
      \left(    
        (\psi_S-\psi_T) \int_t^T \phi_u \sigma_u dW_u^N
        -\frac{1}{2}(\psi_S-\psi_T)^2 \int_t^T \phi_u^2 \sigma_u^2 du
      \right).
\]
Again we can rewrite $Y_T$ as $\frac{D_{tT}}{D_{tS}}e^{X_t}$ where 
$X_t\sim N(-\frac{\sigma_p^2}{2},\sigma_p^2)$ is Gaussian and $\sigma_p$ is
defined in eq. \ref{E:sigma_p}. And the caplet price can be rewritten as
\begin{align*}
  V_t &= D_{tS} E_N[(Y_T-1-K\tau(T,S))^+|F_t] \notag \\
      &= D_{tS} E_N[(\frac{D_{tT}}{D_{tS}}e^{X_t}-1-K\tau(T,S))^+|F_t] \notag \\
      &= D_{tT} E_N[(e^{X_t}-\frac{D_{tS}(1+K\tau(T,S))}{D_{tT}})^+|F_t] 
         \notag \\
\end{align*}
Using eq. \ref{E:int_call} we get caplet price eq. \ref{E:caplet}. Similarly 
we can arrive at floorlet price eq. \ref{E:floorlet}.


%%%%%%%%%%%%%%%%%%%%%%%%%%%%%%%%%%%%%%%%%%%%%
\section{Tree Construction}
This section describes a two-stage procedure for constructing a trinomial 
tree that approximates the evolution of the process $(r_t)_t$:
\[
  d x_t = - a_t x_t dt + \sigma_t dW_t,
\]
and 
\[
  r_t = f(x_t + \varphi_t),
\]
where $f$ is a deterministic function and $\varphi_t$ is determined by the
initial yield curve. Note that in case $f(x)=x$ we have the Hull-White model,
and if $f(x)=e^x$ we have the Black-Karasinski model.

Let us fix a time horizon T and the times $0=t_0 < t_1 < \cdots < t_N=T$,
and set $\Delta t_i = t_{i+1} - t_i$ for each i. The time instances 
$t_i$ do not have to be equally spaced.

The first stage constructs a trinomial tree for process $(x_t)_t$,
the solution of the SDE is:
\[
  x_t = \frac{1}{\phi_t} 
        \left( \phi_s x_s + \int_s^t \phi_u \sigma_u dW_u \right) ,
        \qquad t>s,
\]
and the first two moments are
\[
  E[x_t | F_s] = \frac{x_s \phi_s}{ \phi_t },
\]
\[
  V[x_t | F_s] = \frac{\xi_t - \xi_s}{\phi_t^2}.
\]

The tree nodes are denoted by $(i,j)$, where the time index $i$ ranges from 
0 to $N$ and the space index $j$ ranges from some $\underline{j}_i < 0$ to 
some $\bar{j}_i>0$. We denote by $x_{i,j}$ the process value on node
$(i,j)$. We also have
\[
  E[x(t_{i+1})|x(t_i)=x_{i,j}] = \frac{\phi_i}{\phi_{i+1}} x_{i,j} \equiv
M_{i,j},
\]
\[
  V[x(t_{i+1})|x(t_i)=x_{i,j}] = \frac{\xi_{i+1}-\xi_i}{\phi_{i+1}^2} 
                   \equiv V_i^2.
\]
We then set $x_{i,j}=j \Delta x_i$, where 
$\underline{j}_i \leq j \leq \bar{j}_i$, and
\[
  \Delta x_i = V_{i-1} \sqrt{3}.
\]
Assuming that at time $t_i$ we are on node (i,j) with associated value
$x_{i,j}$, the process can move to $x_{i+1,k+1}$, $x_{i+1,k}$, or 
$x_{i+1,k-1}$ at time $t_{i+1}$ with probabilities $p_u$, $p_m$,
and $p_d$, respectively. The central node k is chosen so that
$x_{i+1,k}$ is as close as possible to $M_{i,j}$, i.e.,
\[
  k = \text{round} \left( \frac{M_{i,j}}{\Delta x_{i+1}}   \right).
\]

The probabilities can be solved using
\[
  \begin{cases}
    p_u + p_m + p_d = 1,      &\text{(total probability)} \\
    p_u (k+1) \Delta x_{i+1} + p_m k \Delta x_{i+1} + p_d (k-1) \Delta x_{i+1} 
      = M_{ij},               &\text{(match mean)}  \\
    p_u (k+1)^2 \Delta x_{i+1}^2 + p_m k^2 \Delta x_{i+1}^2 
      + p_d (k-1)^2 \Delta x_{i+1}^2 = M_{ij}^2 + V_i^2, 
      &\text{(match variance)},
  \end{cases}
\]
this yields
\[
  \begin{cases}
    p_u = \frac{1}{6} + \frac{\eta_{j,k}^2}{6 V_i^2} 
          + \frac{\eta_{j,k}}{2 \sqrt{3} V_i},  \\
    p_m = \frac{2}{3} - \frac{\eta_{j,k}^2}{3 V_i^2}  \\
    p_d = \frac{1}{6} + \frac{\eta_{j,k}^2}{6V_i^2} 
          - \frac{\eta_{j,k}}{2 \sqrt{3} V_i},
  \end{cases}
\]
where $\eta_{j,k}=M_{i,j}-x_{i+1,k}$. Note that from the definintion of k,
we get $|\eta_{j,k}|\leq \Delta x_{i+1}/2 = V_i \sqrt{3} /2$, and this
guarantees that all three are valid probabilities ($0\leq p \leq 1$).

The second stage consists in constructing the tree for $r$ using the tree for 
$x$ and the initial yield curve. 
%Let
%\[
%  r_t = f^{-1}( x_t + \alpha_t ),
%\]
%where $\alpha_t$ is to be determined. Note that
%\[
%  d\alpha_t = (\theta_t - a_t \alpha_t) dt,
%\]
%hence $\alpha$ is deterministic and is the same for each node at a fixed
%time. However, we can not solve it explicitly because $\theta_t$ is not
%known yet. Thus we have
We have the discrete version
\[
  r_{i,j} = f(x_{i,j} + \varphi_i) = f(j \Delta x_i + \varphi_i).
\]

Let $Q_{i,j}$ be the price at time zero of an instrument paying 1 if node
(i,j) is reached and zero otherwise, i.e.,
\[
  Q_{i,j} = E \left[ e^{-\int_0^{t_i} r_u du} 1_{r(t_i)=r_{i,j}} \right],
\]
we get the following recursive formula
\begin{align*}
  Q_{i+1,j} &= E \left[ e^{-\int_0^{t_i} r_u du} 
                E \left[
                    e^{-\int_{t_i}^{t_{i+1}} r_u du} 1_{r_{(t_{i+1})=r_{i+1,j}}} 
                    | F_i 
                  \right] 
                \right]  \notag \\
    &= \sum_h Q_{i,h} \, p_{(i,h),(i+1,j)} 
      \exp( -f(\varphi_i+h\Delta x_i) \Delta t_i),
\end{align*}
where $p_{(i,h),(i+1,j)}$ is the probability from node (i,h) to node
(i+1,j).

Given the initial yield curve, or equivalently the intial bond prices
\[
  P(0,t_i) = E \left[ e^{-\int_0^{t_i} r_u du} \right],
\]
we can determine the value of $\varphi$. Note that
\begin{align*}
  P(0, t_{i+1}) &= \sum_j Q_{i+1,j}  \notag \\
     &= \sum_h Q_{i,h}
      \exp( -f(\varphi_i+h\Delta x_i)\, \Delta t_i),
\end{align*}
where we have used the recursive formula for $Q_{i+1,j}$ and the fact that
$\sum_j p_{(i,h),(i+1,j)}=1$.

Starting from time step zero, we have $Q_{0,0}=1$. At any time
step i, we first calculate $\varphi_i$ using the price of bond matured at the
next time step, i.e., $P(0,t_{i+1})$. Next we use this value to calculate 
$Q_{i+1,j}, \underline{j}_{i+1}\leq j \leq \bar{j}_{i+1}$ at time step i+1.


%%%%%%%%%%%%%%%%%%%%%%%%%%%%%%%%
\section{Backward Induction} \label{S:back_ind}
One method to price instruments using a trinomial tree is backward induction.
For instance, the zero curve at each node $(i,j)$ is calculated using backward
induction after a tree is built. The basic idea of backward induction is 
that for $t<s$
\begin{align*}
  V_t &= E \left[ e^{-\int_t^T r_u du} V_T | F_t \right]  \notag \\
      &= E \left[ 
            E \left[ e^{-\int_t^T r_u du} V_T | F_s \right]  | F_t
           \right] \notag \\
      &= E \left[ e^{-\int_t^s r_u du} V_s | F_t \right].
\end{align*}
The recursive formula under the discrete setting is:
\begin{align} \label{E:back_ind1}
  V_{i,j} &= E \left[ e^{-\int_{t_i}^{t_{i+1}} r_u du} V(t_{i+1}) \right] 
             \notag \\
          &= \sum_k e^{-r_{ij} \Delta t_i} p_{(i,j),(i+1,k)} V_{i+1,k}.
\end{align}

%%%%%%%%%%%%%%%%%%%%%%%%%%%%%%%%%%%%%%%
%%   backward induction
%%%%%%%%%%%%%%%%%%%%%%%%%%%%%%%%%%%%%%%
\begin{figure}
  \begin{tikzpicture}[scale=1]
    \coordinate (A) at (0,0);
    \coordinate (B) at (5,1);
    \coordinate (C) at (5,-1);
    \coordinate (D) at (5,-3);

    \draw (A) node[left]{$r_{ij},V_{ij}$} 
          -- (B) node[right]{$r_{i+1,k+1},V_{i+1,k+1}$} node[pos=0.5,above]{$p_u$};
    \draw (A) node[left]{$r_{ij},V_{ij}$} 
          -- (C) node[right]{$r_{i+1,k},V_{i+1,k}$} node[pos=0.5,above]{$p_m$};
    \draw (A) node[left]{$r_{ij},V_{ij}$} 
          -- (D) node[right]{$r_{i+1,k-1},V_{i+1,k-1}$} node[pos=0.5,above]{$p_d$};

    \draw[fill=black,color=black] (A) circle (1pt);
    \draw[fill=black,color=black] (B) circle (1pt);
    \draw[fill=black,color=black] (C) circle (1pt);
    \draw[fill=black,color=black] (D) circle (1pt);

    \draw (0,-4) node[below]{$t_i$} -- (0,2) ;
    \draw (5,-4) node[below]{$t_{i+1}$} -- (5,2) ;
    \draw[->] (3.5,1.5) -- (1.5,1.5);

    \draw (2.5,-3.5) node{$\Delta t_i$} ;
    \draw[->] (3,-3.5) -- (4.5,-3.5);
    \draw[->] (2,-3.5) -- (0.5,-3.5);

  \end{tikzpicture} 
  \caption{Backward induction step for the payoff evaluation on the tree moving
           backward from time $t_{i+1}$ to time $t_i$.}
\end{figure}
%%%%%%%%%%%%%%%%%%%%%%%

In case of a trinomial tree, this becomes
\begin{equation}
  V_{ij} = e^{-r_{ij}\Delta t_i} 
           (p_u V_{i+1,k+1} + p_m V_{i+1,k} + p_d V_{i+1,k-1} ).
\end{equation}

Backward induction can also used to price instruments with early exercise
(e.g. Bermudan or American). If $t_i$ is an exercise date, the formula 
becomes
\begin{align} \label{E:back_ind2}
  V_{i,j} 
    &= \max 
         \left( 
           E \left[ e^{-\int_{t_i}^{t_{i+1}} r_u du} V(t_{i+1}) \right],
           \, I_{ij}
         \right) \notag \\
    &= \max
         \left( 
           \sum_k e^{-r_{ij} \Delta t_i} p_{(i,j),(i+1,k)} V_{i+1,k}, 
           \, I_{ij}
         \right),
\end{align}
where $I_{ij}$ is the intrinsic value of the underlying instrument (i.e. the
payoff if the option is exercised) at node $(i,j)$. If $t_i$ is not an 
exercise date, equation \ref{E:back_ind1} is used. In case of a trinomial 
tree, this becomes
\begin{equation}
  V_{ij} 
    = \max 
        \left(
          \left[ 
            e^{-r_{ij}\Delta t_i} 
            (p_u V_{i+1,k+1} + p_m V_{i+1,k} + p_d V_{i+1,k-1} ) 
          \right],\, I_{ij}
        \right).
\end{equation}
Note that if $t_i$ is the last exercise date, we have instead
\begin{equation}
  V_{ij} = \max (0, I_{ij} ),
\end{equation}
which is consistent with the payoff of an European option.

%%%%%%%%%%%%%%%%%%%%%%%%%%%%%%%%
\section{Random Path Generation from a Trinomial Tree}
\begin{enumerate}
  \item Start from time step $t_0$, choose the only node.
  \item At time step $t_i$, for each node $(i,j)$, retrieve the possible
destinations $(i+1,k+1)$, $(i+1,k)$, and $(i+1,k-1)$ at time step 
$t_{i+1}$ and the corresponding transition probabilities $p_u$,
$p_m$, and $p_d$.
  \item Generate an uniform random number U in $[0,1]$ using the random number
generator, and the node at time step i+1 is
    \begin{itemize}
      \item $k+1$ if $U<=p_u$;
      \item $k$ if $p_u<U<=p_u+p_m$;
      \item $k-1$ if $p_u+p_m<U$.
    \end{itemize}
  \item Increase i by 1 and repeat step 2 until the end of the tree.
\end{enumerate}

%%%%%%%%%%%%%%%%%%%%%%%%%%%%%%%%
\section{Pricing Interest-rate Products}
In this section we describe the pricing of interest-rate products using the
Hull-White model and the Black-Karasinski model.

\begin{enumerate}
\item Swaps and noncallable bonds can be priced without an interest rate model.


\item Options (European, American, and Bermudan) on bonds and swaps can be 
priced using backward induction as described in Section \ref{S:back_ind}.
Analytic formulas are also available for Hull-White model.

\item Caps and floors can be priced using backward induction. 
Analytic formulas are also available for the Hull-White model.


\item Mortgage backed securities are priced using Monte Carlo method that
samples interest rate paths from the short-rate tree.
\end{enumerate}


%%%%%%%%%%%%%%%%%%%%%%%%%%%%%%%%
\section{Model Calibration}
There are two parts to the calibration of short rate models: (1) calibration (of 
function $\theta_t$) to initial bond prices and (2) calibration (of a and
$\sigma_t$) to option prices. The first part can be performed precisely using 
either analytic formula (if available) or the trinomial tree. The second part 
is performed using a "goodness-of-fit" measure and an optimization
method, such as the Levenberg-Marquardt method (see Appendix \ref{C:levenberg}).





