\chapter{Bessel Process}

%%%%%%%%%%%%%%%%%%%%%%%%%%%%
\section{Introduction}
Bessel process with dimension $\delta$ is described by SDE:
\begin{equation}
  dR_t = dW_t + \frac{\delta-1}{2R_t} dt.
\end{equation}

Squared Bessel process (BESQ) with dimension $\delta$ is described by SDE:
\begin{equation}
  d\rho_t = 2\sqrt{\rho_t} dW_t + \delta dt.
\end{equation}

%%%%%%%%%%%%%%%%%%%%%%%%%%%%
\section{Transition Densities}

\begin{lemma} \label{L:pde_besq}
The solution of the partial differential equation
\begin{equation}
  \frac{\partial}{\partial t} u(t,x) = 2x \frac{\partial^2}{\partial x^2} u(t,x)
                                  + \delta \frac{\partial}{\partial x} u(t,x),
\end{equation}
with initial condition $u(0,x)=e^{-\lambda x}$ is
\begin{equation}
  u(t,x) = \frac{1}{(1+2\lambda t)^{\delta /2}} 
           \exp \left(-\frac{\lambda x}{1+2\lambda t} \right).
\end{equation}
\end{lemma}
\begin{proof}
The following is a heuristic proof.
\footnote{How to solve this rigorously?}
First we solve a simpler setting with $\delta=0$, i.e.
\[
  \frac{\partial}{\partial t} u(t,x) = 2x \frac{\partial^2}{\partial x^2} u(t,x)
\]
with initial condition $u(0,x)=e^{-\lambda x}$. We guess the solution to be
\[
  u(t,x) = \exp \left( -\lambda x f(t) \right),
\]
thus $f(t)$ satisfies $f(0)=1$ and
\[
  f'(t)+2\lambda f^2(t) = 0.
\]
Solve this we get
\[
  f(t) = \frac{1}{1+2\lambda t}.
\]
hence we have the solution for the case with $\delta=0$
\[
  u(t,x) = \exp \left(-\frac{\lambda x}{1+2\lambda t} \right).
\]

Next we tackle the case with nonzero $\delta$. This time we guess that the
solution is something like
\[
  u(t,x) = g(t) \exp \left(-\frac{\lambda x}{1+2\lambda t} \right).
\]
With this guess, we have $g(0)=1$ and
\[
  g'(t) = -\frac{\delta \lambda}{1+2\lambda t} g(t).
\]
Solve this we get
\[
  g(t) = \frac{1}{(1+2\lambda t)^{\delta /2}}, 
\]
and the solution of the origial PDE
\[
  u(t,x) = \frac{1}{(1+2\lambda t)^{\delta /2}} 
           \exp \left(-\frac{\lambda x}{1+2\lambda t} \right).
\]
\end{proof}

%%%%%%%%%%%%%%%%%%%%%%%%%%%
\begin{theorem} \label{T:besq_exp}
For a squared Bessel process (BESQ) $\rho_t$ with dimension $\delta$ 
we have
\begin{equation}
  Q_x^{\delta}[\exp(-\lambda\rho_t)] 
    = \frac{1}{(1+2\lambda t)^{\delta /2}} 
      \exp \left(-\frac{\lambda x}{1+2\lambda t} \right).
\end{equation}
\begin{proof}
From the definition of BESQ, the infinitestimal generator is 
\[
  L = 2x\frac{\partial^2}{\partial x^2} + \delta\frac{\partial}{\partial x}.
\]
Thus from the Feynman-Kac formula (for the general diffusion process) we get
that $Q_x^{\delta}[\exp(-\lambda\rho_t)]$ is the unique solution of PDE
\[
  \frac{\partial}{\partial t} u(t,x) = 2x \frac{\partial^2}{\partial x^2} u(t,x)
                                  + \delta \frac{\partial}{\partial x} u(t,x),
\]
with initial condition $u(0,x)=e^{-\lambda x}$. From Lemma \ref{L:pde_besq} we
prove the theorem.
\end{proof}
\end{theorem}

%%%%%%%%%%%%%%%%%%%%%%%%
\begin{proposition}
Let $\delta=2\nu+2$, the transition densition $q_t^{(\nu)}$ of a $BESQ^{\nu}$
with dimension $\delta$ and $\nu>-1$ is
\begin{equation}
  q_t^{\nu}(x,y) = \frac{1}{2t} \left(\frac{y}{x}\right)^{\nu/2}
    \exp\left( -\frac{x+y}{2t} \right)  I_{\nu}\left(\frac{\sqrt{xy}}{t}\right),
\end{equation}
where $I_{\nu}$ is the modified Bessel function \ref{D:bessel_mod}.
\begin{proof}
First note that
\[
  q_t^{\nu}(x,y) = L_{\lambda}^{-1}[Q_x^{\delta}[e^{-\lambda \rho_t}]].
\]
From Theorem \ref{T:besq_exp} we get
\begin{align*}
  q_t^{\nu}(x,y) 
    &= L_{\lambda}^{-1}
      \left[
        \frac{1}{(1+2\lambda t)^{\delta /2}} 
        \exp \left(-\frac{\lambda x}{1+2\lambda t} \right)
      \right]  \notag \\
    &= e^{-x/2t} L_{\lambda}^{-1}
      \left[
        \frac{1}{(1+2\lambda t)^{\nu+1}} 
        \exp \left(\frac{x/2t}{1+2\lambda t} \right)
      \right].  \notag \\
\end{align*}
Using the property of inverse Laplace transform
\[
  L_{\lambda}^{-1}[F(a\lambda+b)] 
    =\frac{1}{a} e^{-by/a} f\left(\frac{y}{a}\right),
\]
we get
\[
  q_t^{\nu}(x,y) 
    = e^{-(x+y)/2t} \left(\frac{1}{2t}\right)
      L_{\lambda}^{-1}  \left[
        \lambda^{-\nu-1} \exp\left(\frac{x/2t}{\lambda}\right)
      \right].
\]
Using inverse Laplace transform \ref{E:ilt5} we prove the proposition.
\end{proof}
\end{proposition}

%%%%%%%%%%%%%%%%%%%%%%%%%%%%%
\begin{proposition}
Let $\delta=2\nu+2$, the transition densition $p_t^{(\nu)}$ of a Bessel process
with dimension $\delta$ and $\nu>-1$ is
\begin{equation}
  p_t^{\nu}(x,y) = \frac{y}{t} \left(\frac{y}{x}\right)^{\nu}
    \exp\left( -\frac{x^2+y^2}{2t} \right)  I_{\nu}\left(\frac{xy}{t}\right),
\end{equation}
where $I_{\nu}$ is the modified Bessel function \ref{D:bessel_mod}.
\begin{proof}
Note that first $\rho_t=R_t^2$ thus $\rho_0=x^2$ and
\[
  P[y<R_t<y+dy]=p_t(x,y)dy = P[y^2<\rho_t<(y+dy)^2]=q_t(x^2,y^2) 2ydy,
\]
i.e.
\begin{equation} \label{E:bes_besq}
  p_t(x,y) = 2y \, q_t(x^2,y^2).
\end{equation}
Using the transition density of BESQ we then get the transition density of the
Bessel process $p_t(x,y)$.
\end{proof}
\end{proposition}


%%%%%%%%%%%%%%%%%%%%%%%%%%%%%%%%%%%%%%%%%%%%%%%%%%%%%%%%%%%%%%%%%%%%%%%%%%%%%%%%%%
\section{Bessel Process with Arbitrary Index}

We define the Bessel process $R_t$ with arbitrary index $\nu\in R$ by the
stochastic differential equation
\begin{equation}
  dR_t = \frac{2\nu+1}{2R_t} dt + dW_t,
\end{equation}
thus the dimension is $\delta=2\nu+2$. The infinitesimal generator is
\begin{equation}
  \mathcal{G} = \frac{1}{2} \frac{d^2}{dx^2} + \frac{2\nu+1}{2x} \frac{d}{dx}.
\end{equation}
In this section we compute the transition density following Section
\ref{S:diff}.

First note that for Bessel process with index $\nu\in R$, we can easily compute
the scale density $s'(x)=x^{-1-2\nu}$ and speed density $m(x)=2x^{1+2\nu}$.

Next we solve the corresponding Sturm-Liouville equation
\[
  (\mathcal{G}-\alpha) u(x) = 0,
\]
i.e.
\[
  \frac{1}{2} \frac{d^2}{dx^2} u(x) + \frac{2\nu+1}{2x} \frac{d}{dx} u(x)
    - \alpha u(x) = 0.
\]
Let $u(x)=x^{-\nu} v(x)$, we get
\[
  x^2 v''(x) + x v'(x) - (\nu^2+2\alpha) v(x) = 0.
\]
Using a simple variable change $y=\sqrt{2\alpha} x$ and the definition 
\ref{D:bessel_mod} of the modified Bessel functions we get
\[
  v(x) = A I_{\nu}(\sqrt{2\alpha}x) + B K_{\nu}(\sqrt{2\alpha}x),
\]
where $A$ and $B$ are arbitrary constants, hence the solution of the
Sturm-Liouville equation for the Bessel process is
\[
  u(x) = A x^{-\nu} I_{\nu}(\sqrt{2\alpha}x) + 
         B x^{-\nu} K_{\nu}(\sqrt{2\alpha}x).
\]

First we observe that
\[
  \lim_{x\to\infty} x^{-\nu} I_{\nu}(\sqrt{2\alpha}x) \to\infty,
\]
hence when $x\to\infty$, the solution is
\[
  \phi(x) = x^{-\nu} K_{\nu}(\sqrt{2\alpha}x).
\]

We then proceed to solve the boundary value problem at $x=0$. 
For convenience we rewrite the general solution as
\footnote{Note from the general definition of $K_{\nu}$, any two functions out
    of $I_{\nu}$,$I_{-\nu}$, and $K_{\nu}$ are linear independent.}
\[
  u(x) = A' x^{-\nu} I_{\nu}(\sqrt{2\alpha}x) + 
         B' x^{-\nu} I_{-\nu}(\sqrt{2\alpha}x).
\]
For the Dirichlet boundary condition $\lim_{x\to 0} f(x)=0$, we observe 
that for small value of $|x|$ we have
\[
  x^{-\nu} I_{\nu}(a x) = \frac{(a/2)^{\nu}}{\Gamma(\nu+1)} 
                          \left( 1+ O(x^2) \right),
\]
and 
\[
  x^{-\nu} I_{-\nu}(a x) = \frac{(a/2)^{-\nu} x^{-2\nu}}{\Gamma(-\nu+1)} 
                          \left( 1+ O(x^2) \right),
\]
hence the solution for the Dirichlet condition is
\[
  \psi(x) = x^{-\nu} I_{-\nu}(\sqrt{2\alpha} x), \qquad (\nu<0).
\]

Similarly for the Neumann boundary condition
\[
  \lim_{x\to 0} \frac{f'(x)}{s'(x)}=0,
\]
where the scale density $s'(x)=x^{-1-2\nu}$,
we observe that for small $|x|$ we have
\[
  \frac{1}{s'(x)} \frac{d}{dx} [x^{-\nu} I_{\nu}(a x)] 
    = 2\frac{(a/2)^{\nu+2} x^{2\nu+2}}{\Gamma(\nu+2)} \left( 1+ O(x^2) \right),
\]
and 
\[
  \frac{1}{s'(x)} \frac{d}{dx} [x^{-\nu} I_{-\nu}(a x)] 
    = -2\nu \frac{(a/2)^{-\nu}}{\Gamma(-\nu+1)} \left( 1+ O(x^2) \right),
\]
hence the solution for the Neumann boundary is
\[
  \psi(x) = x^{-\nu} I_{\nu}(\sqrt{2\alpha} x), \qquad (\nu>-1).
\]

Next we calculate the Wronskian $w_{\alpha}$ defined by
\[
  w_{\alpha} = \frac{1}{s'(x)} (\phi_{\alpha}(x) \psi_{\alpha}'(x)
                                - \psi_{\alpha}(x) \phi_{\alpha}'(x) ).
\]
We know that $w_{\alpha}$ is independent of $x$ for any two independent
solutions $\phi_{\alpha}(x)$ and $\psi_{\alpha}(x)$ of the Sturm-Liouville
equation, hence 
\[
  w_{\alpha}[\phi_{\alpha}(x),\psi_{\alpha}(x)] = w_{\alpha}(x=0).
\]
From the asymptic we used above ($a=\sqrt{2\alpha}$), we get the 
\begin{align*}
  w_{\alpha}[x^{-\nu} I_{\nu}(ax), x^{-\nu} I_{-\nu}(ax)]
  &= \lim_{x\to 0} w_{\alpha}[x^{-\nu} I_{\nu}(ax), x^{-\nu} I_{-\nu}(ax)]
     \notag \\
  &= \frac{(a/2)^{\nu}}{\Gamma(\nu+1)} 
     \frac{-2\nu (a/2)^{-\nu}}{\Gamma(-\nu+1)} \left( 1+ O(x^2) \right)
     \notag \\
  &= \frac{-2\nu}{\Gamma(\nu+1)\Gamma(1-\nu)}
     = \frac{-2}{\Gamma(\nu)\Gamma(1-\nu)}, \notag 
\end{align*}
using Euler's reflection formula \ref{P:gamma_pp}, we get
\begin{equation}
  w_{\alpha}[x^{-\nu} I_{\nu}(\sqrt{2\alpha} x), 
    x^{-\nu} I_{-\nu}(\sqrt{2\alpha}x)]
  = \frac{-2\pi}{\sin(\pi\nu)},
\end{equation}
and hence
\begin{equation}
  w_{\alpha}[x^{-\nu} K_{\nu}(\sqrt{2\alpha} x), 
    x^{-\nu} I_{\pm\nu}(\sqrt{2\alpha}x)] = 1.
\end{equation}

Hence under the Dirichlet boundary condition, the Green's function is
\footnote{Borodin \& Salminen, Handbook of Brownian Motion, 2nd ed. (2002), 
    p.133}
\begin{equation}
  G_{\alpha}(x,y) = 
    \begin{cases}
      x^{-\nu} K_{\nu}(\sqrt{2\alpha} x) y^{-\nu} I_{-\nu}(\sqrt{2\alpha}y)
        & x\ge y  \\
      x^{-\nu} I_{-\nu}(\sqrt{2\alpha}x) y^{-\nu} K_{\nu}(\sqrt{2\alpha} y) 
        & x\le y
    \end{cases}
\end{equation}
it exists only if $\nu<0$. And under the Neumann boundary condition, the Green's
function is
\begin{equation}
  G_{\alpha}(x,y) = 
    \begin{cases}
      x^{-\nu} K_{\nu}(\sqrt{2\alpha} x) y^{-\nu} I_{\nu}(\sqrt{2\alpha}y)
        & x\ge y  \\
      x^{-\nu} I_{\nu}(\sqrt{2\alpha}x) y^{-\nu} K_{\nu}(\sqrt{2\alpha} y) 
        & x\le y
    \end{cases}
\end{equation}
it exists only if $\nu>-1$. 

And we can compute the transition density with respect to the speed measure
$p_m(t,x,y)$ as the inverse Laplace transform of the Green's function:
\footnote{Borodin \& Salminen, Handbook of Brownian Motion, 2nd ed. (2002), 
    p.134}
\[
  p_m(t,x,y) = L_{\alpha}^{-1}[G_{\alpha}(x,y)].
\]
Using the inverse Laplace transform \ref{E:ilt6}, we thus get that under the
Dirichlet boundary condition
\begin{equation}
  p_m(t,x,y) = \frac{(xy)^{-\nu}}{2t} \exp\left( -\frac{x^2+y^2}{2t} \right)
      I_{-\nu} \left( \frac{xy}{t} \right) \qquad \nu<0,
\end{equation}
and under the Neumann boundary condition
\begin{equation}
  p_m(t,x,y) = \frac{(xy)^{-\nu}}{2t} \exp\left( -\frac{x^2+y^2}{2t} \right)
      I_{\nu} \left( \frac{xy}{t} \right) \qquad \nu>-1.
\end{equation}
And because speed density is $m(x)=2x^{1+2\nu}$, we can then compute
the transition density in the normal measure $p(t,x,y)$ using the fact that
\[
  p_m(t,x,y) m(y) dy = p(t,x,y) dx.
\]
Thus under the Dirichlet boundary condition
\begin{equation}
  p(t,x,y) = \frac{y}{t} \left( \frac{y}{x} \right)^{\nu} 
     \exp\left( -\frac{x^2+y^2}{2t} \right)
    I_{-\nu} \left( \frac{xy}{t} \right) \qquad \nu<0,
\end{equation}
and under the Neumann boundary condition
\begin{equation}
  p(t,x,y) = \frac{y}{t} \left( \frac{y}{x} \right)^{\nu} 
     \exp\left( -\frac{x^2+y^2}{2t} \right)
      I_{\nu} \left( \frac{xy}{t} \right) \qquad \nu>-1.
\end{equation}

Using Eq. \ref{E:bes_besq}, we get the transition density of a $BESQ^{\nu}$
process: under the Dirichlet boundary condition
\begin{equation}
  q(t,x,y) = \frac{1}{2t} \left( \frac{y}{x} \right)^{\nu/2} 
     \exp\left( -\frac{x+y}{2t} \right)
    I_{-\nu} \left( \frac{\sqrt{xy}}{t} \right) \qquad \nu<0,
\end{equation}
and under the Neumann boundary condition
\begin{equation}
  q(t,x,y) = \frac{1}{2t} \left( \frac{y}{x} \right)^{\nu/2} 
     \exp\left( -\frac{x+y}{2t} \right)
      I_{\nu} \left( \frac{\sqrt{xy}}{t} \right) \qquad \nu>-1.
\end{equation}
