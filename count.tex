\chapter{Counting}

%%%%%%%%%%%%%%%%%%%%%%%%%%%%%%%%%%%%%%%%%%%
\section{Dollar Changing Problem (Polya)}

Problem. 
\footnote{Polya and Szego, Problems and Theorems in Analysis I, Part I, 
          Problem 1, (see also Problem 2), p.25 and p.173.
          Sedgewick and Flajolet, An Introduction to the Analysis
          of Algorithms, 2nd ed., p.126 }
In how many different ways can you change a dollar? That is, in how many 
different ways can you pay 100 cents using five different kinds of coins,
cents, nickels, dimes, quarters and half-dollars (worth 1, 5, 10, 25, and
50 cents, respectively)?

Solution. Here we use a method with generating function. The problem 
above is equivalent to the problem of calculating coefficient $e_{100}$
where 
\begin{align*}
  \sum_{n=0}^{\infty} e_n z^n
    =& (1 + z + z^2 + z^3 + \cdots + z^{\alpha} + \cdots) \notag \\
    & (1 + z^5 + z^{10} + z^{15} + \cdots + z^{5\beta} + \cdots) \notag \\
    & (1 + z^{10} + z^{20} + z^{30} + \cdots + z^{10\gamma} + \cdots) \notag \\
    & (1 + z^{25} + z^{50} + z^{75} + \cdots + z^{10\mu} + \cdots) \notag \\
    & (1 + z^{50} + z^{100} + z^{150} + \cdots + z^{10\nu} + \cdots) \notag \\
    =& \frac{1}{(1-z)(1-z^5)(1-z^{10})(1-z^{25})(1-z^{50})} 
\end{align*}
  
To calculate coefficient $e_n$, let
\[
  \frac{1}{1-z} = a_0 + a_1 z + a_2 z^2 + \cdots
\]
\[
  \frac{1}{(1-z)(1-z^5)} = b_0 + b_1 z + b_2 z^2 + \cdots
\]
\[
  \frac{1}{(1-z)(1-z^5)(1-z^{10})} = c_0 + c_1 z + c_2 z^2 + \cdots
\]
\[
  \frac{1}{(1-z)(1-z^5)(1-z^{10})(1-z^{25})} = d_0 + d_1 z + d_2 z^2 + \cdots
\]
\[
  \frac{1}{(1-z)(1-z^5)(1-z^{10})(1-z^{25})(1-z^{50})} 
     = e_0 + e_1 z + e_2 z^2 + \cdots
\]
it is easy to see that $a_i=1$ and
\[
  b_i = 
    \begin{cases}
      a_i,             & i<5  \\
      a_i + b_{i-5},   & i\ge 5  
    \end{cases}
\]
\[
  c_i = 
    \begin{cases}
      b_i,              & i<10  \\
      b_i + c_{i-10},   & i\ge 10  
    \end{cases}
\]
\[
  d_i = 
    \begin{cases}
      c_i,              & i<25  \\
      c_i + d_{i-25},   & i\ge 25  
    \end{cases}
\]
\[
  e_i = 
    \begin{cases}
      d_i,              & i<50  \\
      d_i + e_{i-50},   & i\ge 50  
    \end{cases}
\]

The coefficients for $a_i,b_i,c_i,d_i,e_i$ are given in the following table
where the coeffients are the same for indexes from $0$ to $4$, from $5$ to
$9$, from $10$ to $15$, etc.
%%%%%%%%%%%%%%%%%%%%%%%%%%%%%%%%%%%%%%%%%%%
\begin{table*}
%\caption{Coefficients of generating functions}
\begin{tabular}{r|r|r|r|r|r|r|r|r|r|r|r|r|r|r|r|r|r|r|r|r|r}
\hline\hline
$i$   & 0 & 5 &10 &15 &20 & 25 & 30 & 35 & 40 & 45 & 50 & 55 & 60 & 65 &  70 &  75 &  80 &  85 &  90 &  95 & 100 \\
\hline
$a_i$ & 1 & 1 & 1 & 1 & 1 &  1 &  1 &  1 &  1 &  1 &  1 &  1 &  1 &  1 &   1 &   1 &   1 &   1 &   1 &   1 &   1 \\
$b_i$ & 1 & 2 & 3 & 4 & 5 &  6 &  7 &  8 &  9 & 10 & 11 & 12 & 13 & 14 &  15 &  16 &  17 &  18 &  19 &  20 &  21 \\
$c_i$ & 1 & 2 & 4 & 6 & 9 & 12 & 16 & 20 & 25 & 30 & 36 & 42 & 49 & 56 &  64 &  72 &  81 &  90 & 100 & 110 & 121 \\
$d_i$ & 1 & 2 & 4 & 6 & 9 & 13 & 18 & 24 & 31 & 39 & 49 & 60 & 73 & 87 & 103 & 121 & 141 & 163 & 187 & 213 & 242 \\
$e_i$ & 1 & 2 & 4 & 6 & 9 & 13 & 18 & 24 & 31 & 39 & 50 & 62 & 77 & 93 & 112 & 134 & 159 & 187 & 218 & 252 & 292 \\
\hline
\end{tabular}
\end{table*}

Hence the answer to our problem is $e_{100}=292$.
