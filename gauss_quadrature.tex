%%%%%%%%%%%%%%%%%%%%%%%%%%%%%%%%
\chapter{Gaussian Quadrature}


%%%%%%%%%%%%%%%%%%%%%%%%%%%%%%%%
\section{Starting Point}

We try to approximate an integral by a summation
\begin{equation}
  \int_a^b f(x) dx = c_1 f(x_1) + c_2 f(x_2) + \cdots + c_n f(x_n) + R(f),
\end{equation}
where $R(f)$ is the residual, and $c_1,c_2,\cdots,c_n$, and $x_1,x_2,\cdots,x_n$
are unknowns. Since there are $2n$ unknowns, we need $2n$ equations to
determine them. This can be solved by assuming the above approximation 
is precise (i.e. $R(f)=0$) for polynomials of degree $2n-1$.
Note that Gauss 
\footnote{Gauss, C.F., Methodus Nova Integralium Valores Per Approximationem
Inveniendi, in Werke, vol. 3, 163-196.}
used continued fraction of his hypergeometric function 
\footnote{Gauss, C.F., Disquisitiones Generales circa Seriem infinitam
$1+\frac{\alpha\beta}{1\times\gamma}x
+\frac{\alpha(\alpha+1)\beta(\beta+1)}{1\cdot2\cdot\gamma(\gamma+1)}xx
+\frac{\alpha(\alpha+1)(\alpha+2)\beta(\beta+1)(\beta+2)}
      {1\cdot2\cdot3\cdot\gamma(\gamma+1)(\gamma+2)}x^3 + etc.$,
in Werke, vol. 3, 123-162.}
to derive his quadrature. 
Here we present an alternative derivation from Berezin and Zhidkov,
Computing Methods, vol. 1.


%%%%%%%%%%%%%%%%%%%%%%%%%%%%%%%%
\section{Abscissae}

%%%%%%%%%%%%%%%%
\begin{lemma}
\footnote{Berezin and Zhidkov, Computing Methods, vol.1, pp.236-240}
The abscissae $x_1,x_2,\cdots,x_n$ are roots of $n$-th order Legendre polynomial
\begin{equation} \label{E:abs}
  w_n(x) = \frac{n!}{(2n)!} \frac{d^n}{dx^n} 
    \left[ (x-a)^n (x-b)^n \right]
    = (x-x_1) (x-x_2) \cdots (x-x_n).
\end{equation}
\end{lemma}
\begin{proof}
For any polynomial $q(x)$ of $(n-1)$-th order, $w_n(x) q(x)$ is of $(2n-1)$-th
order, hence
\begin{equation} \label{E:zeroint}
  \int_a^b w_n(x) q(x) dx = \sum_{k=1}^n c_k w_n(x_k) q(x_k) = 0.
\end{equation}
Let 
\begin{equation}
  \phi_1(x) = \int_a^x w_n(y) dy,
\end{equation}
and
\begin{equation}
  \phi_{i+1}(x) = \int_a^x \phi_i(y) dy  \qquad i>0,
\end{equation}
we have
\begin{align*}
  0 &= \int_a^b w_n(x) q(x) dx   \notag \\
    &= [q(x)\phi_1(x) - q'(x)\phi_2(x)+\cdots 
        + (-1)^{n-1}q^{(n-1)}(x)\phi_n(x)]|_a^b. \notag \\
\end{align*}
Since for any $i=1,2,\cdots,n$ we have $\phi_i(a)=0$ and $q(x)$ is an 
arbitrary polynomial, we must have $\phi_i(b)=0$ too. An obvious choice would be
\[
  \phi_n(x) = C (x-a)^n (x-b)^n,
\]
where $C$ is a constant. Since
\[
  w_n(x) = \frac{d^n}{dx^n} \phi_n(x) = \prod_{k=1} (x-x_k),
\]
we conclude that 
\[
  C = \frac{n!}{(2n)!}.
\]
\end{proof}


%%%%%%%%%%%%%%%%%%%%%%%%%%%%%%%%
\section{Weights}

\begin{lemma}
\footnote{Berezin and Zhidkov, Computing Methods, vol.1, pp.241-243}
The Weights satisfy
\begin{equation} \label{E:weight}
  C_k = \frac{(n!)^4 (b-a)^{2n+1}}{ [(2n)!]^2 (x_k-a) (b-x_k) [w'_n(x_k)]^2 }
    \qquad k=1,2,\cdots,n.
\end{equation}
\end{lemma}
\begin{proof}
Let 
\[
  \psi_k(x) = \frac{w_n(x)}{x-x_k}
    = (x-x_1)(x-x_2)\cdots (x-x_{k-1})(x-x_{k+1})\cdots (x-x_n),
\]
and it is a polynomial of degree $n-1$, hence $\psi_k^2(x)$ is of degree $2n-2$,
thus
\[
  \int_a^b \psi_k^2(x) dx = \sum_i C_i \psi_k^2(x_i) = C_k \psi_k^2(x_k).
\]
Integrate by parts, and using
\[
  w'_n(x) = \frac{d}{dx} (x-x_1)(x-x_2)\cdots (x-x_n) 
          = \sum_{i=1}^n \frac{w_n(x)}{x-x_i},
\]
we have
\begin{align*}
  \int_a^b \psi_k^2(x) dx 
    &= \int_a^b \frac{w_n^2(x)}{(x-x_k)^2} dx  \notag \\
    &= - \frac{w_n^2(x)}{x-x_k}\Big|^b_a + 2\int_a^b w_n(x)\frac{w'_n(x)}{x-x_k} dx
      \notag \\
    &= - \frac{w_n^2(x)}{x-x_k} \Big|^b_a 
       + 2\int_a^b w_n(x) \sum_{i=1}^n \frac{w_n(x)}{(x-x_i)(x-x_k)} dx 
      \notag \\
    &= - \frac{w_n^2(x)}{x-x_k} \Big|^b_a 
       + 2\int_a^b \frac{w_n^2(x)}{(x-x_k)^2} dx.
       \qquad \text{(Using Eq. \ref{E:zeroint})}
      \notag \\
\end{align*}
Thus
\[
  \int_a^b \psi_k^2(x) dx = \frac{w_n^2(x)}{x-x_k} \Big|^b_a 
    = \frac{(n!)^4 (b-a)^{2n+1}}{ [(2n)!]^2 (x_k-a) (b-x_k) }.
\]
Using this and the fact that $\psi_k(x_k)=w'_n(x_k)$, we get Eq. \ref{E:weight}.
\end{proof}


%%%%%%%%%%%%%%%%%%%%%%%%%%%%%%%%
\section{Error}
To get the residual term $R(f)$ of the Gaussian quadrature, we first need the
error of Hermite interpolation.

%%%%%%%%%%%%%%%%%%%
\begin{lemma}[Error of Hermite interpolation] \label{L:hermite}
If $f\in C^{2n}[a,b]$, and let $h_{2n-1}(x)$ be the Hermite interpolation at
points $x_1,x_2,\cdots,x_n$, i.e.,
\[
  h_{2n-1}(x_i)=f(x_i), \qquad h'_{2n-1}(x_i)=f'(x_i), \qquad i=1,2,\cdots,n,
\]
then given $x\in[a,b]$, there exist a $\xi=\xi(x)$ such that
\begin{equation}
  f(x) - h_{2n-1}(x) = \frac{f^{(2n)}(\xi)}{(2n)!} w_n^2(x),
\end{equation}
where again
\[
  w_n(x) = (x-x_1)(x-x_2) \cdots (x-x_n).
\]
\end{lemma}
\begin{proof}
We only need to prove this for $x\neq x_i, i=1,2,\cdots,n$. Let
\[
  \psi(t) = f(t) - h_{2n-1}(t) - \frac{ f(x)-h_{2n-1}(x) }{ w_n^2(x) } w_n^2(t),
\]
it is easy to see that $\psi(x_i)=0,i=1,2,\cdots,n$, and $\psi(x)=0$.
Hence by Rolle's Theorem, $\psi'(t)$ vanishes at $n$ points which lie strictly
between each pair of consecutive points from the set $\{x_1,x_2,\cdots,x_n,x\}$.
Also that $\psi'(x_i)=0,i=1,2,\cdots,n$, hence $\psi(t)$ vanishes at a total of
$2n$ distinct points in $[a,b]$. Apply Rolle's Theorem repeatly, 
$\psi''(t)$ vanishes at $2n-1$ distinct points, $\psi'''(t)$ vanishes at $2n-2$
distinct points, etc., and $\psi^{(2n)}$ vanishes at one point in $[a,b]$, i.e.
there exist a $\xi\in[a,b]$ such that
\[
  0=\psi^{(2n)}(\xi) = f^{(2n)}(\xi) - \frac{f(x)-h_{2n-1}(x)}{w_n^2(x)} (2n)!.
\]
\end{proof}

%%%%%%%%%%%%%%%%%%%
\begin{lemma}[Error of Gaussian quadrature]
\footnote{Berezin and Zhidkov, Computing Methods, vol.1, pp.240-241}
The residual term of Gaussian quadrature is
\begin{equation}
  R(f) = \frac{(n!)^4 (b-a)^{2n+1}}{ [(2n)!]^3 (2n+1)} f^{(2n)}(\xi),
\end{equation}
where $\xi$ is some number in $(a,b)$.
\end{lemma}
\begin{proof}
Let $h_{2n-1}(x)$ be the Hermite interpolation of $f(x)$ on the abscissae
$x_1,x_2,\cdots,x_n$ as defined in Eq. \ref{E:abs}. Since $h_{2n-1}(x)$ is a
polynomial of order $2n-1$, we have
\[
  \int_a^b h_{2n-1}(x) dx = \sum_k C_k h_{2n-1}(x_k) = \sum_k C_k f(x_k).
\]
Thus
\[
  R(f) = \int_a^b f(x) dx - \sum_k C_k f(x_k) = \int_a^b (f(x)-h_{2n-1}(x)) dx,
\]
using Lemma \ref{L:hermite}, we get
\begin{align*}
  R(f) &= \int_a^b \frac{f^{(2n)}(\xi(x))}{(2n)!} w_n^2(x) dx \notag \\
       &= \frac{f^{(2n)}(\xi)}{(2n)!} \int_a^b w_n^2(x) dx 
          \qquad \text{(using mean-value theorem)}. \notag \\
\end{align*}
for some $\xi\in (a,b)$.
Apply integral by parts repeatly, we get
\begin{align*}
  \int_a^b w_n^2(x) dx 
   &= \left[
       w_n(x)\phi_1(x) - w'_n(x)\phi_2(x) + \cdots 
       + (-1)^n w^{(n)}(x) \phi_{n+1}(x)
     \right] \Big|_a^b  \notag \\
   &= (-1)^n w^{(n)}(x) \phi_{n+1}(x) |_a^b = (-1)^n n!\phi_{n+1}(b). \notag \\
\end{align*}
Again we apply integral by parts repeatly, 
\begin{align*}
  \phi_{n+1}(b) 
    &= \int_a^b \phi_n(x) dx = \int_a^b \frac{n!}{(2n)!} (x-a)^n (x-b)^n dx
       \notag \\
    &= \frac{n!}{(2n)!} 
       \left[
         \frac{(x-a)^n (x-b)^{n+1}}{n+1}\Big|_a^b
         - \int_a^b \frac{n}{n+1} (x-a)^{n-1} (x-b)^{n+1} dx
       \right]  \notag \\
    &= \text{etc. etc.} \notag \\
    &= (-1)^n \frac{(n!)^3}{[(2n)!]^2} \int_a^b (x-b)^{2n} dx \notag \\
    &= (-1)^n \frac{(n!)^3}{[(2n)!]^2 (2n+1)} (b-a)^{2n+1} \notag \\
\end{align*}
Thus we have
\begin{align*}
  R(f) &= \frac{f^{(2n)}(\xi)}{(2n)!} \int_a^b w_n^2(x) dx 
          = \frac{ f^{(2n)}(\xi) (-1)^n n! }{(2n)!} \phi_{n+1}(b) \notag \\
       &= \frac{(n!)^4 (b-a)^{2n+1}}{ [(2n)!]^3 (2n+1)} f^{(2n)}(\xi). \notag \\
\end{align*}
\end{proof}





