\chapter{Bonds}

%%%%%%%%%%%%%%%%%%%%%%%%%%%%%%%%%%%%%%%%%%%%%%%%%%%%%%%%%%%%%%%%%%%%%%%%%%%%%%
\section{Day count conventions}
Day count convention is used to calculate the number of days and the year 
fraction between two dates $T_1$ and $T_2$. It is crucial for calculating
interest payments of bonds and other interest rate products.

%%%%%%%%%%%%%%%%%%%%%%%%%%%%%
\subsection{Actual/365}

\begin{align}
  Days(T_1,T_2) &= T_2 - T_1, \notag \\
  YF(T_1,T_2) &= \frac{Days(T_1,T_2)}{365}.
\end{align}

%%%%%%%%%%%%%%%%%%%%%%%%%%%%%
\subsection{Actual/365 NoLeap}
Under this day count convention, any leap day (February 29th) between dates 
$T_1$ and $T_2$ will be ignored.

\begin{align}
  %Days(T_1,T_2) &= 
  %  \begin{cases}
  %    T_2 - T_1 - 1, & \text{if leap day (Feb 29) falls in} [T_1,T_2] \\
  %    T_2 - T_1, & otherwise,
  %  \end{cases} \notag \\
  Days(T_1,T_2) &= T_2 - T_1 - NumberOfLeapDays(T_1,T_2), \notag \\
  YF(T_1,T_2) &= \frac{Days(T_1,T_2)}{365}.
\end{align}

%%%%%%%%%%%%%%%%%%%%%%%%%%%%%
\subsection{Actual/360}

\begin{align}
  Days(T_1,T_2) &= T_2 - T_1, \notag \\
  YF(T_1,T_2) &= \frac{Days(T_1,T_2)}{360}.
\end{align}

%%%%%%%%%%%%%%%%%%%%%%%%%%%%%
\subsection{Actual/366}

\begin{align}
  Days(T_1,T_2) &= T_2 - T_1, \notag \\
  YF(T_1,T_2) &= \frac{Days(T_1,T_2)}{366}.
\end{align}


%%%%%%%%%%%%%%%%%%%%%%%%%%%%%
\subsection{Actual/Actual ICMA}
Under the Actual/Actual ICMA convention, the year fraction between dates $T_1$
and $T_2$ given two reference dates $T_3$ and $T_4$, is given by:

\begin{align}
  Days(T_1,T_2) &= T_2 - T_1, \notag \\
  YF(T_1,T_2) &= \frac{Days(T_1,T_2)}{f \times Days(T_3, T_4)},
\end{align}
here $f$ is the payment frequency. If $f$ is not specified, it can be estimated
by
\begin{equation}
  f = \frac{12.0}{Integer(0.5 + 12 * (T_2 - T_1) / 365.0)}.
\end{equation}

If reference dates $T_3$ and $T_4$ are not given, we can usually set $T_3=T_1$
and $T_4=T_2$, and the year fraction formula becomes
\begin{equation}
  YF(T_1,T_2) = \frac{Days(T_1,T_2)}{f \times Days(T_1, T_2)}
              = \frac{1}{f}.
\end{equation}


%%%%%%%%%%%%%%%%%%%%%%%%%%%%%
\subsection{Actual/Actual ISDA}

\begin{align}
  Days(T_1,T_2) &= T_2 - T_1, \notag \\
  YF(T_1,T_2) &= \frac{\text{Days not in leap year}}{365}
                 +\frac{\text{Days in leap year}}{366}.
\end{align}


%%%%%%%%%%%%%%%%%%%%%%%%%%%%%
\subsection{30/360}

Under the 30/360 convention, each month is treated to have 30 days, and each
year 360 days. To calculate the number of days between two dates $T_1$
(represented by day $D_1$, month $M_1$, year $Y_1$) and $T_2$
(represented by day $D_2$, month $M_2$, year $Y_2$), the following steps must be
performed:

\begin{enumerate}
  \item If $D_1=31$, set $D_1=30$;
  \item If $D_1=30$ and $D_2=31$, set $D_2=30$.
  \item $Days(T_1,T_2) = (D_2 - D_1) + 30 * (M_2 - M_1) + 360 * (Y_2 - Y_1)$.
\end{enumerate}

And the year fraction is given by
\begin{equation}
  YF(T_1,T_2) = \frac{Days(T_1,T_2)}{360}.
\end{equation}


%%%%%%%%%%%%%%%%%%%%%%%%%%%%%%%%%%%%%%%%%%%%%%%%%%%%%%%%%%%%%%%%%%%%%%%%%%%%%%
\section{Bond calculation}

A bond is a type of security under whith the issuer owes the holder a debt, and
is obliged to repay the principal of the bond at the maturity date, as well as a
series of interest over a specified series of dates.

%A coupon bond is a contract that pays at dates 
%$\{t_1,t_2,\cdots,t_{n}\}$ the amounts of $\{cf_1,cf_2,\cdots,cf_n\}$. 


%%%%%%%%%%%%%%%%%%%%%%%%%%%%%%%%%%%%%%%
\begin{figure}
  \begin{tikzpicture}[scale=0.9]
    \draw (-3.5,0) -- (2.8,0);
    \drawbreakright{2.8}{0}
    \drawbreakleft{4.2}{0}
    \draw (4.2,0) -- (9.3,0);
    \draw (-2,0) node[below] {$t_0$} -- (-2,0.2);

    \draw [->, >=triangle 90] 
      (0,0) node[below] {$t_1$} -- (0,2) node[above] {$cf_1$} ;
    \draw [->, >=triangle 90] 
      (2,0) node[below] {$t_2$} -- (2,2) node[above] {$cf_2$} ;
    \draw [->, >=triangle 90] 
      (5,0) node[below] {$t_{n-1}$} -- (5,2) node[above] {$cf_{n-1}$} ;
    \draw [->, >=triangle 90] 
      (7,0) node[below] {$t_{n}$} -- (7,3) node[above] {$cf_{n}$} ;
  \end{tikzpicture} 
  \caption{A prototypical coupon bond pays at dates
           $\{t_1,t_2,\cdots,t_{n}\}$ the fixed amounts of 
           $\{cf_1,cf_2,\cdots,cf_n\}$.}
\end{figure}
%%%%%%%%%%%%%%%%%%%%%%%%%%%%%%%%%%%%%%%

%%%%%%%%%%%%%%%%%%%%%%%%%%%%%
\subsection{Notations}

%%%%%%%%%%%%%%%%%%%%%%%%%%%%%%%%%%%%%%%%%%%
\begin{table*}
%\caption{Notations}
\begin{tabular}{r | l}
\hline\hline
$t_i$ & accrual dates \\
$[t_{i-1},t_{i}]$ & i-th accrual period \\
$N_i$  & starting notional for the i-th accrual period \\
$c_i$  & coupon for the i-th accrual period \\
$cf_i$ & cashflow for the i-th accrual period, paid at $t_i$ \\
$cf_i^{interest}$ & interest cashflow for the i-th accrual period \\
$f$    & payment frequency \\
$Days(t,s)$ & number of days between dates t and s (given day count convention)\\
$YF(t,s)$   & year fraction between dates t and s (given day count convention)\\
$AI$   & accrued interest \\
$YTM$  & yield to maturity \\
\hline
\end{tabular}
\end{table*}
%%%%%%%%%%%%%%%%%%%%%%%%%%%%%%%%%%%%%%%%%%%

%%%%%%%%%%%%%%%%%%%%%%%%%%%%%
\subsection{Cashflows}

The cashflows of a coupon-bearing bond can be divided into interest cashflows
and principal cashflows. The interest cashflow of accrued period $[t_{i-1},t_i]$ 
is 
\begin{equation}
 cf_i^{interest} = N_i\cdot c_i \cdot YF(t_{i-1},t_{i}).
\end{equation}
The notional is usually paid in full at 
expiry, though they are some bonds with principal payment before the expiry.
%Typically the cashflows are defined as $c_i=N\delta_i K$ for $i<n$, and 
%$c_n=N\delta_n K+N$ where $K$ is a fixed coupon rate, $N$ is the face value,
%and $\delta_j$ be the year fraction between dates $T_{j-1}$ and $T_{j}$,

For discount bonds, there is no interest cashflows, but one principal payment of
the full notional at expiry.

%%%%%%%%%%%%%%%%%%%%%%%%%%%%%
\subsection{Dirty price and clean price}

Bonds are often quoted in two different prices: dirty price and clean price,
the relationship between these two prices is
\begin{equation}
  P_{dirty} = P_{clean} + AI,
\end{equation}
where $AI$ is the accrued interest at the calculation date.

%%%%%%%%%%%%%%%%%%%%%%%%%%%%%
\subsection{Accrued interest}
Suppose that the calculation date $t$ falls in the accrual period
$[t_{i-1},t_{i}]$, the accrued interest can be calculated by:
\begin{equation}
  %AI = \frac{N_i \cdot c_i}{f} \frac{ Days(t_i,t) }{Days(t_i,t_{i+1})}, 
  AI = cf_i^{interest} \frac{ Days(t_{i-1},t) }{Days(t_{i-1},t_{i})}, 
\end{equation}


%%%%%%%%%%%%%%%%%%%%%%%%%%%%%
\subsection{Present value}
Bond price is calculated by discounting all future cashflows:
\begin{equation}
  PV_t = \sum_{t_i>t} cf_i \times DF_i,
\end{equation}
where the discount factor at time $t_i$ is given by:
\[
  DF_i = \exp(-r_t(t_i)\times (t_i-t)).
\]


%%%%%%%%%%%%%%%%%%%%%%%%%%%%%
\subsection{Yield to maturity}

In practice, bonds are often quoted in YTM (yield to maturity) rather than in
price. Basically, when cashflows are discounted using YTM, the present value 
equals to the market dirty price.

If the calculation date falls in the last accrual period $[t_{n-1},t_n]$, and
the remaining term is shorter than 1 year, the present value can be calculated 
from $YTM$ by simple compounding:
\begin{equation}
  PV_t = \frac{cf_n}{ 1+ YTM \times YF(t,t_i) },
\end{equation}

otherwise we have
\begin{equation}
  PV_t = \sum_{t_i>t} \frac{cf_i}{ (1+YTM/f)^{f\times YF(t,t_i)} }
\end{equation}

YTM may be solved from dirty price using a root-finding algorithm such as
Brent's method. 

%%%%%%%%%%%%%%%%%%%%%%%%%%%%%
\subsection{Yield to option}

Certain bonds may have call/put options embedded. For these bonds, we can 
also calculate the yield to option, defined as the yield if the cashflow stop at
the option exercise date. The calculation method is similar to that of
calculating YTM.

%%%%%%%%%%%%%%%%%%%%%%%%%%%%%
\subsection{Macaulay Duration}
Macaulay duration is the weighted average duration of (remaining) cash flows,
wherein the remaining time of each cash flow is weighted by the present value of
the corresponding cash flow:

\begin{equation}
  Duration_{mac} = \frac{\sum_{t_i>t} YF(t,t_i)\times PV_i}
                        {\sum_{t_i>t} PV_i},
\end{equation}
where $PV_i$ is the present value of the $i$-th cash flow.

%%%%%%%%%%%%%%%%%%%%%%%%%%%%%
\subsection{Modified Duration}
In contrast to Macaulay duration, modified duration is a price senstivity
measure, defined as the percentage derivative of price with respect to yield:

\begin{equation}
  Duration_{mod} = -\frac{1}{PV} \frac{\Delta (PV)}{\Delta Y}
    = \frac{PV(Y-\Delta Y) - PV(Y+\Delta Y)}
                        {(PV(Y-\Delta Y) + PV(Y+\Delta Y)) \times \Delta Y}.
\end{equation}

%%%%%%%%%%%%%%%%%%%%%%%%%%%%%
\subsection{Convexity}
Bond convexity is a measure of the nonlinear relationship of bond prices to
changes in yield, being the second derivative of the bond price with respect to
yield:

\begin{equation}
  Convexity  
    = \frac{PV(Y-\Delta Y) + PV(Y+\Delta Y) - 2PV}{PV \times \Delta Y^2}.
\end{equation}


%%%%%%%%%%%%%%%%%%%%%%%%%%%%%%%%%%%%%%%%%%%%%%%%%%%%%%%%%%%
\section{Interpolation methods}

%%%%%%%%%%%%%%%%%%%%%%%%%%%%%%%%
\subsection{Piecewise linear}

TODO



%%%%%%%%%%%%%%%%%%%%%%%%%%%%%
\subsection{Interpolating cubic splines}

\paragraph{Input:} a set of ordered $x, y$ coordinates
%\footnote{In this
%    document we start with index 1 which differs from C-convention
%    where indices start with 0. E.g.\ in this document $x_1$ equals to
%\code{x[0]} in code, $a_1=$ \code{a[0]}, $c_n=$ \code{c[n-1]}, etc.}
\begin{equation*}
\set{(x_1,y_1),\dots,(x_n,y_n)},\quad \text{with } x_1 < x_2 < \dots < x_n.
\end{equation*}
\paragraph{Output:} a piecewise cubic function
\begin{equation}
\begin{aligned}
    f(x) & = \begin{cases}
            f_0(x), & x<x_1,\\
            f_1(x), & x\in[x_1,x_2),\\
            \dots  \\
            f_{n-1}(x), & x\in[x_{n-1},x_n), \\
            f_{n}(x), & x\geq x_n. \\
            \end{cases} \\
            f_0(x)  & = a_1 + b_1 (x-x_1) + c_0 (x-x_1)^2, & x<x_1,\\
            f_i(x)  & = a_i + b_i (x-x_i) + c_i (x-x_i)^2 + d_i (x-x_i)^3, &
            x\in[x_i, x_{i+1}),\\
                f_n(x)  & = a_n + b_n (x-x_n) + c_n (x-x_n)^2, & x\geq x_n,
\end{aligned}
\end{equation}

\setlength{\unitlength}{1.0cm}
\begin{picture}(15,2)(0,-2)

 \put(0,-1.0){\line(1,0){6}}
 \put(8,-1.0){\line(1,0){5}}

 \put(1,-0.7){\makebox(0,0){$f_0$}}
 \put(2,-0.9){\line(0,-1){0.2}}
 \put(2,-1.4){\makebox(0,0){$x_1$}}

 \put(2.5,-0.7){\makebox(0,0){$f_1$}}
 \put(3,-0.9){\line(0,-1){0.2}}
 \put(3,-1.4){\makebox(0,0){$x_2$}}

 \put(3,-1.1){\vector(1,0){2}}
 \put(5,-1.1){\vector(-1,0){2}}
 \put(4,-1.4){\makebox(0,0){$h_2$}}

 \put(4,-0.7){\makebox(0,0){$f_2$}}
 \put(5,-0.9){\line(0,-1){0.2}}
 \put(5,-1.4){\makebox(0,0){$x_3$}}

 \put(9.5,-0.9){\line(0,-1){0.2}}
 \put(9.5,-1.4){\makebox(0,0){$x_{n-1}$}}

 \put(10.25,-0.7){\makebox(0,0){$f_{n-1}$}}
 \put(11,-0.9){\line(0,-1){0.2}}
 \put(11,-1.4){\makebox(0,0){$x_n$}}
 \put(12,-0.7){\makebox(0,0){$f_n$}}


\end{picture}

which satisfies
\begin{itemize}
   \item $f(x_i) = y_i$, i.e.\ $f$ interpolates all points (implies $a_i=y_i$),
   \item $f\in\Co^1$, i.e.\ is continuously differentiable.
\end{itemize}
More conditions will be imposed to specify $f$ uniquely but this
depends on the type of spline as described below.
Note, the derivatives of $f_i$ are given by
\begin{equation*}
\begin{split}
  f'_i(x)    & = 3 d_i (x-x_i)^2 + 2 c_i (x-x_i) + b_i,\\
  f''_i(x)   & = 6 d_i (x-x_i)+  2 c_i.\\
\end{split}
\end{equation*}
We also define $h_i:=x_{i+1}-x_i$, $i\in\set{1,\dots, n-1}$.


%%%%%%%%%%%%%%%%%%%%%%%%%%%%%%%%%%%%
\subsection{Cubic $\Co^2$ splines}
They are defined by
\begin{itemize}
   \item $f(x_i) = y_i$, i.e.\ $f$ interpolates all points,
   \item $f\in\Co^2$, i.e.\ $f$ is twice continuously differentiable.
\end{itemize}
In detail this means:
\begin{equation*}
\begin{aligned}
   \text{1) interpolates points: } f_i(x_i) & = y_i: &
    a_i & = y_i,\\
   \text{2) continuous } f\in\Co^0 \text{: } f_i(x_{i+1}) & =y_{i+1}: &
	d_i h_i^3 + c_i h_i^2 + b_i h_i & = y_{i+1}-y_i,\\
  \text{3) differentiable }f\in\Co^1 \text{: } f'_{i-1}(x_i) & =
    f'_i(x_i): &
    3 d_{i-1} h_{i-1}^2 + 2 c_{i-1} h_{i-1} + b_{i-1} & = b_i,\\
   \text{4) twice differentiable }f\in\Co^2 \text{: } f''_i(x_{i+1}) & =
    f''_{i+1}(x_{i+1}): &
    6 d_i h_i + 2 c_i & = 2 c_{i+1},\\
\end{aligned}
\end{equation*}
Solving for $d_i$ in the 4th condition ($f''$ continuous) gives
\begin{equation*}
    f\in\Co^2:\, d_i  =\frac{c_{i+1}-c_i}{3 h_i}.
\end{equation*}
Inserting this into the 2nd condition ($f$ continuous) and
solving for $b_i$ gives
\begin{equation*}
    f\in\Co^0:\, b_i  = \frac{y_{i+1}-y_i}{h_i} - \frac{1}{3} (2 c_i + c_{i+1}) h_i.
\end{equation*}
Finally, inserting these two equalities into the 3rd condition
($f'$ continuous) gives a linear equation system for $c$:
\begin{equation}
\label{eq:cubic_c2_spline_coeffs}
\boxed{
\begin{aligned}
 \frac{1}{3} h_{i-1} c_{i-1} + \frac{2}{3} (h_{i-1}+h_i) c_i
 + \frac{1}{3} h_i c_{i+1}   & =
 \frac{y_{i+1}-y_i}{h_i} -  \frac{y_i-y_{i-1}}{h_{i-1}},
 & i\in\set{2,\dots,n-1},\\
 d_i & =\frac{c_{i+1}-c_i}{3 h_i}, & i\in\set{1,\dots,n-1},\\
  b_i & = \frac{y_{i+1}-y_i}{h_i} - \frac{1}{3} (2 c_i + c_{i+1}) h_i,
 & i\in\set{1,\dots,n-1}.
\end{aligned}
}
\end{equation}

%\subsubsection*{Boundary conditions}
Second order conditions give immediate equations for $c_1$ and $c_n$:
\begin{equation*}
\begin{aligned}
    f''(x_1)&=\gamma_1: & c_0=c_1 & = \frac{\gamma_1}{2},\\
    f''(x_n)&=\gamma_n: & c_n & = \frac{\gamma_n}{2}.\\
\end{aligned}
\end{equation*}
First order conditions give immediate relationships for $b_1$ and $b_n$
which need to be translated to conditions for $c$:
\begin{equation*}
\begin{aligned}
    f'(x_1)&=\delta_1: & b_1&=\delta_1 \follows&
        \frac{2}{3}h_1 c_1 +\frac{1}{3}h_1 c_2
        & = \frac{y_2-y_1}{h_1} -\delta_1,\\
    f'(x_n)&=\delta_n: & b_n&=\delta_n \follows&
        \frac{1}{3}h_{n-1} c_{n-1} +\frac{2}{3}h_{n-1} c_n
        & = \delta_n - \frac{y_n-y_{n-1}}{h_{n-1}}.\\
\end{aligned}
\end{equation*}
The equation for the left boundary condition follows directly by inserting
$b_1=\delta_1$ into the third equation of \eqref{eq:cubic_c2_spline_coeffs}.
The equation for the right boundary condition follows from the
differentiability condition $f'_{n-1}(x_n)  = f'_n(x_n)$, replacing
$b_n$ by $\delta_n$ and replacing $b_{n-1}$ and $d_{n-1}$ by the relations
in Equation~\eqref{eq:cubic_c2_spline_coeffs}.

The not-a-knot condition, i.e.\ the continuity of the third derivatives
at the two outer segments, gives $d_1=d_2$ and $d_{n-2}=d_{n-1}$:
\begin{equation*}
\begin{aligned}
    f_1'''(x_2)&=f_2'''(x_2): & d_1&=d_2 \follows&
    -h_2 c_1 + (h_1+h_2) c_2 - h_1 c_3 & = 0,\\
    f_{n-2}'''(x_{n-1})&=f_{n-1}'''(x_{n-1}): & d_{n-2}&=d_{n-1}\follows&
    -h_{n-1} c_{n-2} + (h_{n-2}+h_{n-1}) c_{n-1} - h_{n-2} c_n & = 0.
\end{aligned}
\end{equation*}



In all cases, $d_n$ and $b_n$ are then determined by:
\begin{equation*}
\begin{aligned}
    \text{quadratic extrapolation:} & & d_n & = 0,\\
    f'_{n-1}(x_n)  = f'_n(x_n): & & b_n & = 3 d_{n-1} h_{n-1}^2 + 2 c_{n-1} h_{n-1} + b_{n-1}.
\end{aligned}
\end{equation*}


%%%%%%%%%%%%%%%%%%%%%%%%%%%%%%%%%%%%%%%%%%%%%%%%%%%%%%%%%%%%%%%%%%%%%%%
\subsection{Hermite $\Co^1$ splines}
They are defined by
\begin{itemize}
   \item $f(x_i) = y_i$, i.e.\ $f$ interpolates all points,
   \item $f\in\Co^1$, i.e.\ $f$ is continuously differentiable,
   \item $f'(x_i) = \delta_i$, i.e.\ derivatives are specified at each inner
       point, e.g.\ by three point finite differences
       \begin{equation*}
       \delta_i=-\frac{h_i}{h_{i-1}(h_{i-1} + h_i)} y_{i-1}
           + \frac{h_i-h_{i-1}}{h_{i-1} h_i} y_i
           + \frac{h_{i-1}}{h_i(h_{i-1} + h_i)} y_{i+1}.
       \end{equation*}
\end{itemize}
In detail this means:
\begin{equation*}
\begin{aligned}
    \text{1) interpolates points: } f_i(x_i) & = y_i: &
    a_i & = y_i,\\
    \text{2) prescribed derivative: } f'_i(x_i) & = \delta_i: &
    b_i & = \delta_i,\\
    \text{3) continuous } f\in\Co^0 \text{: } f_i(x_{i+1}) & =y_{i+1}: &
	d_i h_i^3 + c_i h_i^2 + b_i h_i & = y_{i+1}-y_i,\\
    \text{4) differentiable }f\in\Co^1 \text{: } f'_i(x_{i+1}) & =\delta_{i+1}: &
    3 d_i h_i^2 + 2 c_i h_i  & = \delta_{i+1}-\delta_i,\\
\end{aligned}
\end{equation*}
which can be solved to obtain
\begin{equation}
\label{eq:hermite_spline_coeffs}
\boxed{
\begin{aligned}
  \text{1) } a_i & =  y_i, \\
  \text{2) } b_i & =  \delta_i, \\
  \text{3,4) } c_i & = -\frac{2 b_i+b_{i+1}}{h_i}+3\frac{a_{i+1}-a_i}{h_i^2}
    & = -\frac{2\delta_i+\delta_{i+1}}{h_i} + 3 \frac{y_{i+1}-y_i}{h_i^2},
        \quad i\in\set{1,\dots,n-1},\\
  \text{4) } d_i & = -\frac{2 c_i}{3 h_i}+\frac{b_{i+1}-b_i}{3 h_i^2}
    & = \frac{\delta_i+\delta_{i+1}}{h_i^2}-2\frac{y_{i+1}-y_i}{h_i^3},
        \quad i\in\set{1,\dots,n-1}.
\end{aligned}
}
\end{equation}

%\subsubsection*{Boundary conditions}
First order conditions are trivial:
\begin{equation*}
\begin{aligned}
    f'(x_1)&=\delta_1: & b_1 & = \delta_1,\, c_0 = 0,\\
    f'(x_n)&=\delta_n: & b_n & = \delta_n,\, c_n = 0.
\end{aligned}
\end{equation*}

Second order conditions imply a value for $c$ but we can re-express
this in terms of $b$:
\begin{equation*}
\begin{aligned}
    f''(x_1)&=\gamma_1: & c_0 = c_1 &= \frac{1}{2}\gamma_1 \equivalent
        b_1 = \frac{1}{2}\left(-b_2 -\frac{\gamma}{2} h_1
            + 3 \frac{y_2-y_1}{h_1}\right),\\
    f_n''(x_n)&=\gamma_n: & c_n &= \frac{1}{2}\gamma_n,\\
            f_{n-1}''(x_n)&=\gamma_n: & 6 d_{n-1}h_{n-1} + 2 c_{n-1} &= \frac{1}{2}\gamma_n \equivalent
            b_n = \frac{1}{2}\left(-b_{n-1} +\frac{\gamma}{2} h_{n-1}
            + 3 \frac{y_n-y_{n-1}}{h_{n-1}}\right).\\
\end{aligned}
\end{equation*}

The not-a-knot contition gives $d_1=d_2$ and $d_{n-2}=d_{n-1}$ and
re-expressing in terms of $b$ yields:
\begin{equation*}
\begin{aligned}
    f_1'''(x_2)&=f_2'''(x_2): & d_1&=d_2 \equivalent &
    b_1 & = -b_2+2\frac{y_2-y_1}{h_1}+\frac{h_1^2}{h_2^2}
        \left(b_2+b_3-2\frac{y_3-y_2}{h_2}\right),\\
    f_{n-2}'''(x_{n-1})&=f_{n-1}'''(x_{n-1}): & d_{n-2}&=d_{n-1}\equivalent&
    b_n & = -b_{n-1}+2\frac{y_n-y_{n-1}}{h_{n-1}}\\
        & & & & & +\frac{h_{n-1}^2}{h_{n-2}^2}
        \left(b_{n-2}+b_{n-1}-2\frac{y_{n-1}-y_{n-2}}{h_{n-2}}\right).
\end{aligned}
\end{equation*}
To determine a value for $c_n$ we additionally impose second order
continuity at the boundary:
\begin{equation*}
    f_{n-1}''(x_n)=f_n''(x_n): \, c_n =3d_{n-1}h_{n-1}+c_{n-1} \equivalent\,
        c_n=\frac{b_{n-1}+2 b_n}{h_{n-1}}-3\frac{y_n-y_{n-1}}{h_{n-1}^2}.
\end{equation*}


%%%%%%%%%%%%%%%%%%%%%%%%%%%%%%%%%%%%%%%%%%%%%
\section{Yield curve construction: parametric methods}

The estimation of a zero-coupon yield curve is based on an assumed functional
relationship between either par yields, spot rates, forward rates or discount 
factors on the one hand and maturities on theother. Discount factors are the 
quantities used at a given point in time to obtain the present value of
future cash flows. A discount function $\{D_{t,T}\}_T$ is the collection of 
discount factors at time $t$ for all maturities $T$. Spot rates $r_{t,T}$, the 
yields earned on bonds which pay no coupon, are related to discount factors
according to:

\begin{equation}
  D_{t,T} = \exp(-r_{t,T} T).
\end{equation}

Because spot interest rates depend on the time horizon, it is natural to define
the forward rates $f_T$m as the instantaneous rates which, when compounded 
continuously up to the time to maturity, yield the spot rates (instantaneous 
forward rates are, thus, rates for which the difference between settlement
time and maturity time approaches zero):

\begin{equation}
  r_{t,T} = -\frac{1}{T} \int_0^T f(u) du,
\end{equation}
or equivalently
\begin{equation}
  D_{t,T} = \exp \left[ -\int_0^T f(u) du \right].
\end{equation}

However, the general absence of available pure discount bonds that can be used
to compute zerocoupon interest rates presents a problem to practitioners. In
other words, zero coupon rates are rarely directly observable in financial 
markets. Attempting to extract zero-coupon rates from the prices of
those risk-free coupon-bearing instruments which are observable, namely
government bonds, various models and numerical techniques have been developed. 
Such models can broadly be categorised into parametric and spline-based 
approaches, where a different trade-off between the flexibility to
represent shapes generally associated with the yield curve (goodness-of-fit) 
and the smoothness characterizes the different approaches. 
In this section, we discuss only the parametric methods.

The underlying principle of parametric models, also referred to as
function-based models, is the specification of a single-piece function that 
is defined over the entire maturity domain. While the various approaches 
in this class of models advocate different choices of this function, they 
all share the general approach that the model parameters are determined 
through the minimization of the squared deviations of theoretical prices 
from observed prices.

%%%%%%%%%%%%%%%%%%%%%%%%%%%%%%%%%%%%%%%%%%%%%%%%%%
\subsection{Vasicek-Fong exponential spline model}
Vasicek and Fong (1982) use exponential splines that produce a smooth and
continuous forward rate curve. Given a set of knot points 
$t_0, t_1,t_2,\cdots,t_n$, the discount factor at maturity $t$ is assumed to be
of the form:

\begin{equation}
  D_t = a_i + b_i e^{-\mu t} + c_i e^{-2\mu t} + d_i e^{-3\mu t}, 
        \quad t\in[t_i,t_{i+1}],
\end{equation}
where $a_i,b_i,c_i,d_i,\mu$ are parameters.


%%%%%%%%%%%%%%%%%%%%%%%%%%%%%%%%%%%%%%%%%%%%%%%%%%
\subsection{Nelson and Siegel method}
Nelson and Siegel (1987) suggested the following parametric form for the forward
rate curve:
\begin{equation}
  f(t) = \beta_0 + \beta_1 e^{-t/\lambda} 
         + \frac{\beta_2 t}{\lambda} e^{-t/\lambda},
\end{equation}
where $\beta_0,\beta_1,\beta_2,\lambda>0$ are parameters. The spot rate function
under this model is then:
\begin{equation}
  r(t) = \beta_0 + \frac{\beta_1 \lambda}{t} \left(1- e^{-\lambda t} \right)
         - \frac{\beta_2 \lambda}{t} \left(1- e^{-\lambda t} \right)
         - \beta_2 e^{-\lambda t}.
\end{equation}

%%%%%%%%%%%%%%%%%%%%%%%%%%%%%%%%%%%%%%%%%%%%%%%%%%
\subsection{Svensson method}
Svensson(1992) extended the Nelson and Siegel (1987) model by adding a second
hump term to the instantaneous forward rate function:

\begin{equation}
  f(t) = \beta_0 + \beta_1 e^{-t/\lambda_1} 
         + \frac{\beta_2 t}{\lambda_1} e^{-t/\lambda_1}
         + \frac{\beta_3 t}{\lambda_2} e^{-t/\lambda_2},
\end{equation}

where $\beta_0,\beta_1,\beta_2,\lambda_1>0, \lambda_2>0$ are parameters. The 
spot rate function under this model is then:

\begin{equation}
  r(t) = \beta_0 + \frac{\beta_1 \lambda_1}{t} \left(1- e^{-\lambda_1 t} \right)
         - \frac{\beta_2 \lambda_1}{t} \left(1- e^{-\lambda_1 t} \right)
         - \beta_2 e^{-\lambda_1 t}
         - \frac{\beta_3 \lambda_2}{t} \left(1- e^{-\lambda_2 t} \right)
         - \beta_3 e^{-\lambda_2 t}.
\end{equation}

%%%%%%%%%%%%%%%%%%%%%%%%%%%%%%%%%%%%%%%%%%%%%%%%%%
\subsection{Merrill Lynch Exponential Spline Model}

Under the Merrill Lynch expoential spline model, the discount factor curve is
assumed to be of the form:
\begin{equation}
  D_t = \sum_{k=1}^9 z_k e^{-k\alpha t },
\end{equation}
where $z_1,\dots,z_9$ and $\alpha$ are the parameters to be estimated.



%%%%%%%%%%%%%%%%%%%%%%%%%%%%%%%%%%%%%%%%%%%%%%%%%%%%%%%%%%%%%%%%%%%%%%%%
\section{Yield curve construction: bootstrapping method}
Zero yield curve can also be constructed from market quotes of swap price by
using the bootstrapping method.

%%%%%%%%%%%%%%%%%%%%%%%%%%%%%%%%%%%%%%%%%%%%%%%%%%
\subsection{Interest rate swap pricing}

An interest rate swap is an agreement between two counterparties
to exchange a series of cashflows on pre-agreed dates in the future (see Figure
\ref{F:swap}). In general the payment frequencies of the fixed and floating legs
are different. A payer's swap pays fixed rate $K$ and receives the floating rate, 
while a receiver's swap pays floating rate and receives fixed rate.

Let $\{T_0,T_1,\cdots,T_{n-1}\}$ be the reset dates, and
$\{T_1,T_2,\cdots,T_{n}\}$ be the payment dates of the floating leg, and
$\{S_0,S_1,\cdots,S_{m-1}\}$ be the reset dates, and
$\{S_1,S_2,\cdots,S_{n}\}$ be the payment dates of the fixed leg. 
Let $\delta^S_j$ be the year fraction between dates $S_{j-1}$ and $S_{j}$,
and let $\delta^T_{j}$ be the year fraction between dates $T_{j-1}$ and 
$T_{j}$. For a payer's swap, the payoff at time $T_j$ is
\[
  \delta^T_j L_{T_{j-1}}[T_{j-1},T_j]  \qquad j=1,2,\cdots,n
\]
and at the time $S_j$, the payoff is
\[
  - \delta^S_j K  \qquad j=1,2,\cdots,m.
\]

%%%%%%%%%%%%%%%%%%%%%%%%%%%%%%%%%%%%%%%
\begin{figure}
  \begin{tikzpicture}[scale=1]
    \draw (-3,0) -- (2.8,0);
    \drawbreakright{2.8}{0}
    \drawbreakleft{4.2}{0}
    \draw (4.2,0) -- (8.5,0);

    % floating payments using snake lines (with my own code for \drawsnake)
    \drawsnake{(0,0)}{(0,1.1)}{$\delta^T_{1} L_0$}
    \drawsnake{(2,0)}{(2,1.6)}{$\delta^T_{2} L_1$}
    \drawsnake{(5,0)}{(5,1.1)}{$\delta^T_{n-1} L_{n-2}$}
    \drawsnake{(7,0)}{(7,1.6)}{$\delta^T_{n} L_{n-1}$}

    \draw (-2,0) node[below right]{$S_0$} node[above right]{$T_0$} ; % text
    \draw (-2,-0.2) -- (-2,0.2);
    \draw (0,0) node[above right] {$T_1$} -- (0,0.2);
    \draw (5,0) node[above right] {$T_{n-1}$} -- (5,0.2);

    \draw [->, >=triangle 90] 
      (2,0) node[above right]{$T_2$} node[below right] {$S_1$} 
      -- (2,-1) node[below] {$\delta^S_{1}K$} ;
    \draw [->, >=triangle 90] 
      (7,0) node[above right]{$T_{n}$} node[below right] {$S_{m}$} 
      -- (7,-1) node[below] {$\delta^S_{m}K$} ;
  \end{tikzpicture} 
  \caption{A payer's swap pays fixed and receives floating rate. Here we use
           abbreviation $L_j$ for $L_j[T_j,T_{j+1}]$. In this example the
           floating leg pays twice as frequent as the fixed leg, i.e. $n=2m$.}
  \label{F:swap}
\end{figure}
%%%%%%%%%%%%%%%%%%%%%%%%%%%%%%%%%%%%%%%

The price of the fixed leg at time $t<T_0$ is
\[
  K \sum_{j=1}^m \delta^S_j D_{tS_j},
\]
while the price of the floating leg is simply
\[
  \sum_{j=1}^n \delta^T_j L_{T_{j-1}}[T_{j-1},T_j] D_{tT_j}
    = D_{tT_0} - D_{tT_n}.
\]
And the price of a payer's swap is the difference between the floating leg and
the fixed leg:
\begin{equation}
  PFS(t,\{T_j\}_{j\le n},\{S_j\}_{j\le m},K)
    = D_{tT_0} - D_{tT_n} - K\sum_{j=1}^m \delta^S_j D_{tS_j}.
\end{equation}
Similarly the price of a receiver's swap equals the fixed leg substracts the
floating leg:
\begin{equation}
  RFS(t,\{T_j\}_{j\le n},\{S_j\}_{j\le m},K)
    = -D_{tT_0} + D_{tT_n} + K\sum_{j=1}^m \delta^S_j D_{tS_j}.
\end{equation}

The par swap rate is the fixed rate of the swap such that the swap is at the 
money, i.e.
\begin{equation} \label{E:swap}
  y_n(t) \equiv y(t,\{T_j\}_{j\le n},\{S_j\}_{j\le m}) 
    = \frac{ D_{tT_0} - D_{tT_n} }{ \sum_{j=1}^m \delta^S_j D_{tS_j} },
\end{equation}

For interest rate swap with reset inside payment periods, 
i.e. inside payment period $[T_{j-1},T_j]$ there are reset dates
$t_0,t_1,t_2,\dots,t_{n_j}$, we can write down the
(compounded) floating payment at time $T_j$ as:
\begin{equation}
  \prod_{k=1}^{n_j} ( 1 + (r_k+s)\tau(t_{k-1},t_{k}) ) - 1,
\end{equation}
where $r_k=L_{t_{k-1}}(t_{k-1},t_k)$ is the forward rate for the reset
period $[t_{k-1},t_k]$, $s$ the floating spread. 
If simple compounding is used instead, the floating payment is then
\begin{equation}
  \sum_{k=1}^{n_j} (r_k+s) \tau(t_{k-1},t_{k}).
\end{equation}


%%%%%%%%%%%%%%%%%%%%%%%%%%%%%%%%%%%%%%%%%%%%%%%%%%%%%%%%%%%%%%%
\subsection{Bootstrapping zero yield curve from swap data}

The zero yield curve can be bootstrapped from a combination of deposit quotes
(for short tenors) and market interest rate swap data (for long tenors).
We begin by defining the set of swap market quotes $y_1, y_2, \dots, y_M$ 
which are for contracts with (increasing) maturity $T_1<T_2<\dots<T_M$. 
We start from the instrument with the shortest maturity, derive zero rate for
this maturity, then move on to the next instrument in line, derive the zero rate
for this new maturity, and so on.
With this method, it is guaranteed that the market price of all the instruments
used will be reproduced precisely.

And because there are usually multiple cashflows between consecutive maturities, 
we have to make certain interpolation assumptions on the zero rates (or forward
rates, or par swap rates) in order to calculate the present value of swaps. 
Let us suppose between two consecutive swap maturities $T_i$ and $T_{i+1}$ 
there are $n$ cashflows, $T_i<t_1<t_2<\dots<t_n<T_{i+1}$. 
The possible interpolation assumptions are:

\begin{itemize}
  \item Piecewise linear par swap rate

  For the time period $[T_i,T_{i+1}]$, we create pseudo-swaps with maturities 
  at each cashflow date $t=t_1,t_2,\dots,t_n$ and assume the par swap rate is 
  piecewise linear, i.e., the par swap rate for swap with maturity $t$ is given
  by:

  \[
	y(t) = y(T_i) + \frac{y(T_{i+1})-y(T_i)}{T_{i+1}-T_i} (t-T_i).
  \]
  Given the zero curve up to date $T_i$, we first calculate the zero rate at
  time $t_1$ by using the pseudo-swap at $t_1$, then we calculate the zero 
  rate at time $t_2$ by using the pseudo-swap at $t_2$, and so on.

  \item Piecewise linear zero rate

  We assume the zero rate is piecewise linear, i.e. for each 
  at each cashflow date $t=t_1,t_2,\dots,t_n$ the zero rate is

  \[
	r(t) = r(T_i) + \frac{r(T_{i+1})-r(T_i)}{T_{i+1}-T_i} (t-T_i).
  \]

  Using the par swap rate of swap at $T_{i+1}$, we can then solve the zero rate
  at $T_{i+1}$.

  \item Piecewise constant forward rate
  \footnote{This method is equivalent to piecewise linear on the logarithm of
	  the discount factor.}

  We assume that the instantaneous forward rate is piecewise constant, i.e.
  constant at $f_i$ in time period $[T_i,T_{i+1}]$, then for any time $t$ 
  inside this period, the zero rate is: 

  \[
	  r(t) t = r(T_i) T_i + f_i \cdot (t-T_i),
  \]

  we can then solve $f_i$ using the par swap rate at $T_{i+1}$.


  \item Piecewise linear forward rate

  We assume that the instantaneous forward rate is piecewise linear, i.e.
  for any time $t$ inside time period $[T_i,T_{i+1}]$, the instantaneous forward
  rate is: 

  \[
	f(t) = f(T_i) + \frac{f(T_{i+1})-f(T_i)}{T_{i+1}-T_i} (t-T_i),
  \]
  and then the zero rate is:
  \[
	r(t) t = r(T_i) T_i + f(T_i) (t-T_i)
	         + \frac{1}{2} \frac{f(T_{i+1})-f(T_i)}{T_{i+1}-T_i} (t-T_i)^2.
  \]
  we can then solve $f(T_{i+1})$ using the par swap rate at $T_{i+1}$.

  \item Piecewise quadratic forward rate

  Let $f_i^d$ be the input discrete forward rate at node $i$, the instantaneous
  forward rate at point $t_i$ is defined for $i=1,2,\dots,n-1$ by:
  \begin{equation}
    f_i = f(t_i) = \frac{t_i-t_{i-1}}{t_{i+1}-t_{i-1}} f^d_{i+1}
                   + \frac{t_{i+1}-t_{i}}{t_{i+1}-t_{i-1}} f^d_{i},
  \end{equation}
  \begin{equation}
    f_0 = f(t_0) = f_1^d - \frac{1}{2}(f_1 - f_1^d),
  \end{equation}
  \begin{equation}
    f_n = f(t_n) = f_n^d - \frac{1}{2}(f_{n-1} - f_n^d).
  \end{equation}

  For any time $t\in [t_{i-1},t_i)$, the instantaneous forward rate is assumed
  to have the form:
  \begin{equation}
    f(t) = f_i^d + g_i\left( \frac{t-t_i}{t_i-t_{i-1}} \right),
  \end{equation}
  where function $g_i$ is quadratic:
  \[
	  g_i(x) = g_i(0)(1- 4x + 3x^2) + g_i(1)(-2x+3x^2).
  \]


%%%%%%%%%%%%%%%%%%%%%%%%%%%%%%%%%%%%%%%%%%%%%%%%%%%%%%%%%%
\section{Bond Futures}

A bond future is a future contract in which the asset for delivery is a
government bond.
Any government bonds that meet the maturity specification of a
future contract are eligible for delivery.
All eligible delivery bonds construct the delivery basket where each
bond has its own conversion factor.
Conversion factors are used to equalise the coupon and accrued
interest differences of all the deliverable bonds.
The seller usually picks up the cheapest bond in the basket to
deliver, called the cheapest-to-deliver (CTD).
The CTD bond is normally delivered on the last delivery day of the
month.

%%%%%%%%%%%%%%%%%%%%%%%%%
\subsection{Notations}

%%%%%%%%%%%%%%%%%%%%%%%%%%%%%%%%%%%%%%%%%%%
\begin{table*}
%\caption{Notations}
\begin{tabular}{r | l}
\hline\hline
$T$ & delivery date of the bond futures \\
$F_0$ & delivery price of the bond futures \\
$F_t$ & bond futures price at time $t$ \\
$i^*$ & index of the CTD bond \\
$K_{i}$ & conversion factor of the $i$-th bond \\
$A_{i}$ & accrued interest of the $i$-th bond at delivery date $T$\\
$S_i(t)$ & dirty price of the $i$-th bond at time $t$ \\
$D_{t t'}$ & discount factor between dates $t$ and $t'$ \\
$C_{i,j}$ & cashflow of the $i$-th bond at time $t_j$ \\
$I_i(t)=\sum_{t_j\in(t,T]} C_{i^*,j} D_{t,t_j}$ & PV at time $t$ of the 
		cashflows of the$i$-th bond in $(t,T]$ \\
\hline
\end{tabular}
\end{table*}
%%%%%%%%%%%%%%%%%%%%%%%%%%%%%%%%%%%%%%%%%%%

%%%%%%%%%%%%%%%%%%%%%%%%%
\subsection{Present value given CTD bond}

The present value of a bond futures at time $t$ given CTD bond $i^*$ is:
\begin{equation}
  PV_t = (F_t - F_0) D_{tT},
\end{equation}
where
\begin{equation}
  F_t K_{i^*} + A_{i^*} = \frac{S_{i^*}(t) - I_{i^*}(t) }{ D_{tT} },
\end{equation}

The CTD bond might change during the lifetime of the bond futures. One way to
determine the CTD bond is by calculating the implied repo rate of all bonds:
\begin{equation}
  IRR_i = \frac{ F_t K_i + A_i + \sum_{t_j\in(t,T]} C_{i,j} - S_i(t) }
               { S_i(t) \tau(t,T) - \sum_{t_j\in(t,T]} C_{i,j} \tau(t_j,T) }
\end{equation}
The bond with the maximal IRR is the CTD bond.



\end{itemize}
