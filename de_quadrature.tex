%%%%%%%%%%%%%%%%%%%%%%%%%%%%%%%%%%%%%%%%%%%%%%
\chapter{Double Exponential Quadrature}

%%%%%%%%%%%%%%%%%%%%%%%%%%%%%%%%%%%%%%%%%%%%%%
\section{Starting Point}

Starting from the celebrated Euler-Maclaurin formula, 
\footnote{J. Stoer and R. Bulirsch, Introduction to Numerical Analysis, 2nd ed.,
Springer (1993), section 3.3, eq. (3.3.4).}
\begin{align}
  \int_a^b f(x) dx 
    &= h\sum_{j=0}^n f(x_j) - \frac{h}{2}(f(a)+f(b)) \notag \\
    &- \sum_{i=1}^m \frac{h^{2i} B_{2i}}{ (2i)! } 
       \left( f^{(2i-1)}(b) - f^{(2i-1)}(a)  \right)
       - E_m, \notag \\
\end{align}
where $m>0,n>0$, $h=(b-a)/n$,$x_j=a+jh$,
and $B_k$ denote the Bernoulli numbers, i.e., 
\begin{equation}
  \frac{x}{e^x-1} = \sum_{k=0}^{\infty} \frac{B_k x^k}{k!},
\end{equation}
and 
\begin{equation}
  E_m = h^{2m+2} \frac{B_{2m+2}}{(2m+2)!} (b-a) f^{(2m+2)}(\xi)
\end{equation}
for some $\xi \in (a,b)$.

If function $f(x)$ and all its derivatives are zero at the end points (as in a
smooth, bell-shaped function), then the second and third terms vanish and 
\begin{equation}
  \int_a^b f(x) dx = h\sum_{j=0}^n f(x_j) - E_m.
\end{equation}
Since $E_m$ is $O(h^{(2m+2)})$ for arbitrary $m$, we can say the error of the
trapezoidal rule decreases faster than any power of $h$. For a function defined
on $(-\infty,\infty)$, the Euler-Maclaurin formula still applies to the
resulting doubly infinite sum approximation, provided that the function and all
of its derivatives tend rapidly to zero for large positive and negative
arguments.

The double exponential quadrature 
\footnote{T. Ooura and M. Mori, A Robust Double Exponential Formula for
Fourier-type Integrals, Journal of Computational and Applied Mathematics, 112,
229-241 (1999).}
exploits this fact by converting the original
integral into an integral over the entire real axis whose integrand tends to
zero rapidly for large positive and negative arguments, i.e., let $x=\phi(t)$,
we have
\[
  \int_a^b f(x) dx = \int_{-\infty}^{\infty} f(\phi(t)) \phi'(t) dt,
\]
the trapezoidal rule is optimal if $f(\phi(t)) \phi'(t)$ tends to zero rapidly
when $t\to \pm \infty$.

%%%%%%%%%%%%%%%%%%%%%%%%%%%%%%%%%%%%
\section{$\int_{-1}^1 f(x) dx$}

For integral $I=\int_{-1}^1 f(x) dx$, we introduce the following change of
variables
\begin{equation}
  x = \phi(t) = \tanh \left( \frac{\pi}{2} \sinh t \right),
\end{equation}
this yields 
\begin{equation}
  \phi'(t) = \frac{ \frac{\pi}{2} \cosh t }{ \cosh^2 (\frac{\pi}{2} \sinh t) }
           \approx O \left( \exp( -\frac{\pi}{2} e^{|t|} ) \right)
           \qquad \text{as $|t|\to \infty$}.
\end{equation}
and it is optimal to approximate the integral by summation
\begin{equation}
  I_h = h \sum_{k=-\infty}^{\infty} 
    f\left( \tanh \left( \frac{\pi}{2} \sinh kh \right)  \right)
    \frac{ \frac{\pi}{2} \cosh kh }{ \cosh^2 (\frac{\pi}{2} \sinh kh) }
\end{equation}

%%%%%%%%%%%%%%%%%%%%%%%%%%%%%%%%%%%%
\section{$\int_{c}^{\infty} f(x) dx$}

For integral $I=\int_c^{\infty} f(x) dx$, we introduce 
\begin{equation}
  x = \phi(t) = c + \exp \left( \frac{\pi}{2} \sinh t \right),
\end{equation}
this yields
\begin{equation}
  \phi'(t) = 
    \exp \left( \frac{\pi}{2} \sinh t \right)  
    \frac{\pi}{2} \cosh t.
\end{equation}
Note first that as $t\to -\infty$, $\phi'(t)\to \exp(-\exp|t|)$.
And for the integral to converge, the integrand $f(x)$ must decrease at least
faster than $1/x$ as $x\to \infty$. We assume that $f(x)\sim x^{-1-\alpha}$ as
$x\to \infty$, where $\alpha>0$. Hence when $x\to \infty$ (i.e. $t\to \infty$),
\begin{align*}
  f(\phi(t)) \phi'(t)
    &= \left( c+ \exp \left( \frac{\pi}{2} \sinh t \right)  \right)^{-1-\alpha}
       \exp \left( \frac{\pi}{2} \sinh t \right)  \frac{\pi}{2} \cosh t 
       \notag \\
    &\sim \left( c+ \exp \left( \frac{\pi}{2} \sinh t \right)  \right)^{-\alpha}
          \frac{\pi}{2} \cosh t \notag \\
    &\sim \frac{\pi}{4} \exp \left( t-\frac{\pi \alpha}{4} e^t \right).
       \notag \\
\end{align*}
Thus we can again apply the trapezoidal rule 
\begin{equation}
  I_h = h \sum_{k=-\infty}^{\infty} 
    f\left( c+ \exp \left( \frac{\pi}{2} \sinh kh \right)  \right)
    \exp \left( \frac{\pi}{2} \sinh kh \right)  
    \frac{\pi}{2} \cosh kh 
\end{equation}
to approximate the integral.



